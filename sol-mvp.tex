% Created 2025-12-14 Sun 15:01
% Intended LaTeX compiler: lualatex
\documentclass[a4paper,11pt]{book}
\usepackage[utf8]{inputenc}
\usepackage[T1]{fontenc}
\usepackage{graphicx}
\usepackage{longtable}
\usepackage{wrapfig}
\usepackage{rotating}
\usepackage[normalem]{ulem}
\usepackage{amsmath}
\usepackage{amssymb}
\usepackage{capt-of}
\usepackage{hyperref}
\usepackage[paperwidth=6in,paperheight=9in,margin=1in]{geometry}
\setlength{\headheight}{25pt}
\setcounter{tocdepth}{2}
\usepackage{fontspec}
\setmainfont{Libertinus Serif}
\usepackage{polyglossia}
\setmainlanguage{french}
\usepackage{setspace}
\onehalfspacing
\usepackage{minted}
\usemintedstyle{monokai}
\setminted{linenos=true,breaklines=true,fontsize=\small,frame=lines,bgcolor=bg}
\usepackage[most]{tcolorbox}
\newtcolorbox{definition}{colback=blue!5,colframe=blue!50!black,title=Définition}
\newtcolorbox{methode}{colback=green!5,colframe=green!50!black,title=Méthode}
\usepackage{amsmath,amssymb}
\usepackage{graphicx}
\graphicspath{{./img/}}
\usepackage{enumitem}
\usepackage{tikz}
\usepackage{xcolor}
\definecolor{MyOrange}{HTML}{FF6600}
\usepackage{csquotes}
\MakeAutoQuote{«}{»}
\usepackage{fancyhdr}
\pagestyle{fancy}
\fancyhf{}
\fancyhead[LE,RO]{\thepage}
\fancyhead[RE]{\leftmark}
\fancyhead[LO]{\rightmark}
\renewcommand{\headrulewidth}{0.4pt}
\usepackage{hyperref}
\hypersetup{
colorlinks=true,
linkcolor=blue,
citecolor=green,
filecolor=magenta,
urlcolor=cyan,
pdftitle={Manipuler les vecteurs du plan},
pdfauthor={Laurent Garnier},
}
\usepackage[toc]{glossaries}
\makeglossaries
\usepackage{shellesc}  % Important pour minted avec LuaLaTeX
\author{Laurent Garnier}
\date{\today}
\title{Manipuler les vecteurs du plan (solutions)}
\hypersetup{
 pdfauthor={Laurent Garnier},
 pdftitle={Manipuler les vecteurs du plan (solutions)},
 pdfkeywords={},
 pdfsubject={Ce livre propose des exercices et explications pour manipuler les vecteurs},
 pdfcreator={Emacs 29.4 (Org mode 9.6.15)}, 
 pdflang={Fr}}
\begin{document}

\maketitle
\tableofcontents


\part{Solutions des exercices du contenu 1}
\label{sec:orgba065f2}
\chapter{Solution de l'exercice 1}
\label{sec:orga4fd32a}

Soient les points A, B, C tels que :
\begin{itemize}
\item \(\vec{u} = \overrightarrow{AB}\)
\item \(\vec{v} = \overrightarrow{AC}\)
\end{itemize}

Voir la figure :

\begin{center}
\includegraphics[width=.9\linewidth]{./img/ABCvect.png}
\end{center}

\begin{enumerate}
\item L'image du point B par la translation de vecteur
\[\vec{v} = \overrightarrow{AC}\]
est le point D tel que
\[\overrightarrow{BD} = \overrightarrow{AC}\]

Voir figure :
\begin{center}
\includegraphics[width=.9\linewidth]{./img/sol1q1.png}
\end{center}
\item L'image du point C par la translation de vecteur
\[\vec{u} = \overrightarrow{AB}\]
est le point E tel que
\[\overrightarrow{CE} = \overrightarrow{AB}\]

Voir figure :
\begin{center}
\includegraphics[width=.9\linewidth]{./img/sol1q2.png}
\end{center}
\item On peut en déduire que D = E car :
\begin{align*}
\overrightarrow{AD} &= \overrightarrow{AB} + \overrightarrow{BD} = \vec{u} + \vec{v}\\
\overrightarrow{AE} &= \overrightarrow{AC} + \overrightarrow{CE} = \vec{v} + \vec{u}\\
\overrightarrow{AD} &= \overrightarrow{AB} + \overrightarrow{AC} = \vec{u} + \vec{v}\\
\overrightarrow{AE} &= \overrightarrow{AC} + \overrightarrow{AB} = \vec{v} + \vec{u}\\
\overrightarrow{AD} &= \overrightarrow{AE}
\end{align*}
\item ABDC est un parallélogramme car (au choix) :
\begin{align*}
\overrightarrow{AB} &= \overrightarrow{CD}\\
\overrightarrow{AC} &= \overrightarrow{BD}
\end{align*}
Une seule égalité suffit.

Voir figure :
\begin{center}
\includegraphics[width=.9\linewidth]{./img/sol1q4.png}
\end{center}
\end{enumerate}


\chapter{Solution de l'exercice 2}
\label{sec:orgc5075d8}

\begin{enumerate}
\item Par construction, le point C est l'image de B par la translation
de vecteur
\[\vec{u} = \overrightarrow{AB}\]
donc
\[\overrightarrow{BC} = \vec{u}\]
Le vecteur \(\overrightarrow{BC}\) a pour
\begin{itemize}
\item direction : la droite (BC), qui est aussi la droite (AB).
\item sens : de B vers C, qui est aussi de A vers B.
\item norme : la longueur BC, qui est aussi la longueur AB
\end{itemize}

Voir figure :
\begin{center}
\includegraphics[width=.9\linewidth]{./img/sol2q1.png}
\end{center}
\item Le point B représente le milieu du segment [AC].
\item On a :
\begin{align*}
\vec{u} &= \overrightarrow{AB} \\
\overrightarrow{BC} &= \vec{u} \\
\overrightarrow{AC} &= \overrightarrow{AB} + \overrightarrow{BC} \\
\overrightarrow{AB} &= \overrightarrow{BC} = \dfrac{1}{2}\overrightarrow{AC}
\end{align*}
\end{enumerate}


\chapter{Solution programme 1}
\label{sec:org2737c6f}

\begin{minted}[]{python}
print("Un vecteur a 3 caractéristiques fondamentales : ")
entries = ["direction", "norme", "sens"]
definitions = [
    "la droite qui le porte.",
    "la distance entre origine et extrémité.",
    "de l'origine vers l'extrémité."
]
for i in range(len(entries)):
    e, d = entries[i], definitions[i]
    print(f"{i + 1}) {e.capitalize()} : {d}")
\end{minted}


\chapter{Solution du QCM d'auto-évaluation}
\label{sec:org5f65513}

Lorsqu'on parle du vecteur \(\overrightarrow{MM'}\) associé à
la translation qui transforme M en M'. 

\begin{enumerate}
\item Quelle est la direction ?
\begin{enumerate}[label=\alph*.]
\item Toute droite parallèle à l'axe des abscisses.
\item Uniquement la droite (MM').
\item Toute droite parallèle à l'axe des ordonnées.
\item Toute droite parallèle à la droite (MM').

(\textbf{Bonne réponse})
\end{enumerate}
\begin{enumerate}
\item Quelle est le sens ?
\begin{enumerate}[label=\alph*.]
\item De O à M.
\item De M à M'.

(\textbf{Bonne réponse})
\item De M' à M.
\item De M à O.
\end{enumerate}
\item Quelle est la norme ?
\begin{enumerate}[label=\alph*.]
\item La distance OM.
\item La distance OM'.
\item La distance MM'.

(\textbf{Bonne réponse})
\item La distance M'M.

(\textbf{Bonne réponse})
\end{enumerate}
\end{enumerate}
\end{enumerate}


\part{Solutions des exercices du contenu 2}
\label{sec:orgc73ef7d}

\chapter{Solution de l'exercice 3}
\label{sec:orge995701}

\begin{enumerate}
\item Puisque D est l'image du point C par la translation de vecteur
\[\vec{u} = \overrightarrow{AB}\]
alors :
\begin{align*}
\overrightarrow{CD} &= \vec{u}\\
\overrightarrow{AB} &= \overrightarrow{CD}
\end{align*}
Donc ABDC est un parallélogramme.

Voir figure :
\begin{center}
\includegraphics[width=.9\linewidth]{./img/sol3q1.png}
\end{center}
\item On sait que
\[\overrightarrow{DC} = -\overrightarrow{CD}\]
donc
\begin{align*}
\vec{w} &= \overrightarrow{AB} + \overrightarrow{DC} \\
\vec{w} &= \overrightarrow{AB} - \overrightarrow{CD} \\
\vec{w} &= \vec{u} - \vec{u}\\
\vec{w} &= \vec{0}
\end{align*}
Ainsi on remarque que \(\vec{w}\) est le vecteur nul.
\item Puisque E est l'image de D par la translation de vecteur
\[\vec{v} = \overrightarrow{AC}\]
alors
\[\overrightarrow{DE} = \vec{v}\]
donc
\begin{align*}
\overrightarrow{BD} + \overrightarrow{ED} &= \overrightarrow{AC} - \overrightarrow{DE}\\
\overrightarrow{BD} + \overrightarrow{ED} &= \vec{v} - \vec{v}\\
\overrightarrow{BD} + \overrightarrow{ED} &= \vec{0}
\end{align*}

Voir figure :
\begin{center}
\includegraphics[width=.9\linewidth]{./img/sol3q3.png}
\end{center}
\end{enumerate}


\chapter{Solution programme 2}
\label{sec:orgc20841d}

\begin{minted}[]{python}
msg = "Les vecteurs sont-ils égaux ?"
rep = "\n(O/N) "
msg += rep
egal = input(msg)
if egal.upper() == "O":
    msg = "Les vecteurs sont-ils alignés ?"
    align = input(msg)
    if align.upper() == "N":
        print("C'est un parallélogramme.")
    else:
        print("C'est le même vecteur.")
else:
    print("Les vecteurs ne sont pas colinéaires.")
\end{minted}

\chapter{Solution du QCM d'auto-évaluation}
\label{sec:orgb8bd255}

\begin{enumerate}
\item Deux vecteurs \(\vec{u}\) et \(\vec{v}\) sont égaux si :
\begin{enumerate}[label=\alph*.]
\item Ils ont la même direction.
\item Ils ont la même direction et le même sens.
\item Ils ont la même direction et la même norme.
\item Ils ont la même direction, le même sens et la même
norme.

(\textbf{Bonne réponse})
\end{enumerate}
\item On dit qu'un vecteur est nul si :
\begin{enumerate}[label=\alph*.]
\item Sa direction est horizontale.
\item Sa direction est verticale.
\item Il va dans un sens puis dans l'autre.
\item Sa norme vaut zéro.

(\textbf{Bonne réponse})
\end{enumerate}
\end{enumerate}



\part{Solutions des exercices du contenu 3}
\label{sec:org849ba79}

\chapter{Solution de l'exercice 4}
\label{sec:org086362d}

\begin{enumerate}
\item D'après la relation de Chasles :
\[\overrightarrow{AB} + \overrightarrow{BC} =
      \overrightarrow{AC}\]
\item D'après la relation de Chasles :
\[\overrightarrow{AB} + \overrightarrow{BD} =
      \overrightarrow{AD}\]
Or par construction :
\[\overrightarrow{AD} = \overrightarrow{AB} +
      \overrightarrow{AC}\]
Donc
\[\overrightarrow{BD} = \overrightarrow{AC}\]
Ainsi ABDC est un parallélogramme.

Voir figure :
\begin{center}
\includegraphics[width=.9\linewidth]{./img/sol4q2.png}
\end{center}
\item Dans un parallélogramme les diagonales se coupent en leur milieu
donc O est le milieu de [AD] et [BC]. Cette information est
inutile pour cette question mais elle le sera pour la question
suivante.

En utilisant Chasles ou la question précédente (avec la remarque
sur le milieu), on peut trouver plusieurs somme permettant
d'obtenir le vecteur \(\overrightarrow{AO}\) :
\begin{align*}
\overrightarrow{AO} &= \overrightarrow{AB} + \overrightarrow{BO}\\
\overrightarrow{AO} &= \overrightarrow{AC} + \overrightarrow{CO} \\
\overrightarrow{AO} &= \dfrac{1}{2}(\overrightarrow{AB} + \overrightarrow{AC})
\end{align*}
\item Avant de calculer cette somme vectorielle il faut réarranger
l'ordre des vecteurs et utiliser la remarque concernant les
milieux. En effet, puisque O est le milieu du segment [AD] alors
\[\overrightarrow{AO} = \overrightarrow{OD}\]
de même puisque O est le milieu du segment [BC] alors
\[\overrightarrow{BO} = \overrightarrow{OC}\]
Par conséquent :
\begin{align*}
\overrightarrow{AO} + \overrightarrow{BO} + \overrightarrow{CO} + \overrightarrow{DO} &= \overrightarrow{AO} + \overrightarrow{DO} + \overrightarrow{BO} + \overrightarrow{CO}\\
\overrightarrow{AO} + \overrightarrow{BO} + \overrightarrow{CO} + \overrightarrow{DO} &= \overrightarrow{AO} - \overrightarrow{OD} + \overrightarrow{BO} - \overrightarrow{OC}\\
\overrightarrow{AO} + \overrightarrow{BO} + \overrightarrow{CO} + \overrightarrow{DO} &= \overrightarrow{AO} - \overrightarrow{AO} + \overrightarrow{BO} - \overrightarrow{BO}\\
\overrightarrow{AO} + \overrightarrow{BO} + \overrightarrow{CO} + \overrightarrow{DO} &= \vec{0}
\end{align*}

Voir figure :
\begin{center}
\includegraphics[width=.9\linewidth]{./img/sol4q4.png}
\end{center}
\end{enumerate}


\chapter{Solution programme 3}
\label{sec:orgb9a84e6}

\begin{minted}[]{python}
abc = """
 C <-- B
 ^    /
 |   /
 |  /
 | /
 A
"""
somme = "Vecteur(A, B) + Vecteur(B, C)"
result = "Vecteur(A, C)"
chasles = somme + " = " + result
print(chasles)
print(abc)
\end{minted}

\chapter{Solution du QCM d'auto-évaluation}
\label{sec:org00344c0}

\begin{enumerate}
\item Ajouter deux vecteurs revient à :
\begin{enumerate}[label=\alph*.]
\item enchaîner deux translations successives

(\textbf{Bonne réponse})
\item faire une rotation
\item faire une symétrie
\item faire une homothétie
\end{enumerate}
\item La relation de Chasles :
\begin{enumerate}[label=\alph*.]
\item augmente la norme d'un vecteur
\item décompose un vecteur en sommes de vecteurs

(\textbf{Bonne réponse})
\item consiste à passer un coup de fil à Michel
\item revient à faire une transformation géométrique sur un
vecteur
\end{enumerate}
\end{enumerate}

\part{Solutions des exercices du contenu 4}
\label{sec:org11e9aea}

\chapter{Solution de l'exercice 5}
\label{sec:orga92f719}

\begin{enumerate}
\item Puisque ABCD est un carré alors les droites (AB) et (AD) sont
orthogonales et les longueurs AB et AD sont égales. Par
conséquent les vecteurs sont associés sont orthogonaux et de
même norme. Ainsi
\[(\vec{u} , \vec{v}) = (\overrightarrow{AB} ,
      \overrightarrow{AD})\]
est bien une base orthonormée.

Voir figure :
\begin{center}
\includegraphics[width=.9\linewidth]{./img/sol5q1.png}
\end{center}
\item Pour déterminer les coordonnées des points dans cette base il
faut exprimer les vecteurs partant de l'origine du repère comme
combinaisons linéaires des vecteurs de la base. Tout d'abord il
faut remarquer que l'origine du repère est le point A car il
s'agit de l'origine de chacun des vecteurs de la base. On
obtient ainsi :
\begin{align*}
\overrightarrow{AA} &= 0\times\vec{u} + 0\times\vec{v}\Rightarrow A(0 ; 0)\\
\overrightarrow{AB} &= 1\times\vec{u} + 0\times\vec{v}\Rightarrow B(1 ; 0)\\
\overrightarrow{AC} &= 1\times\vec{u} + 1\times\vec{v}\Rightarrow C(1 ; 1)\\
\overrightarrow{AD} &= 0\times\vec{u} + 1\times\vec{v}\Rightarrow D(0 ; 1)
\end{align*}
\item Calculons la norme du vecteur \(\overrightarrow{AC}\) :
\begin{align*}
\overrightarrow{AC} &= \sqrt{1^2 + 1^2}\\
\overrightarrow{AC} &= \sqrt{2}
\end{align*}
\end{enumerate}

\chapter{Solution de l'exercice 5 bis}
\label{sec:org1d088c1}

On se place dans le plan muni repère orthonormé \((O ; \vec{i} ;
    \vec{j})\). Avec :
\begin{align*}
\vec{i}&\begin{pmatrix}1\\0\end{pmatrix} && \vec{j}\begin{pmatrix}0\\1\end{pmatrix}
\end{align*}

Calculons les normes des vecteurs suivants :

\begin{enumerate}
\item \begin{align*}
 n_1 &= \lvert\lvert \vec{i} + \vec{j}\rvert\rvert\\
 n_1 &= \lvert\lvert \begin{pmatrix}1\\1\end{pmatrix}\rvert\rvert\\
 n_1 &= \sqrt{1^2 + 1^2}\\
 &\Rightarrow n_1 = \sqrt{2}
\end{align*}
Voir figure :
\begin{center}
\includegraphics[width=.9\linewidth]{./img/sol5bisq1.png}
\end{center}
\item \begin{align*}
 n_2 &= \lvert\lvert \vec{i} - \vec{j}\rvert\rvert\\
 n_2 &= \lvert\lvert \begin{pmatrix}1\\ -1 \end{pmatrix}\rvert\rvert\\
 n_2 &= \sqrt{1^2 + (-1)^2}\\
 &\Rightarrow n_2 = \sqrt{2}
\end{align*}
Voir figure :
\begin{center}
\includegraphics[width=.9\linewidth]{./img/sol5bisq2.png}
\end{center}
\item \begin{align*}
 n_3 &= \lvert\lvert 3\vec{i}\rvert\rvert\\
 n_3 &= 3\lvert\lvert \vec{i}\rvert\rvert\\
 &\Rightarrow n_3 = 3
\end{align*}
Voir figure :
\begin{center}
\includegraphics[width=.9\linewidth]{./img/sol5bisq3.png}
\end{center}
\item \begin{align*}
 n_4 &= \lvert\lvert 3\vec{i} + 4\vec{j}\rvert\rvert\\
 n_4 &= \lvert\lvert \begin{pmatrix}3\\4\end{pmatrix}\rvert\rvert\\
 n_4 &= \sqrt{3^2 + 4^2}\\
 &\Rightarrow n_4 = 5
\end{align*}
Voir figure :
\begin{center}
\includegraphics[width=.9\linewidth]{./img/sol5bisq4.png}
\end{center}
\item \begin{align*}
 n_5 &= \lvert\lvert -5\vec{j}\rvert\rvert\\
 n_5 &= |-5|\lvert\lvert \vec{j}\rvert\rvert\\
 &\Rightarrow n_5 = 5
\end{align*}
Voir figure :
\begin{center}
\includegraphics[width=.9\linewidth]{./img/sol5bisq5.png}
\end{center}
\end{enumerate}


\chapter{Solution de l'exercice 5 ter}
\label{sec:org2d7f281}

On se place dans le plan muni repère orthonormé \((O ; \vec{i} ;
    \vec{j})\). Avec :
\begin{align*}
\vec{i}&\begin{pmatrix}1\\0\end{pmatrix} && \vec{j}\begin{pmatrix}0\\1\end{pmatrix}
\end{align*}

Calculons et comparons les normes ajoutées séparément
\[s = \lvert\lvert\vec{u}\rvert\rvert + \lvert\lvert\vec{v}\rvert\rvert\]

avec celles des vecteurs sommes

\[e = \lvert\lvert\vec{u} + \vec{v}\rvert\rvert\]

\begin{enumerate}
\item \[(\vec{u}, \vec{v}) = (\vec{i}, \vec{j})\]
\begin{align*}
\lvert\lvert\vec{u}\rvert\rvert &= \lvert\lvert\vec{v}\rvert\rvert = 1\\
s &= 1 + 1 = 2\\
e &= \lvert\lvert\vec{u} + \vec{v}\rvert\rvert\\
e &= \lvert\lvert\begin{pmatrix}1 \\ 1\end{pmatrix}\rvert\rvert\\
e &= \sqrt{1^2 + 1^2} = \sqrt{2}\\
&\Rightarrow s = 2 > e = \sqrt{2}
\end{align*}
Voir figure :
\begin{center}
\includegraphics[width=.9\linewidth]{./img/sol5terq1.png}
\end{center}
\item \[(\vec{u}, \vec{v}) = (\vec{i} + \vec{j}, \vec{i} - \vec{j})\]
\begin{align*}
\lvert\lvert\vec{u}\rvert\rvert &= \lvert\lvert\vec{v}\rvert\rvert = \sqrt{2}\\
s &= \sqrt{2} + \sqrt{2} = 2\sqrt{2}\\
e &= \lvert\lvert\vec{u} + \vec{v}\rvert\rvert\\
e &= \lvert\lvert 2\vec{i}\rvert\rvert = 2\lvert\lvert\vec{i}\rvert\rvert\\
e &= 2\\
&\Rightarrow s = 2\sqrt{2} > e = 2
\end{align*}
Voir figure :
\begin{center}
\includegraphics[width=.9\linewidth]{./img/sol5terq2.png}
\end{center}
\item \[ (\vec{u}, \vec{v}) = (a\vec{i} + b\vec{j}, c\vec{i} + d\vec{j}) \]
\begin{align*}
\lvert\lvert\vec{u}\rvert\rvert &= \sqrt{a^2 + b^2}\\
\lvert\lvert\vec{v}\rvert\rvert &= \sqrt{c^2 + d^2}\\
s &= \lvert\lvert\vec{u}\rvert\rvert + \lvert\lvert\vec{v}\rvert\rvert = \sqrt{a^2 + b^2} + \sqrt{c^2 + d^2}\\
s^2 &= a^2 + b^2 + c^2 + d^2 + 2\sqrt{(a^2 + b^2)(c^2 + d^2)}\\
e &= \lvert\lvert\vec{u} + \vec{v}\rvert\rvert = \lvert\lvert (a + c)\vec{i} + (b + d)\vec{j}\rvert\rvert \\
e &= \sqrt{(a + c)^2 + (b + d)^2} \\
e &= \sqrt{a^2 + b^2 + c^2 + d^2 + 2(ac + bd)}\\
e^2 &= a^2 + b^2 + c^2 + d^2 + 2(ac + bd)\\
s^2 - e^2 &= 2\left(\sqrt{(a^2 + b^2)(c^2 + d^2)} - (ac + bd)\right)\\
(a^2 + b^2)(c^2 + d^2) &= (ac)^2 + (bd)^2 + (ad)^2 + (bc)^2\\
(ac + bd)^2 &= (ac)^2 + (bd)^2 + 2abcd\\
(a^2 + b^2)(c^2 + d^2) &- (ac + bd)^2 = (ad)^2 + (bc)^2 - 2abcd\\
(a^2 + b^2)(c^2 + d^2) &- (ac + bd)^2 = (ad - bc)^2 \geqslant 0\\
&\Rightarrow s^2 - e^2 \geqslant 0\\
&\Rightarrow s^2 \geqslant e^2\\
&\Rightarrow s \geqslant e
\end{align*}
\end{enumerate}


\chapter{Solution programme 4}
\label{sec:org77740dd}

\begin{minted}[]{python}
base = """On dit que les vecteurs u et v forment une base s'ils ne sont
pas colinéaires.
Algérbiquement : il n'existe aucun réel k tel que u = kv.
Géométriquement : leurs directions sont des droites sécantes.
Rapidement : det(u, v) != 0.\n
"""

draw_base = """
    ^
   / v
  /
 /       (u, v) forme une base
x---------------------------->
          u
"""

orthonormal = """Ortho vient du grec pour dire droit donc ici qui forme un
angle droit (directions orthogonales). Normal pour norme, ici de même
norme.\n
"""

draw_orthornormal_base = """
      ^
      |    i et j sont orthogonaux
      |    ||i|| = ||j|| ont même norme
      |    
      ^    (i, j) forme une base orthonormée
      | j
------0->---------------------------------->
       i
"""

relation = """Si (i, j) est une base orthonormale alors tout vecteur
u = a * i + b * j a pour coordonnées (a, b).\n
Par exemple le vecteur horizontal v_1 de coordonnées (2, 0) s'écrit :
v_1 = 2 * i + 0 * j.\n
Par exemple le vecteur vertical v_2 de coordonnées (0, 3) s'écrit :
v_2 = 0 * i + 3 * j.\n
Par exemple le vecteur diagonal v_3 de coordonnées (4, 4) s'écrit :
v_3 = 4 * i + 4 * j.\n
Si le vecteurs v_4 = 2 * v_1 + v_2 alors v_4 = 4 * i + 3 * j
a pour coordonnées : (4, 3).\n
"""

v1 = """
^
|
| --> v_1 = 2 * i
| -> i
|
0->-------->
 i
"""

v2 = """
  ^       ^
  |       |
  |    ^  |
  |    |  |
  ^    j  v_2 = 3 * j
j |   
  0----------------->

"""



v3 = """
^  v_3 = 4 * i + 4 * j    
|
8      ^ (7, 8)
| v_3 /|
6    / | 4 * j 
|   /  |
4--x---> (7, 4)  
|  |4i |        
2  |   |        
|  |   |        
0--3---7----->
"""

v4 = """
  ^      v_4 = 4 * i + 3 * j
  |     
  ^    /^    
  |   / |
  |  /  | v_2 = 3 * j
  ^ /   ^
j |/    |
  0->--->----------------->
   i
    2 * v_1 = 4 * i
"""

vectors = [v1, v2, v3, v4]

norme = """Un vecteur u = a * i + b * j de coordonnées (a, b) a pour norme
||u|| = sqrt{a^2 + b^2} = (a^2 + b^2) ** 0.5\n
ou dit autrement ||u||^2 = a^2 + b^2\n
Par exemple le vecteur horizontal v_1 de coordonnées (2, 0) a pour norme :
||v_1|| = sqrt{2^2 + 0^2} = sqrt{2^2} = 2.\n
Par exemple le vecteur vertical v_2 de coordonnées (0, 3) a pour norme :
||v_2|| = sqrt{0^2 + 3^2} = sqrt{3^2} = 3.\n 
Par exemple le vecteur diagonal v_3 de coordonnées (5, 5) a pour norme :
||v_3|| = sqrt{5^2 + 5^2} = sqrt{2 * 5^2} = 5 * sqrt{2}.\n
Si le vecteur v_4 = 2 * v_1 + v_2 alors v_4 = 4 * i + 3 * j a pour norme :
||v_4|| = sqrt{4^2 + 3^2} = sqrt{16 + 9} = sqrt{25} = sqrt{5^2} = 5.\n
"""


menu = """
MENU
1) Définition d'une base orthornomée.
2) Relation vectorielle entre coordonnées et vecteurs de la base.
3) Formule de calcul de la norme.
0) Quitter.
"""
titles = [
    "1) Définition d'une base orthonormée",
    "2) Relation vectorielle coordonnées base",
    "3) Norme d'un vecteur"
]
underlines = ["-" * len(t) for t in titles]
continuer = True
while continuer:
    choix = int(input(menu + "\nVotre choix : "))
    if choix == 1:
        print()
        title1 = titles[choix - 1]
        underline1 = underlines[choix - 1]
        print(title1)
        print(underline1)
        print()
        print("Une base orthonormée est une base de vecteurs orthonormaux.")
        print("base :", base)
        draw = int(input("Taper 1 afficher le dessin\n"))
        print(draw_base)
        print("orthonormal :", orthonormal)
        draw = int(input("Taper 1 afficher le dessin\n"))
        print(draw_orthornormal_base)


    elif choix == 2:
        print()
        title2 = titles[choix - 1]
        underline2 = underlines[choix - 1]
        print(title2)
        print(underline2)
        print()
        print(relation)

        for i in range(len(vectors)):
            msg = f"Taper {i + 1} pour afficher le vecteur v_{i + 1} : "
            dessin = int(input(msg))
            if dessin == i + 1:
                print(vectors[i])

    elif choix == 3:
        print()
        title3 = titles[choix - 1]
        underline3 = underlines[choix - 1]
        print(title3)
        print(underline3)
        print()
        print(norme)

        def ask_coord(axe, c):
            msg = " du vecteur dont vous voulez calculer la norme "
            msg = f"{axe} {msg} {c} = "
            return msg

        x = float(input(ask_coord("Abscisse", "x")))
        y = float(input(ask_coord("Ordonnée", "y")))
        n = (x ** 2 + y ** 2) ** 0.5
        print(f"Norme à 2 décimales près n = {n:.2f}")

    elif choix == 0:
        continuer = False

\end{minted}


\chapter{Solution du QCM d'auto-évaluation}
\label{sec:org43dbe99}

\begin{enumerate}
\item Une base orthonormée du plan est :
\begin{enumerate}[label=\alph*.]
\item une station de lancement de fusée
\item un couple de vecteurs ayant des directions distinctes
\item un couple de vecteurs ayant des directions orthogonales et
la même norme

(\textbf{Bonne réponse})
\item un couple de vecteurs ayant la même direction et la même
norme
\end{enumerate}
\item Quelles sont les coordonnées d'un vecteur \(\vec{u}\) dans une
base orthonormée \((\vec{i} , \vec{j})\) ?
\begin{enumerate}[label=\alph*.]
\item Le couple de nombres \((a ; b)\) tel que \(\vec{u} =
          a\vec{i} + b\vec{j}\).

(\textbf{Bonne réponse})
\item Les coordonnées du point M obtenu par la translation de
vecteur \(\vec{u}\) à partir du point O.
\item La somme de celles des vecteurs \(\vec{i}\) et \(\vec{j}\).
\item Les coefficients de toute combinaison linéaire des vecteurs
de la base.
\end{enumerate}
\item Considérons le vecteur \(\vec{u} = a\vec{i} + b\vec{j}\) alors
l'expression de sa norme est :
\begin{enumerate}[label=\alph*.]
\item \(\lvert\lvert\vec{u}\rvert\rvert = a + b\)
\item \(\lvert\lvert\vec{u}\rvert\rvert = a^2 - b^2\)
\item \(\lvert\lvert\vec{u}\rvert\rvert = a^2 + b^2\)
\item \(\lvert\lvert\vec{u}\rvert\rvert = \sqrt{a^2 + b^2}\)

(\textbf{Bonne réponse})
\end{enumerate}
\end{enumerate}

\part{Solutions des exercices du contenu 5}
\label{sec:orgc338db4}

\chapter{Solution de l'exercice 6}
\label{sec:org83518cf}

\begin{enumerate}
\item Les coordonnées du vecteur \(\overrightarrow{OA}\) sont les
mêmes que celles du point A d'où la relation :
\[\overrightarrow{OA} = x_A\vec{i} + y_A\vec{j}\]
\item Les coordonnées du vecteur \(\overrightarrow{OB}\) sont les
mêmes que celles du point B d'où la relation :
\[\overrightarrow{OB} = x_B\vec{i} + y_B\vec{j}\]
\item Utilisons la relation de Chasles :
\begin{align*}
\overrightarrow{AB} &= \overrightarrow{AO} + \overrightarrow{OB}\\
\overrightarrow{AB} &= \overrightarrow{OB} - \overrightarrow{OA}
\end{align*}
\item On rassemble les résultats obtenus aux questions précédentes :
\begin{align*}
\overrightarrow{AB} &= \overrightarrow{OB} - \overrightarrow{OA}\\
\overrightarrow{AB} &= x_B\vec{i} + y_B\vec{j} - (x_A\vec{i} + y_A\vec{j})\\
\overrightarrow{AB} &= (x_B - x_A)\vec{i} + (y_B - y_A)\vec{j}
\end{align*}
\item Ainsi on obtient :
\[\overrightarrow{AB}\begin{pmatrix}x_B - x_A\\ y_B - y_A\end{pmatrix}\]
\end{enumerate}


\chapter{Solution du programme 5}
\label{sec:org7d5f8d9}

\begin{minted}[]{python}
print("Coordonnées du vecteur AB")
print("x_{AB} = x_B - x_A")
print("y_{AB} = y_B - y_A")
x_A = float(input("x_A = "))
y_A = float(input("y_A = "))
x_B = float(input("x_B = "))
y_B = float(input("y_B = "))
x_AB = x_B - x_A
y_AB = y_B - y_A
print(f"x_AB = {x_AB:.2f}")
print(f"y_AB = {y_AB:.2f}")
\end{minted}



\chapter{Solution du QCM d'auto-évaluation}
\label{sec:orgf6576ba}

On considère le vecteur \(\overrightarrow{AB}\) dans le plan muni
du repère orthonormé \((O ; \vec{i}, \vec{j})\).

\begin{enumerate}
\item En utilisant Chasles on peut écrire :
\begin{enumerate}[label=\alph*.]
\item \(\overrightarrow{AB} = \overrightarrow{OA} +
          \overrightarrow{OB}\)
\item \(\overrightarrow{AB} = \overrightarrow{OA} - \overrightarrow{OB}\)
\item \(\overrightarrow{AB} = \overrightarrow{AO} +
          \overrightarrow{BO}\)
\item \(\overrightarrow{AB} = \overrightarrow{AO} +
          \overrightarrow{OB}\)

(\textbf{Bonne réponse})
\end{enumerate}
\item En utilisant les coordonnées des points A et B on a :
\begin{enumerate}[label=\alph*.]
\item \(\overrightarrow{AB}\begin{pmatrix}x_A + x_B\\y_A +
          y_B\end{pmatrix}\)
\item \(\overrightarrow{AB}\begin{pmatrix}x_A - x_B\\y_A -
          y_B\end{pmatrix}\)
\item \(\overrightarrow{AB}\begin{pmatrix}x_B - x_A\\y_B -
          y_A\end{pmatrix}\)

(\textbf{Bonne réponse})
\item \(\overrightarrow{AB}\begin{pmatrix}x_A \times x_B\\y_A
          \times y_B\end{pmatrix}\)
\end{enumerate}
\end{enumerate}



\part{Solutions des exercices du contenu 6}
\label{sec:org2595ba5}

\chapter{Solution de l'exercice 7}
\label{sec:org1e76837}

\begin{enumerate}
\item Voir figure :
\begin{center}
\includegraphics[width=.9\linewidth]{./img/sol7q1.png}
\end{center}
\item On peut voir sur la figure que les points A et B sont sur l'axe
des abscisses donc les vecteurs \(\overrightarrow{OA}\) et
\(\overrightarrow{OB}\) sont colinéaires. Concrètement :
\[\overrightarrow{OB} = \dfrac{3}{2}\overrightarrow{OA}\]
\item De même on peut voir sur la figure que les points C et D sont sur l'axe
des ordonnées donc les vecteurs \(\overrightarrow{OC}\) et
\(\overrightarrow{OD}\) sont colinéaires. Concrètement :
\[\overrightarrow{OD} = \dfrac{3}{2}\overrightarrow{OC}\]
\item Voir figure :
\begin{center}
\includegraphics[width=.9\linewidth]{./img/sol7q4.png}
\end{center}
\item D'une part on a :
\begin{align*}
\overrightarrow{DF} &= \overrightarrow{DO} + \overrightarrow{OF}\\
\overrightarrow{DF} &= \vec{v} - 3\vec{j}\\
\overrightarrow{DF} &= 2\vec{i} + 3\vec{j} - 3\vec{j}\\
\overrightarrow{DF} &= 2\vec{i}
\end{align*}

D'autre part on a :
\begin{align*}
\overrightarrow{CE} &= \overrightarrow{CO} + \overrightarrow{OE}\\
\overrightarrow{CE} &= \vec{u} - 2\vec{j}\\
\overrightarrow{CE} &= 3\vec{i} + 2\vec{j} - 2\vec{j}\\
\overrightarrow{CE} &= 3\vec{i}
\end{align*}

Ainsi :
\[\overrightarrow{CE} = \dfrac{3}{2}\overrightarrow{DF}\]
\end{enumerate}

\chapter{Solution du programme 6}
\label{sec:org30e9c7e}

\begin{minted}[]{python}
x_u = float(input("Abscisse du 1er vecteur = "))
y_u = float(input("Ordonnée du 1er vecteur = "))
x_v = float(input("Abscisse du 2e vecteur = "))
y_v = float(input("Ordonnée du 2e vecteur = "))
d = x_u * y_v - x_v * y_u
if d == 0:
    print("Les vecteurs sont colinéaires.")
    k = x_v / x_u
    print(f"Vecteur 2 = {k} * Vecteur 1")
else:
    print("Les vecteurs ne sont pas colinéaires.")
    print("Ils forment donc une base.")
\end{minted}


\chapter{Solution du QCM d'auto-évaluation}
\label{sec:orgad9603c}

\begin{enumerate}
\item Si on multiplie un vecteur par un nombre réel supérieur à 1
alors :
\begin{enumerate}[label=\alph*.]
\item Le vecteur change de direction.
\item Le vecteur augmente sa norme.

(\textbf{Bonne réponse})
\item Le vecteur change de sens.
\item Le vecteur reste identique.
\end{enumerate}
\item Si on multiplie un vecteur par un nombre réel inférieur à -1
alors :
\begin{enumerate}[label=\alph*.]
\item Le vecteur change de direction.
\item Le vecteur augmente sa norme.

(\textbf{Bonne réponse})
\item Le vecteur change de sens.

(\textbf{Bonne réponse})
\item Le vecteur reste identique.
\end{enumerate}
\item Si on multiplie un vecteur par un nombre réel supérieur à -1 et
inférieur à 1 alors :
\begin{enumerate}[label=\alph*.]
\item Le vecteur change de direction.
\item Le vecteur diminue sa norme.

(\textbf{Bonne réponse})
\item Le vecteur change de sens.
\item Le vecteur reste identique.
\end{enumerate}
\item Si on multiplie un vecteur par un nombre réel alors :
\begin{enumerate}[label=\alph*.]
\item Le vecteur obtenu n'est pas colinéaire au vecteur initial.
\item Le vecteur obtenu est colinéaire au vecteur initial.

(\textbf{Bonne réponse})
\end{enumerate}
\end{enumerate}



\part{Solutions des exercices du contenu 7}
\label{sec:org1d00b6c}

\chapter{Solution de l'exercice 8}
\label{sec:org0341c67}

On reprend la configuration finale de l'exercice 7.

Voir figure :

\begin{center}
\includegraphics[width=.9\linewidth]{./img/sol7q4.png}
\end{center}

\begin{enumerate}
\item Calculs de déterminants :
\begin{align*}
d_1 &= det(\vec{i}, \vec{j}) = \begin{vmatrix}1&0\\0&1\end{vmatrix} = 1\times 1 - 0\times 0 = 1\\
d_2 &= det(\vec{u}, \vec{v}) = \begin{vmatrix}3&2\\2&3\end{vmatrix} = 3\times 3 - 2\times 2 = 5\\
d_3 &= det(\overrightarrow{OA}, \overrightarrow{OB}) = \begin{vmatrix}2&3\\0&0\end{vmatrix} = 2\times 0 - 0\times 3 = 0\\
d_4 &= det(\overrightarrow{OC}, \overrightarrow{OD}) = \begin{vmatrix}0&0\\2&3\end{vmatrix} = 0\times 3 - 2\times 0 = 0 \\
d_5 &= det(\overrightarrow{OA}, \overrightarrow{OC}) = \begin{vmatrix}2&0\\0&2\end{vmatrix} = 2\times 2 - 0\times 0 = 4 \\
d_6 &= det(\overrightarrow{OB}, \overrightarrow{OD}) = \begin{vmatrix}3&0\\0&3\end{vmatrix} = 3\times 3 - 0\times 0 = 9
\end{align*}
\item En utilisant le déterminant montrons que les vecteurs
\(\overrightarrow{DF}\) et \(\overrightarrow{CE}\) sont
colinéaires :
\begin{align*}
 det(\overrightarrow{DF}, \overrightarrow{CE}) &= \begin{vmatrix}2&3\\0&0\end{vmatrix}\\
 det(\overrightarrow{DF}, \overrightarrow{CE}) &= 2\times 0 - 0\times 3\\
 det(\overrightarrow{DF}, \overrightarrow{CE}) &= 0
\end{align*}
 Or \(\lvert\lvert\overrightarrow{DF}\rvert\rvert = 2\) et
\(\lvert\lvert\overrightarrow{CE}\rvert\rvert = 3\).

On en déduit que le quadrilatère DCEF est un trapèze.
\item Faisons de même pour \(\overrightarrow{AF}\) et
\(\overrightarrow{BE}\) et le quadrilatère ABEF.
\begin{align*}
  det(\overrightarrow{AF}, \overrightarrow{BE}) &= \begin{vmatrix}0&0\\3&2\end{vmatrix}\\
  det(\overrightarrow{AF}, \overrightarrow{BE}) &= 0\times 2 - 3\times 0\\
  det(\overrightarrow{AF}, \overrightarrow{BE}) &= 0
 \end{align*}
 Or \(\lvert\lvert\overrightarrow{AF}\rvert\rvert = 3\) et
\(\lvert\lvert\overrightarrow{BE}\rvert\rvert = 2\).

On en déduit que le quadrilatère ABFE est un trapèze.
\item Calculons les normes des vecteurs \(\overrightarrow{IF}\) et
\(\overrightarrow{JE}\) :
\begin{align*}
\lvert\lvert\overrightarrow{IF}\rvert\rvert &= \sqrt{(x_F - x_I)^2 + (y_F - y_I)^2}\\
\lvert\lvert\overrightarrow{IF}\rvert\rvert &= \sqrt{(2 - 1)^2 + (3 - 0)^2}\\
\lvert\lvert\overrightarrow{IF}\rvert\rvert &= \sqrt{10}\\
\lvert\lvert\overrightarrow{JE}\rvert\rvert &= \sqrt{(x_E - x_J)^2 + (y_E - y_J)^2}\\
\lvert\lvert\overrightarrow{JE}\rvert\rvert &= \sqrt{(3 - 0)^2 + (2 - 1)^2}\\
\lvert\lvert\overrightarrow{JE}\rvert\rvert &= \sqrt{10}
\end{align*}
\item Comparons les vecteurs \(\overrightarrow{IJ}\) et
\(\overrightarrow{EF}\).
\begin{align*}
\overrightarrow{IJ} &= \overrightarrow{IO} + \overrightarrow{OJ}\\
\overrightarrow{IJ} &= \vec{j} - \vec{i}\\
\overrightarrow{EF} &= \overrightarrow{EO} + \overrightarrow{OF}\\
\overrightarrow{EF} &= \vec{v} - \vec{u}\\
\overrightarrow{EF} &= 2\vec{i} + 3\vec{j} - (3\vec{i} + 2\vec{j})\\
\overrightarrow{EF} &= \vec{j} - \vec{i}
\end{align*}
Ainsi \[\overrightarrow{IJ} = \overrightarrow{EF}\]

 Le quadrilatère IEFJ est donc un parallélogramme.
 Or d'après la question précédente ses diagonales [IF] et [JE] sont
égales.
Par conséquent IEFJ est un rectangle.
\item Puisque G l'intersection des segments [IF] et [JE] et que ce
sont les diagonales d'un rectangle alors G est leur milieu.
Déterminons ses coordonnées :
\begin{align*}
G&\begin{pmatrix}\frac{x_I + x_F}{2}\\\frac{y_I + y_F}{2}\end{pmatrix}\\
G&\begin{pmatrix}\frac{3}{2}\\\frac{3}{2}\end{pmatrix}
\end{align*}
Le point H est tel que \(\overrightarrow{OG} =
      \overrightarrow{GH}\) alors en appliquant la relation de Chasles :
\[\overrightarrow{OH} = 2\overrightarrow{OG}\]
Ainsi on obtient les coordonnées de H en doublant celles de G,
H(3 ; 3). 
Étudions la nature du quadrilatère OBHD :
\begin{align*}
\overrightarrow{OH}&= 3\vec{i} + 3\vec{j}\\
\overrightarrow{OB}&= 3\vec{i}\\
\overrightarrow{OD}&= 3\vec{j}\\
\overrightarrow{OH}&= \overrightarrow{OB} + \overrightarrow{OD}
\end{align*}
On vient de prouver que OBHD est un carré. Pourquoi ? Parce que
\(\overrightarrow{OH}\) est la somme de deux vecteurs
orthogonaux de même norme.
\item Le triangle OFE est isocèle en O car les vecteurs \(\vec{u}\) et
\(\vec{v}\) ont même norme :
\begin{align*}
\lvert\lvert\vec{u}\rvert\rvert &= \sqrt{3^2 + 2^2} = \sqrt{13}\\
\lvert\lvert\vec{v}\rvert\rvert &= \sqrt{3^2 + 2^2} = \sqrt{13}
\end{align*}
\item D'après la question 6 on sait que OBHD est un carré et que G est
le milieu de la diagonale OH. Par conséquent G est aussi le
milieu de la diagonale BD. Ainsi l'image de G par la translation
de vecteur \(\overrightarrow{BG}\) est D.
\item Le point K tel que \(\overrightarrow{BK} = \vec{j}\) a pour
coordonnées K(3 ; 1).
\item Le point L tel que \(\overrightarrow{DL} = \vec{i}\) a pour
coordonnées L(1 ; 3).
\end{enumerate}


\begin{center}
\includegraphics[width=.9\linewidth]{./img/sol8q10.png}
\end{center}


\chapter{Solution du programme 7}
\label{sec:org5456b37}

\begin{minted}[]{python}
def abc_aligned():
  points = []

  for i in range(3):
    x = float(input(f"Abscisse du {i + 1}e point = "))
    y = float(input(f"Ordonnée du {i + 1}e point = "))
    points.append((x, y))

  x_p1p2 = points[1][0] - points[0][0]
  y_p1p2 = points[1][1] - points[0][1]
  x_p1p3 = points[2][0] - points[0][0]
  y_p1p3 = points[2][1] - points[0][1]

  det = x_p1p2 * y_p1p3 - x_p1p3 * y_p1p2

  if det == 0:
    return True
  else:
    return False 


# Test
if abc_aligned():
  print("Les points sont alignés.")
else:
  print("Les points ne sont pas alignés.")

\end{minted}



\chapter{Solution du QCM d'auto-évaluation}
\label{sec:orge7710ee}

\begin{enumerate}
\item Considérons les vecteurs
\[\vec{u}_1\begin{pmatrix}x_1\\y_1\end{pmatrix}\] et
\[\vec{u}_2\begin{pmatrix}x_2\\y_2\end{pmatrix}\] alors :
\begin{enumerate}[label=\alph*.]
\item \[det(\vec{u}_1 , \vec{u}_2) = \begin{vmatrix}x_1&x_2\\y_1&y_2\end{vmatrix}
          = x_1x_2 + y_1y_2\]
\item \[det(\vec{u}_1 , \vec{u}_2) = \begin{vmatrix}x_1&x_2\\y_1&y_2\end{vmatrix}
          = x_1x_2 - y_1y_2\]
\item \[det(\vec{u}_1 , \vec{u}_2) = \begin{vmatrix}x_1&x_2\\y_1&y_2\end{vmatrix}
          = x_1y_2 + y_1x_2\]
\item \[det(\vec{u}_1 , \vec{u}_2) = \begin{vmatrix}x_1&x_2\\y_1&y_2\end{vmatrix}
          = x_1y_2 - y_1x_2\]

(\textbf{Bonne réponse})
\end{enumerate}
\item Considérons les mêmes vecteurs que précédemment.

On dira que \(\vec{u}_1\) et \(\vec{u}_2\) sont colinéaires
si :

\begin{enumerate}[label=\alph*.]
\item \(det(\vec{u}_1 , \vec{u}_2) = 1\)
\item \(det(\vec{u}_1 , \vec{u}_2) = 0\)

(\textbf{Bonne réponse})
\item il existe un réel \(k\) tel que \(x_1 = kx_2\) et \(y_1 =
          ky_2\)

(\textbf{Bonne réponse})
\item \(\dfrac{x_1}{x_2} = \dfrac{y_1}{y_2}\)

(\textbf{Bonne réponse})
\end{enumerate}
\end{enumerate}

\part{Solutions des exercices de la \(Ca_1\)}
\label{sec:orgca753b9}
\chapter{Solution de l'exercice 9}
\label{sec:org9630cb7}

On considère le triangle ABC représenté sur la figure avec le
 quadrillage :

\begin{center}
\includegraphics[width=.9\linewidth]{./img/ABCtrig.png}
\end{center}

\begin{enumerate}
\item Pour construire le point D tel que \(\overrightarrow{BD} =
       \overrightarrow{AB}\) il faut d'abord comprendre que D est le
symétrique de A par rapport à B. Ou, dit autrement, B est le
milieu de [AD].

Voir figure :
\begin{center}
\includegraphics[width=.9\linewidth]{./img/sol9q1.png}
\end{center}

\item Pour construire le point E tel que \(\overrightarrow{CE} =
       \overrightarrow{AB}\) il faut bien comprendre que l'on complète
le triangle pour obtenir un parallélogramme.

Voir figure :
\begin{center}
\includegraphics[width=.9\linewidth]{./img/sol9q2.png}
\end{center}

\item Utilisons la relation de Chasles pour décomposer le vecteur
\(\overrightarrow{BE}\) : 
\begin{align*}
\overrightarrow{BE} &= \overrightarrow{BC} + \overrightarrow{CE}\\
\overrightarrow{BE} &= \overrightarrow{BC} + \overrightarrow{AB}\\
\overrightarrow{BE} &= \overrightarrow{AC}
\end{align*}

Voir figure :
\begin{center}
\includegraphics[width=.9\linewidth]{./img/sol9q3.png}
\end{center}
\end{enumerate}



\chapter{Solution du QCM d'auto-évaluation}
\label{sec:orga1b4134}

\begin{enumerate}
\item Un vecteur est caractérisé par :
\begin{enumerate}[label=\alph*.]
\item Sa longueur uniquement
\item Sa direction et son sens uniquement
\item Sa direction, son sens et sa norme

(\textbf{Bonne réponse})
\item Son origine et son extrémité
\end{enumerate}

\item Les vecteurs \(\overrightarrow{AB}\) et \(\overrightarrow{CD}\) 
sont égaux si :
\begin{enumerate}[label=\alph*.]
\item A = C et B = D

(\textbf{Bonne réponse})
\item ABDC est un parallélogramme

(\textbf{Bonne réponse})
\item AB = CD (distances égales)
\item Les segments [AB] et [CD] sont parallèles
\end{enumerate}
\end{enumerate}

\part{Solutions des exercices de la \(Ca_2\)}
\label{sec:orga3a7021}
\chapter{Solution de l'exercice 10}
\label{sec:org4ee24f8}

   On considère la configuration initiale de l'exercice 9 voir
figure :

\begin{center}
\includegraphics[width=.9\linewidth]{./img/ABCtrig.png}
\end{center}

\begin{enumerate}
\item Pour construire le point D tel que \(\overrightarrow{BD} =
       \overrightarrow{BC} + \overrightarrow{BA}\) il faut reporter le
vecteur \(\overrightarrow{BA}\) à partir du point C.

\item On en déduit que le quadrilatère ABCD est un
parallélogramme. En effet :
\begin{align*}
\overrightarrow{BD} &= \overrightarrow{BC} + \overrightarrow{BA}\\
\overrightarrow{BD} &= \overrightarrow{BC} + \overrightarrow{CD}\\
&\Rightarrow \overrightarrow{BA} = \overrightarrow{CD}\\
\overrightarrow{BD} &= \overrightarrow{BC} + \overrightarrow{BA}\\
\overrightarrow{BD} &= \overrightarrow{BA} + \overrightarrow{AD}\\
&\Rightarrow \overrightarrow{BC} = \overrightarrow{AD}
\end{align*}

Voir figure :
\begin{center}
\includegraphics[width=.9\linewidth]{./img/sol10q2.png}
\end{center}
\item Construisons le point E tel que \(\overrightarrow{AE} =
       \overrightarrow{AB} + \overrightarrow{AC}\) de façon similaire
à la question 1.

\item On en déduit que le quadrilatère ABEC est un parallélogramme de
façon similaire à la question 2. En effet :
\begin{align*}
\overrightarrow{AE} &= \overrightarrow{AB} + \overrightarrow{AC}\\
\overrightarrow{AE} &= \overrightarrow{AB} + \overrightarrow{BE}\\
&\Rightarrow \overrightarrow{AC} = \overrightarrow{BE}\\
\overrightarrow{AE} &= \overrightarrow{AC} + \overrightarrow{CE}\\
&\Rightarrow \overrightarrow{AB} = \overrightarrow{CE}
\end{align*}

Voir figure :
\begin{center}
\includegraphics[width=.9\linewidth]{./img/sol10q4.png}
\end{center}
\end{enumerate}


\chapter{Solution du QCM d'auto-évaluation}
\label{sec:org73e1691}

Considérons deux vecteurs \(\vec{u}\) et \(\vec{v}\).

\begin{enumerate}
\item Pour construire géométriquement la somme il faut :
\begin{enumerate}[label=\alph*.]
\item partir de l'origine du vecteur \(\vec{u}\) puis, arrivé à
son extrémité appliquer le vecteur \(\vec{v}\)

(\textbf{Bonne réponse})
\item partir de l'origine du vecteur \(\vec{v}\) puis, arrivé à
son extrémité appliquer le vecteur \(\vec{u}\)

(\textbf{Bonne réponse})
\item les deux propositions précédentes aboutissent au même
résultat

(\textbf{Bonne réponse})
\end{enumerate}
\item Si \(\vec{u}\) et \(\vec{v}\) ne sont pas colinéaires alors
\(\vec{w} = \vec{u} + \vec{v}\) :
\begin{enumerate}[label=\alph*.]
\item est un vecteur ayant même direction que \(\vec{u}\) et
\(\vec{v}\)
\item représente la diagonale du parallélogramme obtenu en
faisant partir \(\vec{u}\) et \(\vec{v}\) de la même origine

(\textbf{Bonne réponse})
\end{enumerate}
\end{enumerate}

\part{Solutions des exercices de la \(Ca_3\)}
\label{sec:orga81900a}
\chapter{Solution de l'exercice 11}
\label{sec:org7bd32a2}

On se place dans le plan muni repère orthonormé \((O ; \vec{i} ;
    \vec{j})\). Avec :
\begin{align*}
\vec{i}&\begin{pmatrix}1\\0\end{pmatrix} && \vec{j}\begin{pmatrix}0\\1\end{pmatrix}
\end{align*}

\begin{enumerate}
\item Représenter le vecteur
\(\vec{u}\begin{pmatrix}3\\ 5\end{pmatrix}\).

Voir figure :
\begin{center}
\includegraphics[width=.9\linewidth]{./img/sol11q1.png}
\end{center}
\item Lire les coordonnées du vecteur \(\vec{v}\) sur la figure :
\begin{center}
\includegraphics[width=.9\linewidth]{./img/sol11q2.png}
\end{center}

On peut voir sur la figure que :
\[\vec{v} = 5\vec{i} + 3\vec{j}\]
Ainsi les coordonnées du vecteur \(\vec{v}\) sont :
\[\vec{v}\begin{pmatrix}5\\ 3\end{pmatrix}\]
\end{enumerate}


\chapter{Solution du QCM d'auto-évaluation}
\label{sec:org1bc85b0}

Considérons le vecteur
\(\vec{u}\begin{pmatrix}a\\b\end{pmatrix}\). Alors :

\begin{enumerate}
\item Pour le représenter dans le repère \((O ; \vec{i} , \vec{j})\) :
\begin{enumerate}[label=\alph*.]
\item on se place au point M de coordonnées (a ; b) et on trace le
vecteur en se déplaçant de a unités sur l'axe horizontal et
b unités sur l'axe vertical.
\item partant de l'origine du repère on se déplace de a unités sur
l'axe horizontal et b unités sur l'axe vertical.

(\textbf{Bonne réponse})
\end{enumerate}
\item Pour lire les cordonnées d'un vecteur \(\vec{v}\) :
\begin{enumerate}[label=\alph*.]
\item on se place à son origine et on reporte les coordonnées du
point
\item on trace un représentant du vecteur en partant de l'origine
du repère et on lit les coordonnées de son extrémité

(\textbf{Bonne réponse})
\end{enumerate}
\end{enumerate}

\part{Solutions des exercices de la \(Ca_4\)}
\label{sec:orgcf39fb2}
\chapter{Solution de l'exercice 12}
\label{sec:orgf5d195d}

On se place dans le plan muni repère orthonormé \((O ; \vec{i} ;
    \vec{j})\). Avec :
\begin{align*}
\vec{i}&\begin{pmatrix}1\\0\end{pmatrix} && \vec{j}\begin{pmatrix}0\\1\end{pmatrix}
\end{align*}

Calculons les coordonnées des vecteurs :
\begin{align*}
\vec{s} &= \vec{i} + \vec{j} = \begin{pmatrix}1\\ 1\end{pmatrix}\\
\vec{d} &= \vec{i} - \vec{j} = \begin{pmatrix}1\\ -1\end{pmatrix}\\
\vec{p} &= 3\vec{d} = \begin{pmatrix}3\\ -3\end{pmatrix}\\
\vec{c} &= 2\vec{s} - 5\vec{d} = \begin{pmatrix}-3\\ 7\end{pmatrix}
\end{align*}

\chapter{Solution du QCM d'auto-évaluation}
\label{sec:org60739d0}

\begin{enumerate}
\item Soient \[\vec{u}_1\begin{pmatrix}x_1\\y_1\end{pmatrix}\] et
\[\vec{u}_2\begin{pmatrix}x_2\\y_2\end{pmatrix}\] deux vecteurs
alors \[\vec{u}_3 = \vec{u}_1 + \vec{u}_2\] a pour
coordonnées : 
\begin{enumerate}[label=\alph*.]
\item \(\vec{u}_3\begin{pmatrix}x_1x_2\\y_1y_2\end{pmatrix}\)
\item \(\vec{u}_3\begin{pmatrix}x_1 - x_2\\y_1 - y_2\end{pmatrix}\)
\item \(\vec{u}_3\begin{pmatrix}x_1 + x_2\\y_1 +
          y_2\end{pmatrix}\)

(\textbf{Bonne réponse})
\item \(\vec{u}_3\begin{pmatrix}\dfrac{x_1}{x_2}\\\dfrac{y_1}{y_2}\end{pmatrix}\)
\end{enumerate}
\item Soient un vecteur \(\vec{u}\begin{pmatrix}x\\y\end{pmatrix}\)
et un réel k alors \(\vec{v} = ku\) vérifie :
\begin{enumerate}[label=\alph*.]
\item \(\vec{v}\begin{pmatrix}kx\\ y\end{pmatrix}\)
\item \(\vec{v}\begin{pmatrix}x\\ ky\end{pmatrix}\)
\item \(\vec{v}\begin{pmatrix}kx\\ y\end{pmatrix}\)

(\textbf{Bonne réponse})
\item \(\vec{v}\begin{pmatrix}\frac{1}{k}x\\ \frac{1}{k}y\end{pmatrix}\)
\end{enumerate}
\end{enumerate}


\part{Solutions des exercices de la \(Ca_5\)}
\label{sec:orge19c3b1}
\chapter{Solution de l'exercice 13}
\label{sec:orgb0050ad}


On se place dans le plan muni repère orthonormé \((O ; \vec{i} ;
    \vec{j})\). Avec :
\begin{align*}
\vec{i}&\begin{pmatrix}1\\0\end{pmatrix} && \vec{j}\begin{pmatrix}0\\1\end{pmatrix}
\end{align*}

On considère les points A(2 ; 3), B(-3 ; 2), C(-2 ; -3) et D(3 ;
-2) tels que sur la figure :

\begin{center}
\includegraphics[width=.9\linewidth]{./img/figexo13.png}
\end{center}

\begin{enumerate}
\item Calculs des distances AB, AC, AD, BC, BD, CD :
\begin{align*}
AB &= \sqrt{(-3 - 2)^2 + (2 - 3)^2} = \sqrt{26}\\
AC &= \sqrt{(-2 - 2)^2 + (-3 - 3)^2} = \sqrt{52}\\
AD &= \sqrt{(3 - 2)^2 + (-2 - 3)^2} = \sqrt{26}\\
BC &= \sqrt{(-2 - (-3))^2 + (-3 - 2)^2} = \sqrt{26}\\
BD &= \sqrt{(3 - (-3))^2 + (-2 - 2)^2} = \sqrt{52}\\
CD &= \sqrt{(3 - (-2))^2 + (-2 - (-2))^2} = \sqrt{26}
\end{align*}

Voir figure :
\begin{center}
\includegraphics[width=.9\linewidth]{./img/sol13q1.png}
\end{center}
\item Calculs des coordonnées des milieux des segments [AB], [AC],
[AD], [BC], [BD], [CD] :
\begin{align*}
M_1&\begin{pmatrix}\frac{2+(-3)}{2}\\ \frac{3+2}{2}\end{pmatrix} = \begin{pmatrix}-\frac{1}{2}\\ \frac{5}{2}\end{pmatrix}\\
M_2&\begin{pmatrix}\frac{2+(-2)}{2}\\ \frac{3+(-3)}{2}\end{pmatrix} = \begin{pmatrix}0\\ 0\end{pmatrix} = O\\
M_3&\begin{pmatrix}\frac{2+3}{2}\\ \frac{3+(-2)}{2}\end{pmatrix} = \begin{pmatrix}\frac{5}{2}\\ \frac{1}{2}\end{pmatrix}\\
M_4&\begin{pmatrix}\frac{(-3)+(-2)}{2}\\ \frac{2+(-3)}{2}\end{pmatrix} = \begin{pmatrix}-\frac{5}{2}\\ -\frac{1}{2}\end{pmatrix}\\
M_5&\begin{pmatrix}\frac{(-3)+3}{2}\\ \frac{2+(-2)}{2}\end{pmatrix} = \begin{pmatrix}0\\ 0\end{pmatrix} = O\\
M_6&\begin{pmatrix}\frac{(-2)+3}{2}\\ \frac{(-3)+(-2)}{2}\end{pmatrix} = \begin{pmatrix}\frac{1}{2}\\ -\frac{5}{2}\end{pmatrix}
\end{align*}

Voir figure :
\begin{center}
\includegraphics[width=.9\linewidth]{./img/sol13q2.png}
\end{center}
\end{enumerate}


\chapter{Solution du QCM d'auto-évaluation}
\label{sec:org451f186}

\begin{enumerate}
\item La distance entre \(A(x_A ; y_A)\) et \(B(x_B ; y_B)\) vaut :
\begin{enumerate}[label=\alph*.]
\item \(AB = x_Ax_B + y_Ay_B\)
\item \(AB = (x_A + x_B)^2 + (y_A + y_B)^2\)
\item \(AB = (x_A - x_B)^2 + (y_A - y_B)^2\)
\item \(AB = \sqrt{(x_A - x_B)^2 + (y_A - y_B)^2}\)

(\textbf{Bonne réponse})
\end{enumerate}
\item Le milieu \(M(x_M ; y_M)\) du segment [AB] vérifie :
\begin{enumerate}[label=\alph*.]
\item \((x_M ; y_M) = \left(\dfrac{x_A + x_B}{2} ; \dfrac{y_A +
          y_B}{2}\right)\)

(\textbf{Bonne réponse})
\item \((x_M ; y_M) = \left(\dfrac{x_A - x_B}{2} ; \dfrac{y_A -
          y_B}{2}\right)\)
\item \((x_M ; y_M) = \left(\dfrac{x_A \times x_B}{2} ; \dfrac{y_A \times
          y_B}{2}\right)\)
\item \((x_M ; y_M) = \left(\dfrac{x_A \div x_B}{2} ; \dfrac{y_A \div
          y_B}{2}\right)\)
\end{enumerate}
\end{enumerate}


\part{Solutions des exercices de la \(Ca_6\)}
\label{sec:org0b11382}
\chapter{Solution de l'exercice 14}
\label{sec:org36f43ca}

On reprend la configuration de l'exercice précédent.

Voir la figure :

\begin{center}
\includegraphics[width=.9\linewidth]{./img/figexo13.png}
\end{center}

\begin{enumerate}
\item Le point O étant l'origine du repère on peut facilement
déterminer les coordonnées des vecteurs \(\overrightarrow{OA}\)
et \(\overrightarrow{OC}\) et ainsi déterminer si les points A,
O et C sont alignés :
\begin{align*}
\overrightarrow{OA} &= 2\vec{i} + 3\vec{j}\\
\overrightarrow{OC} &= -2\vec{i} - 3\vec{j}\\
\overrightarrow{OC} &= -\overrightarrow{OA}
\end{align*}

Voir figure :
\begin{center}
\includegraphics[width=.9\linewidth]{./img/sol14q1.png}
\end{center}
\item D'après ce qui précède on a :
\begin{align*}
\lvert\lvert\vec{u}\rvert\rvert &= \lvert\lvert\overrightarrow{OA}\rvert\rvert = \lvert\lvert\overrightarrow{OC}\rvert\rvert \\
\lvert\lvert\vec{u}\rvert\rvert &= \sqrt{13}
\end{align*}
On en déduit que O est le milieu du segment [AC].
\item Calculons le déterminant \(det(\overrightarrow{OD} ,
       \overrightarrow{OB})\) :
\begin{align*}
det(\overrightarrow{OD} , \overrightarrow{OB}) &= \begin{vmatrix}3&-3\\ -2&2\end{vmatrix} = 3\times 2 - (-2)\times (-3) = 0
\end{align*}
Le déterminant est nul donc les vecteurs
\(\overrightarrow{OD}\) et \(\overrightarrow{OB}\) sont
colinéaires. Puisqu'ils ont la même origine on en déduit que
les points B, O et D sont alignés.

Voir figure :
\begin{center}
\includegraphics[width=.9\linewidth]{./img/sol14q3.png}
\end{center}

\item Comparons OB et OD :
\begin{align*}
OB &= \sqrt{(-3)^2 + 2^2} = \sqrt{13}\\
OD &= \sqrt{3^2 + (-2)^2} = \sqrt{13}\\
OB &= OD
\end{align*}
On en déduit que O est le milieu de [BD] et donc que ABCD est
un rectangle car ses diagonales sont de même longueur
\(2\sqrt{13} = \sqrt{52}\).

Voir figure :
\begin{center}
\includegraphics[width=.9\linewidth]{./img/sol14q4.png}
\end{center}

\item Puisque ABCD est un rectangle, c'est donc un parallélogramme
donc on a :
\begin{align*}
\overrightarrow{AB} &= \overrightarrow{DC}\\
\overrightarrow{AB} &= -\overrightarrow{CD}
\end{align*}
Ainsi les vecteurs \(\overrightarrow{AB}\) et
\(\overrightarrow{CD}\) sont colinéaires.

Voir figure :
\begin{center}
\includegraphics[width=.9\linewidth]{./img/sol14q5.png}
\end{center}

\item Comparons AB et AD :
\begin{align*}
AB &= \sqrt{(-3 - 2)^2 + (2 - 3)^2} = \sqrt{26}\\
AD &= \sqrt{(3 - 2)^2 + (-2 - 3)^2} = \sqrt{26}\\
AB &= AD
\end{align*}

Voir figure :
\begin{center}
\includegraphics[width=.9\linewidth]{./img/sol14q6.png}
\end{center}

\item Le quadrilatère ABCD est un carré car c'est un rectangle avec
deux côtés consécutifs de même longueur.

Voir figure :
\begin{center}
\includegraphics[width=.9\linewidth]{./img/sol14q7.png}
\end{center}
\end{enumerate}

\chapter{Solution du QCM d'auto-évaluation}
\label{sec:org7800a2b}

\begin{enumerate}
\item Pour montrer que A, B et C sont alignés il faut :
\begin{enumerate}[label=\alph*.]
\item que \(\overrightarrow{AB} = \overrightarrow{AC}\)
\item qu'il existe un réel k tel que \(\overrightarrow{AB} =
          k\overrightarrow{AC}\)

(\textbf{Bonne réponse})
\item vérifier que \(\overrightarrow{AB} + \overrightarrow{BC} =
          \overrightarrow{AC}\)
\item vérifier que \(det(\overrightarrow{AB} ,
          \overrightarrow{AC}) = 0\)

(\textbf{Bonne réponse})
\end{enumerate}
\item Pour montrer que les droites (AB) et (CD) sont parallèles il faut :
\begin{enumerate}[label=\alph*.]
\item montrer que les vecteurs \(\overrightarrow{AB}\) et
\(\overrightarrow{CD}\) sont colinéaires

(\textbf{Bonne réponse})
\item vérifier que \(det(\overrightarrow{AB} ,
          \overrightarrow{CD}) = 0\)

(\textbf{Bonne réponse})
\item montrer que \(\overrightarrow{AB} = \overrightarrow{CD}\)
\item vérifier que \(det(\overrightarrow{AB} ,
          \overrightarrow{CD}) \neq 0\)
\end{enumerate}
\end{enumerate}

\part{Solutions pour la \(Ca_7\)}
\label{sec:orgbdb2257}
\chapter{Solution de l'exercice 15}
\label{sec:orge2ad88f}

Pour chacune des situations suivantes, indiquons comment la
résoudre selon la représentation vectorielle parmi :

\begin{itemize}
\item Analytique (coordonnées, calculs algébriques)
\item Colinéarité (proportionnalité, déterminant)
\item Géométrique (relation de Chasles, parallélogramme)
\end{itemize}


\begin{enumerate}
\item Démontrer que les points A, B, C sont alignés.
\begin{itemize}
\item Analytique : à l'aide des coordonnées de chaque point on peut
calculer les coordonnées des vecteurs \(\overrightarrow{AB}\)
et \(\overrightarrow{AC}\) et vérifier si les vecteurs sont
colinéaires ou pas.
\item Colinéarité : à l'aide des coordonnées ou de la décomposition
de Chasles on peut calculer le déterminant ou établir une
relation du type \(\overrightarrow{AB} =
         k\overrightarrow{AC}\) si les vecteurs sont colinéaires.
\item Géométrique : la relation de Chasles nous permet d'établie si
la relation \(\overrightarrow{AB} = k\overrightarrow{AC}\)
existe ou pas
\end{itemize}
\item Calculer la distance entre deux points A(2;3) et B(5;7)
\begin{itemize}
\item Analytique : on applique la formule (qui découle de
Pythagore)
\item Colinéarité : pour ce type de problème la colinéarité est
inutile
\item Géométrique : pour ce type de problème ni Chasles ni les
identités du parallélogramme ne peuvent servir
\end{itemize}
\item Montrer que ABCD est un parallélogramme
\begin{itemize}
\item Analytique : à l'aide des coordonnées on peut vérifier si on
a l'égalité vectorielle
\[\overrightarrow{AB} = \overrightarrow{DC}\]
ou pas.
\item Colinéarité : à l'aide des coordonnées ou de la relation de
Chasles on peut vérifier si on a l'égalité vectorielle
\[\overrightarrow{AB} = \overrightarrow{DC}\]
ou pas. On peut également calculer les deux déterminants
importants
\[det(\overrightarrow{AB} , \overrightarrow{DC})\quad
         det(\overrightarrow{AD} , \overrightarrow{BC})\]
\item Géométrique : en utilisant Chasles on peut vérifier si on
obtient la relation
\[\overrightarrow{AB} + \overrightarrow{AD} = \overrightarrow{AC}\]
\end{itemize}
\item Trouver les coordonnées du point M tel que :
\[\overrightarrow{AM} = 2\overrightarrow{AB} +
       3\overrightarrow{AC}\]
\begin{itemize}
\item Analytique : on calcule les coordonnées des vecteurs \(\overrightarrow{AB}\)
et \(\overrightarrow{AC}\) et on résout les deux équations
pour obtenir les coordonnées du points M(x ; y).
Concrètement :
\begin{align*}
x - x_A &= 2(x_B - x_A) + 3(x_C - x_A)\\
y - y_A &= 2(y_B - y_A) + 3(y_C - y_A)\\
x &= -4x_A + 2x_B + 3x_C\\
y &= -4y_A + 2y_B + 3y_C
\end{align*}
\item Colinéarité : on ne peut pas obtenir les coordonnées du point
M uniquement avec le déterminant.
\item Géométrique : on peut construire le point M grâce à la
relation vectorielle puis en utilisant Chasles on peut
exprimer le vecteur \(\overrightarrow{OM}\) en fonction des
vecteurs \(\overrightarrow{OA}\), \(\overrightarrow{OB}\) et
\(\overrightarrow{OC}\)
\end{itemize}
\item Vérifier si deux droites (AB) et (CD) sont parallèles
\begin{itemize}
\item Analytique : on peut comparer les coordonnées des vecteurs
\(\overrightarrow{AB}\) et \(\overrightarrow{CD}\)
\item Colinéarité : on peut calculer le déterminant
\(det(\overrightarrow{AB} , \overrightarrow{CD}\) et voir
s'il est nul ou pas
\item Géométrique : on peut vérifier si on obtient une identité du
parallélogramme ou pas
\end{itemize}
\end{enumerate}


\chapter{Solution du programme}
\label{sec:orgdf8ef90}

Écrire un programme Python qui résout le problème
\[\overrightarrow{AM} = a\overrightarrow{AB} +
    b\overrightarrow{AC}\]
C'est-à-dire un programme qui permet d'exprimer les coordonnées du
point M en fonction des paramètres a et b et des coordonnées des
points déjà existants A, B, et C.

\begin{minted}[]{python}
def get_M(a, b, A, B, C):
    """
    Cette fonction prend en entrées :
    + a : 1 float correspondant au coefficient du vecteur AB
    + b : 1 float correspondant au coefficient du vecteur AC
    + A : 1 tuple de float correspondant aux coordonnées du point A
    + B : 1 tuple de float correspondant aux coordonnées du point B
    + C : 1 tuple de float correspondant aux coordonnées du point C
    et elle renvoie 1 tuple de float correspondant aux coordonnées du point M
    """
    x = (1 - a - b) * A[0] + a * B[0] + b * C[0]
    y = (1 - a - b) * A[1] + a * B[1] + b * C[1]
    return (x, y)


def vectAB(A, B):
    """
    Cette fonction prend en entrées :
    + A : 1 tuple de float correspondant aux coordonnées du point A
    + B : 1 tuple de float correspondant aux coordonnées du point B
    et renvoie 1 tuple de float correspondant aux coordonnées du vecteur AB
    """
    x = B[0] - A[0]
    y = B[1] - A[1]
    return (x, y)


# Tests
a, b, A, B, C = 1, 1, (3, 2), (-3, 2), (3, -2)
M = get_M(a, b, A, B, C)
relation = f"On a la relation Vect(A, M) = {a}Vect(A, B) + {b}Vect(A, C)"
print(relation)
input("Pour voir les coordonnées du point M tapez 1\t")
coordM = f"Voici les coordonnées du point M({M[0]}, {M[1]})"
print(coordM)
coordAB = vectAB(A, B)
coordAC = vectAB(A, B = C)
eqX = f"x - {A[0]} = {a} * {coordAB[0]} + {b} * {coordAC[0]}"
input("Pour voir l'équation en x tapez 1\t")
print(eqX)
eqY = f"y - {A[1]} = {a} * {coordAB[1]} + {b} * {coordAC[1]}"
input("Pour voir l'équation en y tapez 1\t")
print(eqY)
solve_x = f"x = {A[0] + a * coordAB[0] + b * coordAC[0]}"
input("Pour voir la solution de l'équation en x tapez 1\t")
print(solve_x)
solve_y = f"y = {A[1] + a * coordAB[1] + b * coordAC[1]}"
input("Pour voir la solution de l'équation en y tapez 1\t")
print(solve_y)
\end{minted}


\chapter{Solution du QCM d'auto-évaluation}
\label{sec:orgafe62a9}

\textbf{Parmi les réponses proposées au moins une est la bonne. Cela
signifie qu'il peut y avoir \emph{plusieurs} bonnes réponses.}

\begin{enumerate}
\item Si ABC est un triangle et que D est un 4ème point qui vérifie
l'égalité \[\overrightarrow{AD} = \overrightarrow{AB} +
       \overrightarrow{AC}\]
alors on peut en déduire que :
\begin{enumerate}[label=\alph*.]
\item \(\overrightarrow{AB} + \overrightarrow{BC} =
          \overrightarrow{AC}\) la relation de Chasles n'est pas une
déduction, elle existe toujours
\item ABCD est un parallélogramme
\item ABDC est un parallélogramme

(\textbf{Bonne réponse})
\item \(\overrightarrow{AB} = \overrightarrow{CD}\) et
\(\overrightarrow{AC} = \overrightarrow{BD}\) les deux
égalités sont vraies

(\textbf{Bonne réponse})
\item \(\overrightarrow{AB} = \overrightarrow{CD}\) ou
\(\overrightarrow{AC} = \overrightarrow{BD}\) une seule des
deux égalités est vraie
\item \(\overrightarrow{AB} \neq \overrightarrow{CD}\) et
\(\overrightarrow{AC} \neq \overrightarrow{BD}\) aucune
des égalités n'est vraie
\item le point D est à l'intérieur du triangle ABC
\item le point D est l'image du point A par la symétrie de
centre le milieu du segment [BC]

(\textbf{Bonne réponse})
\item le point D est à l'extérieur du triangle ABC

(\textbf{Bonne réponse})
\item le point D est l'image du point I par la translation de
vecteur \(\overrightarrow{AI}\) où I est le milieu du
segment [BC]

(\textbf{Bonne réponse})
\end{enumerate}
\item Si ABCD est un carré alors : 
\begin{enumerate}[label=\alph*.]
\item Les vecteurs \(\overrightarrow{AB}\) et
\(\overrightarrow{AC}\) forment une base orthornomée
\item Les vecteurs \(\overrightarrow{AB}\) et
\(\overrightarrow{AD}\) forment une base orthornomée

(\textbf{Bonne réponse})
\item \(det(\overrightarrow{AB} , \overrightarrow{AC}) = 1\)
\item \(det(\overrightarrow{AB} , \overrightarrow{AD}) = 0\)
\item \(det(\overrightarrow{AB} , \overrightarrow{AD}) = 1\)
\item \(det(\overrightarrow{AB} , \overrightarrow{AC}) = 0\)
\item \(AC^2 = AB^2 + BC^2\)

(\textbf{Bonne réponse})
\item \(\overrightarrow{AB} + \overrightarrow{CD} =
          2\overrightarrow{AB}\)
\item \(\overrightarrow{AB} + \overrightarrow{CD} = \vec{0}\)

(\textbf{Bonne réponse})
\item Le centre O du carré vérifie
\[\overrightarrow{OA} + \overrightarrow{OB} +
           \overrightarrow{OC} + \overrightarrow{OD} = \vec{0}\]

(\textbf{Bonne réponse})
\end{enumerate}
\item Si ABCD est un rectangle et O l'intersection des droites (AC)
et (BD) alors : 
\begin{enumerate}[label=\alph*.]
\item \(\overrightarrow{OA} + \overrightarrow{OB} +
          \overrightarrow{OB} + \overrightarrow{OD} = \vec{0}\)

(\textbf{Bonne réponse})
\item \(\overrightarrow{AO} = \overrightarrow{OC} =
          \dfrac{1}{2}\overrightarrow{AC}\)

(\textbf{Bonne réponse})
\item \(det(\overrightarrow{AB} , \overrightarrow{AC}) = 1\)
\item \(det(\overrightarrow{AB} , \overrightarrow{AD}) = 0\)
\item \(det(\overrightarrow{AB} , \overrightarrow{AD}) = 1\)
\item \(det(\overrightarrow{AB} , \overrightarrow{AC}) = 0\)
\item \(AC > AB + BC\)
\item \(AC < AB + AD\)

(\textbf{Bonne réponse})
\item \(AC = BD\)

(\textbf{Bonne réponse})
\item \(AC\neq BD\)
\end{enumerate}
\item Soient A et B deux points distincts du plan. Si C est l'image
de B par la translation de vecteur \(\overrightarrow{AB}\)
alors : 	 
\begin{enumerate}[label=\alph*.]
\item B est le milieu du segment [AC]

(\textbf{Bonne réponse})
\item \(\overrightarrow{AC} = 2\overrightarrow{AB}\)
\item \(\overrightarrow{OC} = \overrightarrow{OA} +
          2\overrightarrow{OB}\)
\item les coordonnées de C vérifient :
\begin{align*}
x_C &= 2x_B - x_A\\
y_C &= 2y_B - y_A
\end{align*}
\item les coordonnées de C vérifient :
\begin{align*}
x_C &= 2x_B + x_A\\
y_C &= 2y_B + y_A
\end{align*}

(\textbf{Bonne réponse})
\item C est l'image de A par la symétrie de centre B.

(\textbf{Bonne réponse})
\item C est le milieu du segment [AB].
\item \(\overrightarrow{AB} = \overrightarrow{BC}\)

(\textbf{Bonne réponse})
\item \(\overrightarrow{AB} + \overrightarrow{BC} =
          2\overrightarrow{AB}\)

(\textbf{Bonne réponse})
\item \(det(\overrightarrow{AB} , \overrightarrow{AC}) = 0\)

(\textbf{Bonne réponse})
\end{enumerate}
\item Si on a \(det(\overrightarrow{AB}, \overrightarrow{AD}) \neq 0\)
et \(det(\overrightarrow{AB}, \overrightarrow{DC}) = 0\) alors :
\begin{enumerate}[label=\alph*.]
\item ABCD ou ABDC est un trapèze.

(\textbf{Bonne réponse})
\item Si AB = DC alors ABCD ou ABDC est un parallélogramme.

(\textbf{Bonne réponse})
\item Si AB = AD = DC alors ABCD est un losange.

(\textbf{Bonne réponse})
\item Si AB = AC = CD alors ABDC est un losange.

(\textbf{Bonne réponse})
\item Si AB = DC et CA = BD alors ABDC est un rectangle.
\item Si AB = DC et AD = BC alors ABDC est un rectangle.
\item Si AB = DC et AC = BD alors ABCD est un rectangle.

(\textbf{Bonne réponse})
\item Si AB = DC et AD = BC alors ABCD est un rectangle.
\item Si AB = DC et AC = BD alors ABCD est un losange.
\item Si AB = DC et AD = BC alors ABCD est un losange.

(\textbf{Bonne réponse})
\end{enumerate}
\item Soient \(\vec{u}\), \(\vec{v}\) et \(\vec{w}\) trois vecteurs
tels que \(\vec{w} = \vec{u} + \vec{v}\).
\begin{enumerate}[label=\alph*.]
\item Si \(\lvert\lvert\vec{u}\rvert\rvert +
          \lvert\lvert\vec{v}\rvert\rvert =
          \lvert\lvert\vec{w}\rvert\rvert\) alors les trois vecteurs
sont colinéaires.
\item Il est impossible que les trois vecteurs soient colinéaires.
\item Si \(det(\vec{u}, \vec{v}) = 0\) alors les trois vecteurs
sont colinéaires.
\item Si \(det(\vec{u}, \vec{v}) = 0\) alors \(det(\vec{v},
          \vec{w}) = 0\).
\item Si \(det(\vec{w}, \vec{u}) = 0\) alors \(det(\vec{u},
          \vec{v}) = 0\).
\item \(\lvert\lvert\vec{u}\rvert\rvert +
          \lvert\lvert\vec{v}\rvert\rvert >
          \lvert\lvert\vec{w}\rvert\rvert\)
\item \(\lvert\lvert\vec{u}\rvert\rvert +
          \lvert\lvert\vec{v}\rvert\rvert <
          \lvert\lvert\vec{w}\rvert\rvert\)
\item Si \(det(\vec{u}, \vec{v}) = 0\) alors soit \(\vec{u} =
          \vec{0}\) soit \(\vec{v} = \vec{0}\)
\item Si \(det(\vec{u}, \vec{v}) = 0\) alors soit \(\vec{w} =
          \vec{u}\) soit \(\vec{w} = \vec{v}\)
\item Si \(det(\vec{w}, \vec{u}) = 0\) alors soit \(\vec{u} =
          -\vec{v}\) soit \(\vec{w} = \vec{0}\)
\end{enumerate}
\item Soit un vecteur \(\vec{u}\) de norme
\(\lvert\lvert\vec{u}\rvert\rvert = 5\).
\begin{enumerate}[label=\alph*.]
\item Si \(x_{\vec{u}} = 3\) alors \(y_{\vec{u}} = 4\).
\item Si \(x_{\vec{u}} = 3\) alors \(y_{\vec{u}} = -4\).
\item Si \(x_{\vec{u}} = -3\) alors \(y_{\vec{u}} = -4\).
\item Si \(x_{\vec{u}} = -3\) alors \(y_{\vec{u}} = 4\).
\item Si \(x_{\vec{u}} = \pm 3\) alors \(y_{\vec{u}} = \pm 4\).

(\textbf{Bonne réponse})
\item Si \(x_{\vec{u}} = -4\) alors \(y_{\vec{u}} = 3\).
\item Si \(x_{\vec{u}} = -4\) alors \(y_{\vec{u}} = -3\).
\item Si \(x_{\vec{u}} = 4\) alors \(y_{\vec{u}} = -3\).
\item Si \(x_{\vec{u}} = 4\) alors \(y_{\vec{u}} = 3\).
\item Si \(x_{\vec{u}} = \pm 4\) alors \(y_{\vec{u}} = \pm 3\).

(\textbf{Bonne réponse})
\end{enumerate}
\item Dans un repère orthonormée \((O ; \vec{i} , \vec{j})\) on
considère les points A(3 ; 2), B(3 ; -2), C(-3 ; -2), D(3 ;
-1), E(-3 ; -1), F(-1 ; -1), G(1 ; 1), H(1 ; 2), I(-1 ; 2).       
\begin{enumerate}[label=\alph*.]
\item Les vecteurs \(\overrightarrow{AB}\) et
\(\overrightarrow{AC}\) sont colinéaires car
\[det(\overrightarrow{AB} , \overrightarrow{AC}) = 0\]
\item Les vecteurs \(\overrightarrow{AH}\) et
\(\overrightarrow{AI}\) sont colinéaires car
\[det(\overrightarrow{AH} , \overrightarrow{AI}) = 0\]
\item Les points A, B et C sont alignés.
\item Les points D, E et F sont alignés.

(\textbf{Bonne réponse})
\item ABC est un triangle rectangle en B.

(\textbf{Bonne réponse})
\item ABD est un triangle rectangle en B.
\item GDF est un triangle rectangle en G.

(\textbf{Bonne réponse})
\item GDF est un triangle isocèle en G.

(\textbf{Bonne réponse})
\item BCHA est un trapèze.

(\textbf{Bonne réponse})
\item Les vecteurs \(\overrightarrow{BC}\) et
\(\overrightarrow{AH}\) sont colinéaires.

(\textbf{Bonne réponse})
\end{enumerate}
\item Dans un repère orthonormée \((O ; \vec{i} , \vec{j})\) on
considère les points A(2 ; 3), B(-3 ; 2), C(-2 ; -3); D(3 ;
-2).
\begin{enumerate}[label=\alph*.]
\item Le triangle BOA est isocèle en A.
\item Le triangle BOA est isocèle en B.
\item Le triangle BOA est isocèle en O.

(\textbf{Bonne réponse})
\item Le triangle BOA est rectangle en A.
\item Le triangle BOA est rectangle en B.
\item Le triangle BOA est rectangle en O.

(\textbf{Bonne réponse})
\item C est l'image de O par la translation de vecteur
\(\overrightarrow{AO}\).

(\textbf{Bonne réponse})
\item D est l'image de O par la translation de vecteur
\(\overrightarrow{BO}\).

(\textbf{Bonne réponse})
\item ABDC est un carré.
\item ABCD est un carré.

(\textbf{Bonne réponse})
\end{enumerate}
\end{enumerate}


\part{Solutions des exercices complémentaires}
\label{sec:orgbf5ad5d}
\chapter{Pour s'exercer davantage}
\label{sec:org26974d9}
\section{Solution exercice 16}
\label{sec:org537b840}


On se place dans le plan muni du repère orthonormé \((O ; \vec{i},
    \vec{j})\). 

\begin{enumerate}
\item Placer les points O(0 ; 0), I(1 ; 0), J(0 ; 1) et A tel que
\[\overrightarrow{OA} = \vec{i} + \vec{j}\]
On peut voir sur la figure que le triangle OIA est rectangle et
isocèle en I.
\begin{align*}
OI^2 + IA^2 &= 1^2 + 0^2 + 0^2 + 1^2 = 2\\
OA^2 &= 1^2 + 1^2 = 2
\end{align*}

Voir figure :
\begin{center}
\includegraphics[width=.9\linewidth]{./img/sol16q1.png}
\end{center}

\item Placer le point B tel que
\[\overrightarrow{AB} = \vec{j} - \vec{i}\]
Une égalité vectorielle plus simple pour placer le
point B est
\[\overrightarrow{OB} = 2\vec{j}\]
Pour la trouver on utilise Chasles :
\begin{align*}
\overrightarrow{AB} &= \overrightarrow{AO} +
\overrightarrow{OB}\\
\overrightarrow{OB} &= \overrightarrow{AB} -
\overrightarrow{AO}\\
\overrightarrow{OB} &= \overrightarrow{AB} +
\overrightarrow{OA}\\
\overrightarrow{OB} &= \vec{j} - \vec{i} + \vec{i} + \vec{j}\\
\overrightarrow{OB} &= 2\vec{j}
\end{algin*}
Par Pythagore :
\[OB^2 = OA^2 + AB^2\]
En effet :
\begin{align*}
OB^2 &= 0^2 + 2^2 = 4\\
OA^2 + AB^2 &= 1^2 + 1^2 + (0 - 1)^2 + (2 - 1)^2 = 4
\end{align*}
Le triangle OAB est isocèle et rectangle en A.

Voir figure :
\begin{center}
\includegraphics[width=.9\linewidth]{./img/sol16q2.png}
\end{center}

\item Placer le point C tel que
\[\overrightarrow{OC} = 2\overrightarrow{AB}\]
Une égalité vectorielle plus simple pour placer le
point C est
\[\overrightarrow{BC} = -2\vec{i}\]
On la trouve en utilisant Chasles :
\begin{align*}
\overrightarrow{OC} &= \overrightarrow{OB} + \overrightarrow{BC}\\
\overrightarrow{BC} &= \overrightarrow{OC} - \overrightarrow{OB}\\
\overrightarrow{BC} &= 2(\vec{j} - \vec{i}) - 2\vec{j}\\
\overrightarrow{BC} &= -2\vec{i}
\end{align*}
Par Pythagore :
\[OC^2 = OB^2 + BC^2\]
En effet :
\begin{align*}
OC^2 &= (-2)^2 + 2^2 = 8\\
OB^2 + BC^2 &= 0^2 + 2^2 + (-2)^2 + 0^2 = 8
\end{align*}

Voir figure :
\begin{center}
\includegraphics[width=.9\linewidth]{./img/sol16q3.png}
\end{center}

\item Placer le point D tel que
\[\overrightarrow{OD} = -4\vec{i}\]
Le triangle COD est isocèle et rectangle en C :
\begin{align*}
OC^2 &= (-2)^2 + 2^2 = 8\\
\overrightarrow{CD} &= \overrightarrow{CO} + \overrightarrow{OD}\\
\overrightarrow{CD} &= 2\vec{i} - 2\vec{j} - 4\vec{i}\\
\overrightarrow{CD} &= -2(\vec{i} + \vec{i})\\
CD^2 &= (-2)^2 + (-2)^2 = 8\\
OD^2 &= (-4)^2 + 0^2 = 16
\end{align*}

Voir figure :
\begin{center}
\includegraphics[width=.9\linewidth]{./img/sol16q4.png}
\end{center}

\item Placer le point E tel que
\[\overrightarrow{DE} = -4\vec{j}\]
Le triangle ODE est isocèle est rectangle en D.
Les vecteurs \(\overrightarrow{OA}\) et \(\overrightarrow{OE}\) sont
colinéaires.
\begin{align*}
\overrightarrow{OE} &= \overrightarrow{OD} + \overrightarrow{DE}\\
\overrightarrow{OE} &= -4\vec{i} - 4\vec{j}\\
\overrightarrow{OE} &= -4\overrightarrow{OA}\\
DE^2 &= 0^2 + (-4)^2 = 16\\
OD^2 &= (-4)^2 + 0^2 = 16\\
OE^2 &= (-4)^2 + (-4)^2 = 32
\end{align*}

Voir figure :
\begin{center}
\includegraphics[width=.9\linewidth]{./img/sol16q5.png}
\end{center}

\item Placer le point F tel que
\[\overrightarrow{OF} = -4\overrightarrow{OB}\]
Le triangle EOF est isocèle est rectangle en E.
\begin{align*}
\overrightarrow{EF} &= \overrightarrow{EO} + \overrightarrow{OF}\\
\overrightarrow{EF} &= 4(\vec{i} + \vec{j}) - 4(2\vec{j})\\
\overrightarrow{EF} &= 4(\vec{i} - \vec{j})\\
EF^2 &= 4^2 + (-4)^2 = 32\\
OE^2 &= (-4)^2 + (-4)^2 = 32\\
OF^2 &= (-4)^2\times 2^2 = 64
\end{align*}

Voir figure :
\begin{center}
\includegraphics[width=.9\linewidth]{./img/sol16q6.png}
\end{center}

\item Placer le point G tel que
\[\overrightarrow{FG} = 8\vec{i}\]
Le triangle FOG est isocèle et rectangle en F.
\begin{align*}
OF^2 &= 64\\
FG^2 &= 8^2 + 0^2 = 64\\
OG^2 &= 64 + 64 = 128
\end{align*}

Voir figure :
\begin{center}
\includegraphics[width=.9\linewidth]{./img/sol16q7.png}
\end{center}

\item Calcul des coordonnées des points H, K, L, M, N, P milieux
respectifs des segments [OA], [OC], [OD], [OE], [OF], [OG] :

\begin{align*}
H\begin{pmatrix}\frac{1}{2}\\ \frac{1}{2}\end{pmatrix} &&
K\begin{pmatrix}-1\\ 1\end{pmatrix}\\
L\begin{pmatrix}-2\\ 0\end{pmatrix} &&
M\begin{pmatrix}-2\\ -2\end{pmatrix}\\
N\begin{pmatrix}0\\ -4\end{pmatrix} &&
P\begin{pmatrix}4\\ -4\end{pmatrix}
\end{align*}

Voir figure :
\begin{center}
\includegraphics[width=.9\linewidth]{./img/sol16q8.png}
\end{center}

\item Voir figure :
\begin{center}
\includegraphics[width=.9\linewidth]{./img/sol16q9.png}
\end{center}

\item On remarque que les cercles sont circonscrits aux triangles
respectifs.

Voir figure :
\begin{center}
\includegraphics[width=.9\linewidth]{./img/sol16q10.png}
\end{center}
\end{enumerate}




\part{Solutions des exercices divers}
\label{sec:org9ffc104}
\chapter{Solutions des paradoxes}
\label{sec:org24a1e64}
\section{Solution de l'exercice 77}
\label{sec:org14d58c3}

\begin{enumerate}
\item Placement des points O(0 ; 0) et A(1 ; 0) ainsi que la
construction du vecteur
\[\vec{u}_1 = \overrightarrow{OA}\]

\begin{center}
\includegraphics[width=.9\linewidth]{./img/sol77q1.png}
\end{center}
\item Placement du point B(3 ; 1) et construction du vecteur
\[\vec{v}_1 = \overrightarrow{OB}\]

\begin{center}
\includegraphics[width=.9\linewidth]{./img/sol77q2.png}
\end{center}
\item Placement des points C(0 ; 3) et D(2 ; 7) ainsi que la
construction du vecteur
\[\vec{u}_2 = \overrightarrow{CD}\]

\begin{center}
\includegraphics[width=.9\linewidth]{./img/sol77q3.png}
\end{center}
\item Placement du point E(0 ; 4) et construction du vecteur
\[\vec{v}_2 = \overrightarrow{CE}\]

\begin{center}
\includegraphics[width=.9\linewidth]{./img/sol77q4.png}
\end{center}
\item \textbf{Rappel} : La pente d'un vecteur se calcule en faisant le
rapport entre son ordonnée (déplacement vertical) divisée par
son abscisse (déplacement horizontal).

Concrètement :
\begin{align*}
p_{\vec{u}_1} &= \dfrac{y_{\vec{u}_1}}{x_{\vec{u}_1}} = \dfrac{y_A - y_O}{x_A - x_O}\\
p_{\vec{u}_1} &= \dfrac{0 - 0}{1 - 0} = 0\\
p_{\vec{v}_1} &= \dfrac{y_{\vec{v}_1}}{x_{\vec{v}_1}} = \dfrac{y_B - y_O}{x_B - x_O}\\
p_{\vec{v}_1} &= \dfrac{1 - 0}{3 - 0} = \dfrac{1}{3}\\
&\Rightarrow p_{\vec{u}_1} = 0 < p_{\vec{v}_1} = \dfrac{1}{3}
\end{align*}

Les calculs confirment ce qu'on peut observer graphiquement, le
vecteur \(\vec{v}_1\) est plus incliné vers le haut
verticalement que le vecteur \(\vec{u}_1\).

\item \textbf{Rappel} : La pente d'un vecteur se calcule en faisant le
rapport entre son ordonnée (déplacement vertical) divisée par
son abscisse (déplacement horizontal).

Concrètement :
\begin{align*}
p_{\vec{u}_2} &= \dfrac{y_{\vec{u}_2}}{x_{\vec{u}_2}} = \dfrac{y_D - y_C}{x_D - x_C}\\
p_{\vec{u}_2} &= \dfrac{7 - 3}{2 - 0} = \dfrac{4}{2} = 2\\
p_{\vec{v}_2} &= \dfrac{y_{\vec{v}_2}}{x_{\vec{v}_2}} = \dfrac{y_E - y_C}{x_E - x_C}\\
p_{\vec{v}_2} &= \dfrac{4 - 3}{0 - 0} = +\infty \\
&\Rightarrow p_{\vec{u}_2} = 2 < p_{\vec{v}_2} = +\infty
\end{align*}

Les calculs confirment ce qu'on peut observer graphiquement, le
vecteur \(\vec{v}_2\) est plus incliné vers le haut
verticalement que le vecteur \(\vec{u}_2\).

\item Placement des points F(4 ; 2) et G(7 ; 4) ainsi que la
construction du vecteur
\[\vec{v} = \vec{v}_1 + \vec{v}_2 = \overrightarrow{FG}\]

\begin{center}
\includegraphics[width=.9\linewidth]{./img/sol77q7.png}
\end{center}

\item Placement du point H(7 ; 6) et construction du vecteur
\[\vec{u} = \vec{u}_1 + \vec{u}_2 = \overrightarrow{FH}\]

\begin{center}
\includegraphics[width=.9\linewidth]{./img/sol77q8.png}
\end{center}

\item \textbf{Rappel} : La pente d'un vecteur se calcule en faisant le
rapport entre son ordonnée (déplacement vertical) divisée par
son abscisse (déplacement horizontal).

Concrètement :
\begin{align*}
p_{\vec{u}} &= \dfrac{y_{\vec{u}}}{x_{\vec{u}}} = \dfrac{y_H - y_F}{x_H - x_F}\\
p_{\vec{u}} &= \dfrac{6 - 2}{7 - 4} = \dfrac{4}{3}\\
p_{\vec{v}} &= \dfrac{y_{\vec{v}}}{x_{\vec{v}}} = \dfrac{y_G - y_F}{x_G - x_F}\\
p_{\vec{v}} &= \dfrac{4 - 2}{7 - 4} = \dfrac{2}{3} \\
&\Rightarrow p_{\vec{u}} = \dfrac{4}{3} > p_{\vec{v}} = \dfrac{2}{3}
\end{align*}

Les calculs confirment ce qu'on peut observer graphiquement, le
vecteur \(\vec{u}\) est plus incliné vers le haut
verticalement que le vecteur \(\vec{v}\).

\begin{center}
\includegraphics[width=.9\linewidth]{./img/sol77q9.png}
\end{center}
\end{enumerate}


\chapter{Solutions des exercices d'arithmétique}
\label{sec:org5d1125a}
\section{Solution de l'exercice 78}
\label{sec:org7afba87}

On se place dans le plan muni d'un repère \((O ; \vec{i} ,
    \vec{j})\) avec \((\vec{i} , \vec{j})\) la base canonique.

\begin{enumerate}
\item Pour \(n\in\{1, 2, 3, 4, 5, 6\}\) tracer les 6 vecteurs
\[\vec{u}_n\begin{pmatrix}n\\ 6\end{pmatrix}\]

\begin{center}
\includegraphics[width=.9\linewidth]{./img/sol78q1.png}
\end{center}
\item Calcul des pentes respectives \(p_n\) :
\begin{align*}
p_1 &= \dfrac{6}{1} = 6\\
p_2 &= \dfrac{6}{2} = 3\\
p_3 &= \dfrac{6}{3} = 2\\
p_4 &= \dfrac{6}{4} = \dfrac{3}{2} = 1,5\\
p_5 &= \dfrac{6}{5} = 1,2\\
p_6 &= \dfrac{6}{6} = 1
\end{align*}
\item Les pentes \(p_1, p_2, p_3, p_6\) sont des entiers.
\item On en déduit que les abscisses des vecteurs dont la pente est
un entier sont des diviseurs de 6.
\item Tracer les 6 vecteurs
\[\vec{v}_n\begin{pmatrix}6\\ n\end{pmatrix}\]

\begin{center}
\includegraphics[width=.9\linewidth]{./img/sol78q5.png}
\end{center}

On remarque que \(\vec{v}_6 = \vec{u}_6\).
\item Calcul des pentes respectives \(q_n\) :
\begin{align*}
q_1 &= \dfrac{1}{6}\\
q_2 &= \dfrac{2}{6} = \dfrac{1}{3}\\
q_3 &= \dfrac{3}{6} = \dfrac{1}{2}\\
q_4 &= \dfrac{4}{6} = \dfrac{2}{3}\\
q_5 &= \dfrac{5}{6}\\
q_6 &= \dfrac{6}{6} = 1
\end{align*}
\item On peut obtenir les \(q_n) à partir des \(p_n\) en appliquant
la fonction inverse.
\item Concrètement la relation entre les \(p_n\) et les \(q_n\) est :
\[q_n = \dfrac{1}{p_n}\]
\item Il y a exactement 4 vecteurs du plan dans le cadrant supérieur
droit à coordonnées entières tels que leur produit vaut 6.
\item Les vecteurs à coordonnées entières vérifiant les
conditions de la question précédente sont :

\begin{center}
\includegraphics[width=.9\linewidth]{./img/sol78q10.png}
\end{center}
\end{enumerate}


\section{Solutions de l'exercice 79}
\label{sec:orgcaf4831}

On se place dans le plan muni d'un repère \((O ; \vec{i} ,
    \vec{j})\) avec \((\vec{i} , \vec{j})\) la base canonique.

\begin{enumerate}
\item Pour \(n\in\{1, 2, \dots , 12\}\) voici les 12 vecteurs
\[\vec{u}_n\begin{pmatrix}n\\ 12\end{pmatrix}\]

\begin{center}
\includegraphics[width=.9\linewidth]{./img/sol79q1.png}
\end{center}
\item Calcul des pentes respectives \(p_n\) :
\begin{align*}
p_1 &= \dfrac{12}{1} = 12\\
p_2 &= \dfrac{12}{2} = 6\\
p_3 &= \dfrac{12}{3} = 4\\
p_4 &= \dfrac{12}{4} = 3\\
p_5 &= \dfrac{12}{5} = 2,4\\
p_6 &= \dfrac{12}{6} = 2\\
p_7 &= \dfrac{12}{7} \\
p_8 &= \dfrac{12}{8} = \dfrac{3}{2} = 1,5\\
p_9 &= \dfrac{12}{9} = \dfrac{4}{3}\\
p_{10} &= \dfrac{12}{10} = \dfrac{6}{5} = 1,2\\
p_{11} &= \dfrac{12}{11}\\
p_{12} &= \dfrac{12}{12} = 1
\end{align*}

\item Les pentes \(p_1, p_2, p_3, p_4, p_6, p_{12}\) sont des nombres entiers.
\item On en déduit que les abscisses des vecteurs dont la pente est
un entier sont les diviseurs de 12.
\item Voici les 12 vecteurs
\[\vec{v}_n\begin{pmatrix}12\\ n\end{pmatrix}\]

\begin{center}
\includegraphics[width=.9\linewidth]{./img/sol79q5.png}
\end{center}
On remarque que \(\vec{v}_{12} = \vec{u}_{12}\).
\item Calcul des pentes respectives \(q_n\) :
\begin{align*}
q_1 &= \dfrac{1}{12}\\
q_2 &= \dfrac{2}{12} = \dfrac{1}{6}\\
q_3 &= \dfrac{3}{12} = \dfrac{1}{4}\\
q_4 &= \dfrac{4}{12} = \dfrac{1}{3}\\
q_5 &= \dfrac{5}{12}\\
q_6 &= \dfrac{6}{12} = \dfrac{1}{2} = 0,5\\
q_7 &= \dfrac{7}{12} \\
q_8 &= \dfrac{8}{12} = \dfrac{2}{3}\\
q_9 &= \dfrac{9}{12} = \dfrac{3}{4} = 0,75\\
q_{10} &= \dfrac{10}{12} = \dfrac{5}{6}\\
q_{11} &= \dfrac{11}{12}\\
q_{12} &= \dfrac{12}{12} = 1
\end{align*}
\item On peut les obtenir à partir des \(p_n\) en appliquant la
fonction inverse.
\item La relation entre les \(p_n\) et les \(q_n\) est \[q_n = \dfrac{1}{p_n}\]
\item Il y a 6 vecteurs du plan dans le cadrant supérieur
droit de coordonnées entières tels que leur produit vaut 12.
\item Voici tous les vecteurs à coordonnées entières vérifiant les
conditions de la question précédente :

\begin{center}
\includegraphics[width=.9\linewidth]{./img/sol79q10.png}
\end{center}
\end{enumerate}



\section{Solutions de l'exercice 80}
\label{sec:org58da2f8}

On poursuit la même logique que les deux exercices précédents.

Mais cette fois on va calculer sans représenter les vecteurs parce
qu'on va prendre \(n = 60\).

\begin{enumerate}
\item Pour \(n\in\{1, 2, \dots , 60\}\) les vecteurs
\[\vec{u}_n\begin{pmatrix}n\\ 12\end{pmatrix}\]
qui ont une pente entière sont :
\begin{align*}
\vec{u}_1\begin{pmatrix}1\\ 60\end{pmatrix}\Rightarrow p_1 = 60 &&
\vec{u}_2\begin{pmatrix}2\\ 60\end{pmatrix}\Rightarrow p_2 = 30\\
\vec{u}_3\begin{pmatrix}3\\ 60\end{pmatrix}\Rightarrow p_3 = 20 &&
\vec{u}_4\begin{pmatrix}4\\ 60\end{pmatrix}\Rightarrow p_4 = 15\\
\vec{u}_5\begin{pmatrix}5\\ 60\end{pmatrix}\Rightarrow p_5 = 12 &&
\vec{u}_6\begin{pmatrix}6\\ 60\end{pmatrix}\Rightarrow p_6 = 10\\
\vec{u}_{10}\begin{pmatrix}10\\ 60\end{pmatrix}\Rightarrow p_{10} = 6 &&
\vec{u}_{12}\begin{pmatrix}12\\ 60\end{pmatrix}\Rightarrow p_{12} = 5\\
\vec{u}_{15}\begin{pmatrix}15\\ 60\end{pmatrix}\Rightarrow p_{15} = 4 &&
\vec{u}_{20}\begin{pmatrix}20\\ 60\end{pmatrix}\Rightarrow p_{20} = 3\\
\vec{u}_{30}\begin{pmatrix}30\\ 60\end{pmatrix}\Rightarrow p_{30} = 2 &&
\vec{u}_{60}\begin{pmatrix}60\\ 60\end{pmatrix}\Rightarrow p_{60} = 1
\end{align*}
\item On en déduit que les abscisses des vecteurs dont la
pente est un entier sont les diviseurs de 60.
\item Il y a 12 vecteurs du plan dans le cadrant supérieur
droit de coordonnées entières tels que leur produit vaut 60 :
\begin{align*}
\vec{d}_1\begin{pmatrix}1\\ 60\end{pmatrix} && \vec{d}_{12}\begin{pmatrix}60\\ 1\end{pmatrix}\\
\vec{d}_2\begin{pmatrix}2\\ 30\end{pmatrix} && \vec{d}_{11}\begin{pmatrix}30\\ 2\end{pmatrix}\\
\vec{d}_3\begin{pmatrix}3\\ 20\end{pmatrix} && \vec{d}_{10}\begin{pmatrix}20\\ 3\end{pmatrix}\\
\vec{d}_4\begin{pmatrix}4\\ 15\end{pmatrix} && \vec{d}_9\begin{pmatrix}15\\ 4\end{pmatrix}\\
\vec{d}_5\begin{pmatrix}5\\ 12\end{pmatrix} && \vec{d}_8\begin{pmatrix}12\\ 5\end{pmatrix}\\
\vec{d}_6\begin{pmatrix}6\\ 10\end{pmatrix} && \vec{d}_7\begin{pmatrix}10\\ 6\end{pmatrix}
\end{align*}
\end{enumerate}
\end{document}
