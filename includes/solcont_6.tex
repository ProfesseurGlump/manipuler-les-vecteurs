% Created 2025-12-20 Sat 12:10
% Intended LaTeX compiler: pdflatex
\documentclass[11pt]{article}
\usepackage{fontspec}
\usepackage[french, american]{babel}
\usepackage{amsmath}
\usepackage{amssymb}
\usepackage{graphicx}
\usepackage{hyperref}
\usepackage[AUTO]{babel}
\usepackage{minted}
\author{Digital Nomad}
\date{\today}
\title{}
\hypersetup{
 pdfauthor={Digital Nomad},
 pdftitle={},
 pdfkeywords={},
 pdfsubject={},
 pdfcreator={Emacs 29.4 (Org mode 9.6.15)}, 
 pdflang={English}}
\begin{document}

\tableofcontents

\section{Solution de l'exercice 7}
\label{sec:org951ece4}

\begin{enumerate}
\item Voir figure :
\begin{center}
\includegraphics[width=.9\linewidth]{./img/sol7q1.png}
\end{center}
\item On peut voir sur la figure que les points A et B sont sur l'axe
des abscisses donc les vecteurs \(\overrightarrow{OA}\) et
\(\overrightarrow{OB}\) sont colinéaires. Concrètement :
\[\overrightarrow{OB} = \dfrac{3}{2}\overrightarrow{OA}\]
\item De même on peut voir sur la figure que les points C et D sont sur l'axe
des ordonnées donc les vecteurs \(\overrightarrow{OC}\) et
\(\overrightarrow{OD}\) sont colinéaires. Concrètement :
\[\overrightarrow{OD} = \dfrac{3}{2}\overrightarrow{OC}\]
\item Voir figure :
\begin{center}
\includegraphics[width=.9\linewidth]{./img/sol7q4.png}
\end{center}
\item D'une part on a :
\begin{align*}
\overrightarrow{DF} &= \overrightarrow{DO} + \overrightarrow{OF}\\
\overrightarrow{DF} &= \vec{v} - 3\vec{j}\\
\overrightarrow{DF} &= 2\vec{i} + 3\vec{j} - 3\vec{j}\\
\overrightarrow{DF} &= 2\vec{i}
\end{align*}

D'autre part on a :
\begin{align*}
\overrightarrow{CE} &= \overrightarrow{CO} + \overrightarrow{OE}\\
\overrightarrow{CE} &= \vec{u} - 2\vec{j}\\
\overrightarrow{CE} &= 3\vec{i} + 2\vec{j} - 2\vec{j}\\
\overrightarrow{CE} &= 3\vec{i}
\end{align*}

Ainsi :
\[\overrightarrow{CE} = \dfrac{3}{2}\overrightarrow{DF}\]
\end{enumerate}

\section{Solution du programme 6}
\label{sec:orga364b8f}

\inputminted{python}{../code/prog_6.py}

\section{Solution du QCM d'auto-évaluation}
\label{sec:orgac7519c}

\begin{enumerate}
\item Si on multiplie un vecteur par un nombre réel supérieur à 1
alors :

\begin{enumerate}
\item Le vecteur change de direction.
\item \textbf{Le vecteur augmente sa norme. (Bonne réponse)}
\item Le vecteur change de sens.
\item Le vecteur reste identique.
\end{enumerate}
\item Si on multiplie un vecteur par un nombre réel inférieur à -1
alors :

\begin{enumerate}
\item Le vecteur change de direction.
\item \textbf{Le vecteur augmente sa norme. (Bonne réponse)}
\item \textbf{Le vecteur change de sens. (Bonne réponse)}
\item Le vecteur reste identique.
\end{enumerate}
\item Si on multiplie un vecteur par un nombre réel supérieur à -1 et
inférieur à 1 alors :

\begin{enumerate}
\item Le vecteur change de direction.
\item \textbf{Le vecteur diminue sa norme. (Bonne réponse)}
\item Le vecteur change de sens.
\item Le vecteur reste identique.
\end{enumerate}
\item Si on multiplie un vecteur par un nombre réel alors :

\begin{enumerate}
\item Le vecteur obtenu n'est pas colinéaire au vecteur initial.
\item \textbf{Le vecteur obtenu est colinéaire au vecteur initial. (Bonne
réponse)}
\end{enumerate}
\end{enumerate}
\end{document}
