% Created 2025-12-20 Sat 12:33
% Intended LaTeX compiler: pdflatex
\documentclass[11pt]{article}
\usepackage{fontspec}
\usepackage{amsmath}
\usepackage{amssymb}
\usepackage{graphicx}
\usepackage{hyperref}
\usepackage{minted}
\usepackage{tcolorbox}
\usepackage{geometry}
\usepackage{fancyhdr}
\usepackage{minted}
\author{Digital Nomad}
\date{\today}
\title{}
\hypersetup{
 pdfauthor={Digital Nomad},
 pdftitle={},
 pdfkeywords={},
 pdfsubject={},
 pdfcreator={Emacs 29.4 (Org mode 9.6.15)}, 
 pdflang={English}}
\begin{document}

\tableofcontents

\section{Solution de l'exercice 3}
\label{sec:org2195416}

\begin{enumerate}
\item Puisque D est l'image du point C par la translation de vecteur
\[\vec{u} = \overrightarrow{AB}\]
alors :
\begin{align*}
\overrightarrow{CD} &= \vec{u}\\
\overrightarrow{AB} &= \overrightarrow{CD}
\end{align*}
Donc ABDC est un parallélogramme.

Voir figure :
\begin{center}
\includegraphics[width=.9\linewidth]{./img/sol3q1.png}
\end{center}
\item On sait que
\[\overrightarrow{DC} = -\overrightarrow{CD}\]
donc
\begin{align*}
\vec{w} &= \overrightarrow{AB} + \overrightarrow{DC} \\
\vec{w} &= \overrightarrow{AB} - \overrightarrow{CD} \\
\vec{w} &= \vec{u} - \vec{u}\\
\vec{w} &= \vec{0}
\end{align*}
Ainsi on remarque que \(\vec{w}\) est le vecteur nul.
\item Puisque E est l'image de D par la translation de vecteur
\[\vec{v} = \overrightarrow{AC}\]
alors
\[\overrightarrow{DE} = \vec{v}\]
donc
\begin{align*}
\overrightarrow{BD} + \overrightarrow{ED} &= \overrightarrow{AC} - \overrightarrow{DE}\\
\overrightarrow{BD} + \overrightarrow{ED} &= \vec{v} - \vec{v}\\
\overrightarrow{BD} + \overrightarrow{ED} &= \vec{0}
\end{align*}

Voir figure :
\begin{center}
\includegraphics[width=.9\linewidth]{./img/sol3q3.png}
\end{center}
\end{enumerate}


\section{Solution programme 2}
\label{sec:orgdfd9941}

\inputminted{python}{../code/prog_2.py}



\section{Solution du QCM d'auto-évaluation}
\label{sec:orgd32bde7}

\begin{enumerate}
\item Deux vecteurs \(\vec{u}\) et \(\vec{v}\) sont égaux si :

\begin{enumerate}
\item Ils ont la même direction.
\item Ils ont la même direction et le même sens.
\item Ils ont la même direction et la même norme.
\item \textbf{Ils ont la même direction, le même sens et la même
norme. (Bonne réponse)}
\end{enumerate}

\item On dit qu'un vecteur est nul si :

\begin{enumerate}
\item Sa direction est horizontale.
\item Sa direction est verticale.
\item Il va dans un sens puis dans l'autre.
\item \textbf{Sa norme vaut zéro. (Bonne réponse)}
\end{enumerate}
\end{enumerate}
\end{document}
