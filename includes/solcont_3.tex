% Created 2025-12-20 Sat 12:06
% Intended LaTeX compiler: pdflatex
\documentclass[11pt]{article}
\usepackage{fontspec}
\usepackage[french, american]{babel}
\usepackage{amsmath}
\usepackage{amssymb}
\usepackage{graphicx}
\usepackage{hyperref}
\usepackage[AUTO]{babel}
\usepackage{minted}
\author{Digital Nomad}
\date{\today}
\title{}
\hypersetup{
 pdfauthor={Digital Nomad},
 pdftitle={},
 pdfkeywords={},
 pdfsubject={},
 pdfcreator={Emacs 29.4 (Org mode 9.6.15)}, 
 pdflang={English}}
\begin{document}

\tableofcontents

\section{Solution de l'exercice 4}
\label{sec:org67cfd94}

\begin{enumerate}
\item D'après la relation de Chasles :
\[\overrightarrow{AB} + \overrightarrow{BC} =
      \overrightarrow{AC}\]
\item D'après la relation de Chasles :
\[\overrightarrow{AB} + \overrightarrow{BD} =
      \overrightarrow{AD}\]
Or par construction :
\[\overrightarrow{AD} = \overrightarrow{AB} +
      \overrightarrow{AC}\]
Donc
\[\overrightarrow{BD} = \overrightarrow{AC}\]
Ainsi ABDC est un parallélogramme.

Voir figure :
\begin{center}
\includegraphics[width=.9\linewidth]{./img/sol4q2.png}
\end{center}
\item Dans un parallélogramme les diagonales se coupent en leur milieu
donc O est le milieu de [AD] et [BC]. Cette information est
inutile pour cette question mais elle le sera pour la question
suivante.

En utilisant Chasles ou la question précédente (avec la remarque
sur le milieu), on peut trouver plusieurs somme permettant
d'obtenir le vecteur \(\overrightarrow{AO}\) :
\begin{align*}
\overrightarrow{AO} &= \overrightarrow{AB} + \overrightarrow{BO}\\
\overrightarrow{AO} &= \overrightarrow{AC} + \overrightarrow{CO} \\
\overrightarrow{AO} &= \dfrac{1}{2}(\overrightarrow{AB} + \overrightarrow{AC})
\end{align*}
\item Avant de calculer cette somme vectorielle il faut réarranger
l'ordre des vecteurs et utiliser la remarque concernant les
milieux. En effet, puisque O est le milieu du segment [AD] alors
\[\overrightarrow{AO} = \overrightarrow{OD}\]
de même puisque O est le milieu du segment [BC] alors
\[\overrightarrow{BO} = \overrightarrow{OC}\]
Par conséquent :
\begin{align*}
\overrightarrow{AO} + \overrightarrow{BO} + \overrightarrow{CO} + \overrightarrow{DO} &= \overrightarrow{AO} + \overrightarrow{DO} + \overrightarrow{BO} + \overrightarrow{CO}\\
\overrightarrow{AO} + \overrightarrow{BO} + \overrightarrow{CO} + \overrightarrow{DO} &= \overrightarrow{AO} - \overrightarrow{OD} + \overrightarrow{BO} - \overrightarrow{OC}\\
\overrightarrow{AO} + \overrightarrow{BO} + \overrightarrow{CO} + \overrightarrow{DO} &= \overrightarrow{AO} - \overrightarrow{AO} + \overrightarrow{BO} - \overrightarrow{BO}\\
\overrightarrow{AO} + \overrightarrow{BO} + \overrightarrow{CO} + \overrightarrow{DO} &= \vec{0}
\end{align*}

Voir figure :
\begin{center}
\includegraphics[width=.9\linewidth]{./img/sol4q4.png}
\end{center}
\end{enumerate}

\section{Solution programme 3}
\label{sec:orgd636287}

\inputminted{python}{../code/prog_3.py}



\section{Solution du QCM d'auto-évaluation}
\label{sec:org10b9c2f}

\begin{enumerate}
\item Ajouter deux vecteurs revient à :

\begin{enumerate}
\item \textbf{enchaîner deux translations successives (Bonne réponse)}
\item faire une rotation
\item faire une symétrie
\item faire une homothétie
\end{enumerate}
\item La relation de Chasles :

\begin{enumerate}
\item augmente la norme d'un vecteur
\item \textbf{décompose un vecteur en sommes de vecteurs (Bonne réponse)}
\item consiste à passer un coup de fil à Michel
\item revient à faire une transformation géométrique sur un
vecteur
\end{enumerate}
\end{enumerate}
\end{document}
