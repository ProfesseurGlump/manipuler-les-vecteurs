% Created 2025-12-20 Sat 12:11
% Intended LaTeX compiler: pdflatex
\documentclass[11pt]{article}
\usepackage{fontspec}
\usepackage[french, american]{babel}
\usepackage{amsmath}
\usepackage{amssymb}
\usepackage{graphicx}
\usepackage{hyperref}
\usepackage[AUTO]{babel}
\usepackage{minted}
\author{Digital Nomad}
\date{\today}
\title{}
\hypersetup{
 pdfauthor={Digital Nomad},
 pdftitle={},
 pdfkeywords={},
 pdfsubject={},
 pdfcreator={Emacs 29.4 (Org mode 9.6.15)}, 
 pdflang={English}}
\begin{document}

\tableofcontents

\section{Solution de l'exercice 8}
\label{sec:org6114375}

On reprend la configuration finale de l'exercice 7.

Voir figure :

\begin{center}
\includegraphics[width=.9\linewidth]{./img/sol7q4.png}
\end{center}

\begin{enumerate}
\item Calculs de déterminants :
\begin{align*}
d_1 &= det(\vec{i}, \vec{j}) = \begin{vmatrix}1&0\\0&1\end{vmatrix} = 1\times 1 - 0\times 0 = 1\\
d_2 &= det(\vec{u}, \vec{v}) = \begin{vmatrix}3&2\\2&3\end{vmatrix} = 3\times 3 - 2\times 2 = 5\\
d_3 &= det(\overrightarrow{OA}, \overrightarrow{OB}) = \begin{vmatrix}2&3\\0&0\end{vmatrix} = 2\times 0 - 0\times 3 = 0\\
d_4 &= det(\overrightarrow{OC}, \overrightarrow{OD}) = \begin{vmatrix}0&0\\2&3\end{vmatrix} = 0\times 3 - 2\times 0 = 0 \\
d_5 &= det(\overrightarrow{OA}, \overrightarrow{OC}) = \begin{vmatrix}2&0\\0&2\end{vmatrix} = 2\times 2 - 0\times 0 = 4 \\
d_6 &= det(\overrightarrow{OB}, \overrightarrow{OD}) = \begin{vmatrix}3&0\\0&3\end{vmatrix} = 3\times 3 - 0\times 0 = 9
\end{align*}
\item En utilisant le déterminant montrons que les vecteurs
\(\overrightarrow{DF}\) et \(\overrightarrow{CE}\) sont
colinéaires :
\begin{align*}
 det(\overrightarrow{DF}, \overrightarrow{CE}) &= \begin{vmatrix}2&3\\0&0\end{vmatrix}\\
 det(\overrightarrow{DF}, \overrightarrow{CE}) &= 2\times 0 - 0\times 3\\
 det(\overrightarrow{DF}, \overrightarrow{CE}) &= 0
\end{align*}
 Or \(\lvert\lvert\overrightarrow{DF}\rvert\rvert = 2\) et
\(\lvert\lvert\overrightarrow{CE}\rvert\rvert = 3\).

On en déduit que le quadrilatère DCEF est un trapèze.
\item Faisons de même pour \(\overrightarrow{AF}\) et
\(\overrightarrow{BE}\) et le quadrilatère ABEF.
\begin{align*}
  det(\overrightarrow{AF}, \overrightarrow{BE}) &= \begin{vmatrix}0&0\\3&2\end{vmatrix}\\
  det(\overrightarrow{AF}, \overrightarrow{BE}) &= 0\times 2 - 3\times 0\\
  det(\overrightarrow{AF}, \overrightarrow{BE}) &= 0
 \end{align*}
 Or \(\lvert\lvert\overrightarrow{AF}\rvert\rvert = 3\) et
\(\lvert\lvert\overrightarrow{BE}\rvert\rvert = 2\).

On en déduit que le quadrilatère ABFE est un trapèze.
\item Calculons les normes des vecteurs \(\overrightarrow{IF}\) et
\(\overrightarrow{JE}\) :
\begin{align*}
\lvert\lvert\overrightarrow{IF}\rvert\rvert &= \sqrt{(x_F - x_I)^2 + (y_F - y_I)^2}\\
\lvert\lvert\overrightarrow{IF}\rvert\rvert &= \sqrt{(2 - 1)^2 + (3 - 0)^2}\\
\lvert\lvert\overrightarrow{IF}\rvert\rvert &= \sqrt{10}\\
\lvert\lvert\overrightarrow{JE}\rvert\rvert &= \sqrt{(x_E - x_J)^2 + (y_E - y_J)^2}\\
\lvert\lvert\overrightarrow{JE}\rvert\rvert &= \sqrt{(3 - 0)^2 + (2 - 1)^2}\\
\lvert\lvert\overrightarrow{JE}\rvert\rvert &= \sqrt{10}
\end{align*}
\item Comparons les vecteurs \(\overrightarrow{IJ}\) et
\(\overrightarrow{EF}\).
\begin{align*}
\overrightarrow{IJ} &= \overrightarrow{IO} + \overrightarrow{OJ}\\
\overrightarrow{IJ} &= \vec{j} - \vec{i}\\
\overrightarrow{EF} &= \overrightarrow{EO} + \overrightarrow{OF}\\
\overrightarrow{EF} &= \vec{v} - \vec{u}\\
\overrightarrow{EF} &= 2\vec{i} + 3\vec{j} - (3\vec{i} + 2\vec{j})\\
\overrightarrow{EF} &= \vec{j} - \vec{i}
\end{align*}
Ainsi \[\overrightarrow{IJ} = \overrightarrow{EF}\]

 Le quadrilatère IEFJ est donc un parallélogramme.
 Or d'après la question précédente ses diagonales [IF] et [JE] sont
égales.
Par conséquent IEFJ est un rectangle.
\item Puisque G l'intersection des segments [IF] et [JE] et que ce
sont les diagonales d'un rectangle alors G est leur milieu.
Déterminons ses coordonnées :
\begin{align*}
G&\begin{pmatrix}\frac{x_I + x_F}{2}\\\frac{y_I + y_F}{2}\end{pmatrix}\\
G&\begin{pmatrix}\frac{3}{2}\\\frac{3}{2}\end{pmatrix}
\end{align*}
Le point H est tel que \(\overrightarrow{OG} =
      \overrightarrow{GH}\) alors en appliquant la relation de Chasles :
\[\overrightarrow{OH} = 2\overrightarrow{OG}\]
Ainsi on obtient les coordonnées de H en doublant celles de G,
H(3 ; 3). 
Étudions la nature du quadrilatère OBHD :
\begin{align*}
\overrightarrow{OH}&= 3\vec{i} + 3\vec{j}\\
\overrightarrow{OB}&= 3\vec{i}\\
\overrightarrow{OD}&= 3\vec{j}\\
\overrightarrow{OH}&= \overrightarrow{OB} + \overrightarrow{OD}
\end{align*}
On vient de prouver que OBHD est un carré. Pourquoi ? Parce que
\(\overrightarrow{OH}\) est la somme de deux vecteurs
orthogonaux de même norme.
\item Le triangle OFE est isocèle en O car les vecteurs \(\vec{u}\) et
\(\vec{v}\) ont même norme :
\begin{align*}
\lvert\lvert\vec{u}\rvert\rvert &= \sqrt{3^2 + 2^2} = \sqrt{13}\\
\lvert\lvert\vec{v}\rvert\rvert &= \sqrt{3^2 + 2^2} = \sqrt{13}
\end{align*}
\item D'après la question 6 on sait que OBHD est un carré et que G est
le milieu de la diagonale OH. Par conséquent G est aussi le
milieu de la diagonale BD. Ainsi l'image de G par la translation
de vecteur \(\overrightarrow{BG}\) est D.
\item Le point K tel que \(\overrightarrow{BK} = \vec{j}\) a pour
coordonnées K(3 ; 1).
\item Le point L tel que \(\overrightarrow{DL} = \vec{i}\) a pour
coordonnées L(1 ; 3).
\end{enumerate}


\begin{center}
\includegraphics[width=.9\linewidth]{./img/sol8q10.png}
\end{center}


\section{Solution du programme 7}
\label{sec:orga7f004e}

\inputminted{python}{../code/prog_7.py}


\section{Solution du QCM d'auto-évaluation}
\label{sec:org7e3f77a}

\begin{enumerate}
\item Considérons les vecteurs
\[\vec{u}_1\begin{pmatrix}x_1\\y_1\end{pmatrix}\] et
\[\vec{u}_2\begin{pmatrix}x_2\\y_2\end{pmatrix}\] alors :

\begin{enumerate}
\item \[det(\vec{u}_1 , \vec{u}_2) = \begin{vmatrix}x_1&x_2\\y_1&y_2\end{vmatrix}
          = x_1x_2 + y_1y_2\]
\item \[det(\vec{u}_1 , \vec{u}_2) = \begin{vmatrix}x_1&x_2\\y_1&y_2\end{vmatrix}
          = x_1x_2 - y_1y_2\]
\item \[det(\vec{u}_1 , \vec{u}_2) = \begin{vmatrix}x_1&x_2\\y_1&y_2\end{vmatrix}
          = x_1y_2 + y_1x_2\]
\item \[det(\vec{u}_1 , \vec{u}_2) = \begin{vmatrix}x_1&x_2\\y_1&y_2\end{vmatrix}
          = x_1y_2 - y_1x_2\] \textbf{(Bonne réponse)}
\end{enumerate}
\item Considérons les mêmes vecteurs que précédemment.

On dira que \(\vec{u}_1\) et \(\vec{u}_2\) sont colinéaires
si :

\begin{enumerate}
\item \(det(\vec{u}_1 , \vec{u}_2) = 1\)
\item \textbf{\(det(\vec{u}_1 , \vec{u}_2) = 0\) (Bonne réponse)}
\item \textbf{il existe un réel \(k\) tel que \(x_1 = kx_2\) et \(y_1 =
          ky_2\) (Bonne réponse)}
\item \textbf{\(\dfrac{x_1}{x_2} = \dfrac{y_1}{y_2}\) (Bonne réponse)}
\end{enumerate}
\end{enumerate}
\end{document}
