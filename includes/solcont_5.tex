% Created 2025-12-20 Sat 12:09
% Intended LaTeX compiler: pdflatex
\documentclass[11pt]{article}
\usepackage{fontspec}
\usepackage[french, american]{babel}
\usepackage{amsmath}
\usepackage{amssymb}
\usepackage{graphicx}
\usepackage{hyperref}
\usepackage[AUTO]{babel}
\usepackage{minted}
\author{Digital Nomad}
\date{\today}
\title{}
\hypersetup{
 pdfauthor={Digital Nomad},
 pdftitle={},
 pdfkeywords={},
 pdfsubject={},
 pdfcreator={Emacs 29.4 (Org mode 9.6.15)}, 
 pdflang={English}}
\begin{document}

\tableofcontents

\section{Solution de l'exercice 6}
\label{sec:orgd5f45f8}

\begin{enumerate}
\item Les coordonnées du vecteur \(\overrightarrow{OA}\) sont les
mêmes que celles du point A d'où la relation :
\[\overrightarrow{OA} = x_A\vec{i} + y_A\vec{j}\]
\item Les coordonnées du vecteur \(\overrightarrow{OB}\) sont les
mêmes que celles du point B d'où la relation :
\[\overrightarrow{OB} = x_B\vec{i} + y_B\vec{j}\]
\item Utilisons la relation de Chasles :
\begin{align*}
\overrightarrow{AB} &= \overrightarrow{AO} + \overrightarrow{OB}\\
\overrightarrow{AB} &= \overrightarrow{OB} - \overrightarrow{OA}
\end{align*}
\item On rassemble les résultats obtenus aux questions précédentes :
\begin{align*}
\overrightarrow{AB} &= \overrightarrow{OB} - \overrightarrow{OA}\\
\overrightarrow{AB} &= x_B\vec{i} + y_B\vec{j} - (x_A\vec{i} + y_A\vec{j})\\
\overrightarrow{AB} &= (x_B - x_A)\vec{i} + (y_B - y_A)\vec{j}
\end{align*}
\item Ainsi on obtient :
\[\overrightarrow{AB}\begin{pmatrix}x_B - x_A\\ y_B - y_A\end{pmatrix}\]
\end{enumerate}


\section{Solution du programme 5}
\label{sec:org4d24a0c}

\inputminted{python}{../code/prog_5.py}


\section{Solution du QCM d'auto-évaluation}
\label{sec:orgf0fb455}

On considère le vecteur \(\overrightarrow{AB}\) dans le plan muni
du repère orthonormé \((O ; \vec{i}, \vec{j})\).

\begin{enumerate}
\item En utilisant Chasles on peut écrire :

\begin{enumerate}
\item \(\overrightarrow{AB} = \overrightarrow{OA} +
          \overrightarrow{OB}\)
\item \(\overrightarrow{AB} = \overrightarrow{OA} - \overrightarrow{OB}\)
\item \(\overrightarrow{AB} = \overrightarrow{AO} +
          \overrightarrow{BO}\)
\item \textbf{\(\overrightarrow{AB} = \overrightarrow{AO} +
          \overrightarrow{OB}\) (Bonne réponse)}
\end{enumerate}
\item En utilisant les coordonnées des points A et B on a :

\begin{enumerate}
\item \(\overrightarrow{AB}\begin{pmatrix}x_A + x_B\\y_A +
          y_B\end{pmatrix}\)
\item \(\overrightarrow{AB}\begin{pmatrix}x_A - x_B\\y_A -
          y_B\end{pmatrix}\)
\item \textbf{\(\overrightarrow{AB}\begin{pmatrix}x_B - x_A\\y_B -
          y_A\end{pmatrix}\) (Bonne réponse)}
\item \(\overrightarrow{AB}\begin{pmatrix}x_A \times x_B\\y_A
          \times y_B\end{pmatrix}\)
\end{enumerate}
\end{enumerate}
\end{document}
