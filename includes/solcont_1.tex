% Created 2025-12-20 Sat 12:56
% Intended LaTeX compiler: pdflatex
\documentclass[11pt]{article}
\usepackage{fontspec}
\usepackage{amsmath}
\usepackage{amssymb}
\usepackage{graphicx}
\usepackage{hyperref}
\usepackage{minted}
\usepackage{tcolorbox}
\usepackage{geometry}
\usepackage{fancyhdr}
\usepackage{minted}
\author{Digital Nomad}
\date{\today}
\title{}
\hypersetup{
 pdfauthor={Digital Nomad},
 pdftitle={},
 pdfkeywords={},
 pdfsubject={},
 pdfcreator={Emacs 29.4 (Org mode 9.6.15)}, 
 pdflang={English}}
\begin{document}

\tableofcontents

\section{Solution de l'exercice 1}
\label{sec:org388d291}

Soient les points A, B, C tels que :
\begin{itemize}
\item \(\vec{u} = \overrightarrow{AB}\)
\item \(\vec{v} = \overrightarrow{AC}\)
\end{itemize}

Voir la figure :

\begin{center}
\includegraphics[width=.9\linewidth]{./img/ABCvect.png}
\end{center}

\begin{enumerate}
\item L'image du point B par la translation de vecteur
\[\vec{v} = \overrightarrow{AC}\]
est le point D tel que
\[\overrightarrow{BD} = \overrightarrow{AC}\]

Voir figure :
\begin{center}
\includegraphics[width=.9\linewidth]{./img/sol1q1.png}
\end{center}
\item L'image du point C par la translation de vecteur
\[\vec{u} = \overrightarrow{AB}\]
est le point E tel que
\[\overrightarrow{CE} = \overrightarrow{AB}\]

Voir figure :
\begin{center}
\includegraphics[width=.9\linewidth]{./img/sol1q2.png}
\end{center}
\item On peut en déduire que D = E car :
\begin{align*}
\overrightarrow{AD} &= \overrightarrow{AB} + \overrightarrow{BD} = \vec{u} + \vec{v}\\
\overrightarrow{AE} &= \overrightarrow{AC} + \overrightarrow{CE} = \vec{v} + \vec{u}\\
\overrightarrow{AD} &= \overrightarrow{AB} + \overrightarrow{AC} = \vec{u} + \vec{v}\\
\overrightarrow{AE} &= \overrightarrow{AC} + \overrightarrow{AB} = \vec{v} + \vec{u}\\
\overrightarrow{AD} &= \overrightarrow{AE}
\end{align*}
\item ABDC est un parallélogramme car (au choix) :
\begin{align*}
\overrightarrow{AB} &= \overrightarrow{CD}\\
\overrightarrow{AC} &= \overrightarrow{BD}
\end{align*}
Une seule égalité suffit.

Voir figure :
\begin{center}
\includegraphics[width=.9\linewidth]{./img/sol1q4.png}
\end{center}
\end{enumerate}


\section{Solution de l'exercice 2}
\label{sec:orgb88fe31}

\begin{enumerate}
\item Par construction, le point C est l'image de B par la translation
de vecteur
\[\vec{u} = \overrightarrow{AB}\]
donc
\[\overrightarrow{BC} = \vec{u}\]
Le vecteur \(\overrightarrow{BC}\) a pour
\begin{itemize}
\item direction : la droite (BC), qui est aussi la droite (AB).
\item sens : de B vers C, qui est aussi de A vers B.
\item norme : la longueur BC, qui est aussi la longueur AB
\end{itemize}

Voir figure :
\begin{center}
\includegraphics[width=.9\linewidth]{./img/sol2q1.png}
\end{center}
\item Le point B représente le milieu du segment [AC].
\item On a :
\begin{align*}
\vec{u} &= \overrightarrow{AB} \\
\overrightarrow{BC} &= \vec{u} \\
\overrightarrow{AC} &= \overrightarrow{AB} + \overrightarrow{BC} \\
\overrightarrow{AB} &= \overrightarrow{BC} = \dfrac{1}{2}\overrightarrow{AC}
\end{align*}
\end{enumerate}


\section{Solution programme 1}
\label{sec:org57561cc}

\begin{minted}[bgcolor=bg,fontsize=\footnotesize,linenos=true]{python}
print("Un vecteur a 3 caractéristiques fondamentales : ")

entries = ["direction", "norme", "sens"]

definitions = [
    "la droite qui le porte.",
    "la distance entre origine et extrémité.",
    "de l'origine vers l'extrémité."
]


for i in range(len(entries)):
    e, d = entries[i], definitions[i]
    print(f"{i + 1}) {e.capitalize()} : {d}")

\end{minted}

\section{Solution du QCM d'auto-évaluation}
\label{sec:org15432a9}

Lorsqu'on parle du vecteur \(\overrightarrow{MM'}\) associé à
la translation qui transforme M en M'. 

\begin{enumerate}
\item Quelle est la direction ?

\begin{enumerate}
\item Toute droite parallèle à l'axe des abscisses.
\item Uniquement la droite (MM').
\item Toute droite parallèle à l'axe des ordonnées.
\item \textbf{Toute droite parallèle à la droite (MM'). (Bonne réponse)}
\end{enumerate}
\begin{enumerate}
\item Quelle est le sens ?

\begin{enumerate}
\item De O à M.
\item \textbf{De M à M'. (Bonne réponse)}
\item De M' à M.
\item De M à O.
\end{enumerate}
\item Quelle est la norme ?

\begin{enumerate}
\item La distance OM.
\item La distance OM'.
\item \textbf{La distance MM'. (Bonne réponse)}
\item \textbf{La distance M'M. (Bonne réponse)}
\end{enumerate}
\end{enumerate}
\end{enumerate}
\end{document}
