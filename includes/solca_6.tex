% Created 2025-12-20 Sat 11:08
% Intended LaTeX compiler: pdflatex
\documentclass[11pt]{article}
\usepackage{fontspec}
\usepackage[french, american]{babel}
\usepackage{amsmath}
\usepackage{amssymb}
\usepackage{graphicx}
\usepackage{hyperref}
\usepackage[AUTO]{babel}
\usepackage{minted}
\author{Digital Nomad}
\date{\today}
\title{}
\hypersetup{
 pdfauthor={Digital Nomad},
 pdftitle={},
 pdfkeywords={},
 pdfsubject={},
 pdfcreator={Emacs 29.4 (Org mode 9.6.15)}, 
 pdflang={English}}
\begin{document}

\tableofcontents

\section{Solution de l'exercice 14}
\label{sec:orgab7352b}

On reprend la configuration de l'exercice précédent.

Voir la figure :

\begin{center}
\includegraphics[width=.9\linewidth]{./img/figexo13.png}
\end{center}

\begin{enumerate}
\item Le point O étant l'origine du repère on peut facilement
déterminer les coordonnées des vecteurs \(\overrightarrow{OA}\)
et \(\overrightarrow{OC}\) et ainsi déterminer si les points A,
O et C sont alignés :
\begin{align*}
\overrightarrow{OA} &= 2\vec{i} + 3\vec{j}\\
\overrightarrow{OC} &= -2\vec{i} - 3\vec{j}\\
\overrightarrow{OC} &= -\overrightarrow{OA}
\end{align*}

Voir figure :
\begin{center}
\includegraphics[width=.9\linewidth]{./img/sol14q1.png}
\end{center}
\item D'après ce qui précède on a :
\begin{align*}
\lvert\lvert\vec{u}\rvert\rvert &= \lvert\lvert\overrightarrow{OA}\rvert\rvert = \lvert\lvert\overrightarrow{OC}\rvert\rvert \\
\lvert\lvert\vec{u}\rvert\rvert &= \sqrt{13}
\end{align*}
On en déduit que O est le milieu du segment [AC].
\item Calculons le déterminant \(det(\overrightarrow{OD} ,
       \overrightarrow{OB})\) :
\begin{align*}
det(\overrightarrow{OD} , \overrightarrow{OB}) &= \begin{vmatrix}3&-3\\ -2&2\end{vmatrix} = 3\times 2 - (-2)\times (-3) = 0
\end{align*}
Le déterminant est nul donc les vecteurs
\(\overrightarrow{OD}\) et \(\overrightarrow{OB}\) sont
colinéaires. Puisqu'ils ont la même origine on en déduit que
les points B, O et D sont alignés.

Voir figure :
\begin{center}
\includegraphics[width=.9\linewidth]{./img/sol14q3.png}
\end{center}

\item Comparons OB et OD :
\begin{align*}
OB &= \sqrt{(-3)^2 + 2^2} = \sqrt{13}\\
OD &= \sqrt{3^2 + (-2)^2} = \sqrt{13}\\
OB &= OD
\end{align*}
On en déduit que O est le milieu de [BD] et donc que ABCD est
un rectangle car ses diagonales sont de même longueur
\(2\sqrt{13} = \sqrt{52}\).

Voir figure :
\begin{center}
\includegraphics[width=.9\linewidth]{./img/sol14q4.png}
\end{center}

\item Puisque ABCD est un rectangle, c'est donc un parallélogramme
donc on a :
\begin{align*}
\overrightarrow{AB} &= \overrightarrow{DC}\\
\overrightarrow{AB} &= -\overrightarrow{CD}
\end{align*}
Ainsi les vecteurs \(\overrightarrow{AB}\) et
\(\overrightarrow{CD}\) sont colinéaires.

Voir figure :
\begin{center}
\includegraphics[width=.9\linewidth]{./img/sol14q5.png}
\end{center}

\item Comparons AB et AD :
\begin{align*}
AB &= \sqrt{(-3 - 2)^2 + (2 - 3)^2} = \sqrt{26}\\
AD &= \sqrt{(3 - 2)^2 + (-2 - 3)^2} = \sqrt{26}\\
AB &= AD
\end{align*}

Voir figure :
\begin{center}
\includegraphics[width=.9\linewidth]{./img/sol14q6.png}
\end{center}

\item Le quadrilatère ABCD est un carré car c'est un rectangle avec
deux côtés consécutifs de même longueur.

Voir figure :
\begin{center}
\includegraphics[width=.9\linewidth]{./img/sol14q7.png}
\end{center}
\end{enumerate}

\section{Solution du QCM d'auto-évaluation}
\label{sec:orgf1cf39b}

\begin{enumerate}
\item Pour montrer que A, B et C sont alignés il faut :
\begin{enumerate}[label=\alph*.)]
\item que \(\overrightarrow{AB} = \overrightarrow{AC}\)
\item \textbf{qu'il existe un réel k tel que \(\overrightarrow{AB} =
          k\overrightarrow{AC}\) (Bonne réponse)}
\item vérifier que \(\overrightarrow{AB} + \overrightarrow{BC} =
          \overrightarrow{AC}\)
\item \textbf{vérifier que \(det(\overrightarrow{AB} ,
          \overrightarrow{AC}) = 0\) (Bonne réponse)}
\end{enumerate}
\item Pour montrer que les droites (AB) et (CD) sont parallèles il faut :
\begin{enumerate}[label=\alph*.)]
\item \textbf{montrer que les vecteurs \(\overrightarrow{AB}\) et
\(\overrightarrow{CD}\) sont colinéaires (Bonne réponse)}
\item \textbf{vérifier que \(det(\overrightarrow{AB} ,
          \overrightarrow{CD}) = 0\) (Bonne réponse)}
\item montrer que \(\overrightarrow{AB} = \overrightarrow{CD}\)
\item vérifier que \(det(\overrightarrow{AB} ,
          \overrightarrow{CD}) \neq 0\)
\end{enumerate}
\end{enumerate}
\end{document}
