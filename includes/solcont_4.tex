% Created 2025-12-20 Sat 12:07
% Intended LaTeX compiler: pdflatex
\documentclass[11pt]{article}
\usepackage{fontspec}
\usepackage[french, american]{babel}
\usepackage{amsmath}
\usepackage{amssymb}
\usepackage{graphicx}
\usepackage{hyperref}
\usepackage[AUTO]{babel}
\usepackage{minted}
\author{Digital Nomad}
\date{\today}
\title{}
\hypersetup{
 pdfauthor={Digital Nomad},
 pdftitle={},
 pdfkeywords={},
 pdfsubject={},
 pdfcreator={Emacs 29.4 (Org mode 9.6.15)}, 
 pdflang={English}}
\begin{document}

\tableofcontents

\section{Solution de l'exercice 5}
\label{sec:org9d0316c}

\begin{enumerate}
\item Puisque ABCD est un carré alors les droites (AB) et (AD) sont
orthogonales et les longueurs AB et AD sont égales. Par
conséquent les vecteurs sont associés sont orthogonaux et de
même norme. Ainsi
\[(\vec{u} , \vec{v}) = (\overrightarrow{AB} ,
      \overrightarrow{AD})\]
est bien une base orthonormée.

Voir figure :
\begin{center}
\includegraphics[width=.9\linewidth]{./img/sol5q1.png}
\end{center}
\item Pour déterminer les coordonnées des points dans cette base il
faut exprimer les vecteurs partant de l'origine du repère comme
combinaisons linéaires des vecteurs de la base. Tout d'abord il
faut remarquer que l'origine du repère est le point A car il
s'agit de l'origine de chacun des vecteurs de la base. On
obtient ainsi :
\begin{align*}
\overrightarrow{AA} &= 0\times\vec{u} + 0\times\vec{v}\Rightarrow A(0 ; 0)\\
\overrightarrow{AB} &= 1\times\vec{u} + 0\times\vec{v}\Rightarrow B(1 ; 0)\\
\overrightarrow{AC} &= 1\times\vec{u} + 1\times\vec{v}\Rightarrow C(1 ; 1)\\
\overrightarrow{AD} &= 0\times\vec{u} + 1\times\vec{v}\Rightarrow D(0 ; 1)
\end{align*}
\item Calculons la norme du vecteur \(\overrightarrow{AC}\) :
\begin{align*}
\overrightarrow{AC} &= \sqrt{1^2 + 1^2}\\
\overrightarrow{AC} &= \sqrt{2}
\end{align*}
\end{enumerate}

\section{Solution de l'exercice 5 bis}
\label{sec:org38b862e}

On se place dans le plan muni repère orthonormé \((O ; \vec{i} ;
    \vec{j})\). Avec :
\begin{align*}
\vec{i}&\begin{pmatrix}1\\0\end{pmatrix} && \vec{j}\begin{pmatrix}0\\1\end{pmatrix}
\end{align*}

Calculons les normes des vecteurs suivants :

\begin{enumerate}
\item \begin{align*}
 n_1 &= \lvert\lvert \vec{i} + \vec{j}\rvert\rvert\\
 n_1 &= \lvert\lvert \begin{pmatrix}1\\1\end{pmatrix}\rvert\rvert\\
 n_1 &= \sqrt{1^2 + 1^2}\\
 &\Rightarrow n_1 = \sqrt{2}
\end{align*}
Voir figure :
\begin{center}
\includegraphics[width=.9\linewidth]{./img/sol5bisq1.png}
\end{center}
\item \begin{align*}
 n_2 &= \lvert\lvert \vec{i} - \vec{j}\rvert\rvert\\
 n_2 &= \lvert\lvert \begin{pmatrix}1\\ -1 \end{pmatrix}\rvert\rvert\\
 n_2 &= \sqrt{1^2 + (-1)^2}\\
 &\Rightarrow n_2 = \sqrt{2}
\end{align*}
Voir figure :
\begin{center}
\includegraphics[width=.9\linewidth]{./img/sol5bisq2.png}
\end{center}
\item \begin{align*}
 n_3 &= \lvert\lvert 3\vec{i}\rvert\rvert\\
 n_3 &= 3\lvert\lvert \vec{i}\rvert\rvert\\
 &\Rightarrow n_3 = 3
\end{align*}
Voir figure :
\begin{center}
\includegraphics[width=.9\linewidth]{./img/sol5bisq3.png}
\end{center}
\item \begin{align*}
 n_4 &= \lvert\lvert 3\vec{i} + 4\vec{j}\rvert\rvert\\
 n_4 &= \lvert\lvert \begin{pmatrix}3\\4\end{pmatrix}\rvert\rvert\\
 n_4 &= \sqrt{3^2 + 4^2}\\
 &\Rightarrow n_4 = 5
\end{align*}
Voir figure :
\begin{center}
\includegraphics[width=.9\linewidth]{./img/sol5bisq4.png}
\end{center}
\item \begin{align*}
 n_5 &= \lvert\lvert -5\vec{j}\rvert\rvert\\
 n_5 &= |-5|\lvert\lvert \vec{j}\rvert\rvert\\
 &\Rightarrow n_5 = 5
\end{align*}
Voir figure :
\begin{center}
\includegraphics[width=.9\linewidth]{./img/sol5bisq5.png}
\end{center}
\end{enumerate}

\section{Solution de l'exercice 5 ter}
\label{sec:org60f2ca3}

On se place dans le plan muni repère orthonormé \((O ; \vec{i} ;
    \vec{j})\). Avec :
\begin{align*}
\vec{i}&\begin{pmatrix}1\\0\end{pmatrix} && \vec{j}\begin{pmatrix}0\\1\end{pmatrix}
\end{align*}

Calculons et comparons les normes ajoutées séparément
\[s = \lvert\lvert\vec{u}\rvert\rvert + \lvert\lvert\vec{v}\rvert\rvert\]

avec celles des vecteurs sommes

\[e = \lvert\lvert\vec{u} + \vec{v}\rvert\rvert\]

\begin{enumerate}
\item \[(\vec{u}, \vec{v}) = (\vec{i}, \vec{j})\]
\begin{align*}
\lvert\lvert\vec{u}\rvert\rvert &= \lvert\lvert\vec{v}\rvert\rvert = 1\\
s &= 1 + 1 = 2\\
e &= \lvert\lvert\vec{u} + \vec{v}\rvert\rvert\\
e &= \lvert\lvert\begin{pmatrix}1 \\ 1\end{pmatrix}\rvert\rvert\\
e &= \sqrt{1^2 + 1^2} = \sqrt{2}\\
&\Rightarrow s = 2 > e = \sqrt{2}
\end{align*}
Voir figure :
\begin{center}
\includegraphics[width=.9\linewidth]{./img/sol5terq1.png}
\end{center}
\item \[(\vec{u}, \vec{v}) = (\vec{i} + \vec{j}, \vec{i} - \vec{j})\]
\begin{align*}
\lvert\lvert\vec{u}\rvert\rvert &= \lvert\lvert\vec{v}\rvert\rvert = \sqrt{2}\\
s &= \sqrt{2} + \sqrt{2} = 2\sqrt{2}\\
e &= \lvert\lvert\vec{u} + \vec{v}\rvert\rvert\\
e &= \lvert\lvert 2\vec{i}\rvert\rvert = 2\lvert\lvert\vec{i}\rvert\rvert\\
e &= 2\\
&\Rightarrow s = 2\sqrt{2} > e = 2
\end{align*}
Voir figure :
\begin{center}
\includegraphics[width=.9\linewidth]{./img/sol5terq2.png}
\end{center}
\item \[ (\vec{u}, \vec{v}) = (a\vec{i} + b\vec{j}, c\vec{i} + d\vec{j}) \]
\begin{align*}
\lvert\lvert\vec{u}\rvert\rvert &= \sqrt{a^2 + b^2}\\
\lvert\lvert\vec{v}\rvert\rvert &= \sqrt{c^2 + d^2}\\
s &= \lvert\lvert\vec{u}\rvert\rvert + \lvert\lvert\vec{v}\rvert\rvert = \sqrt{a^2 + b^2} + \sqrt{c^2 + d^2}\\
s^2 &= a^2 + b^2 + c^2 + d^2 + 2\sqrt{(a^2 + b^2)(c^2 + d^2)}\\
e &= \lvert\lvert\vec{u} + \vec{v}\rvert\rvert = \lvert\lvert (a + c)\vec{i} + (b + d)\vec{j}\rvert\rvert \\
e &= \sqrt{(a + c)^2 + (b + d)^2} \\
e &= \sqrt{a^2 + b^2 + c^2 + d^2 + 2(ac + bd)}\\
e^2 &= a^2 + b^2 + c^2 + d^2 + 2(ac + bd)\\
s^2 - e^2 &= 2\left(\sqrt{(a^2 + b^2)(c^2 + d^2)} - (ac + bd)\right)\\
(a^2 + b^2)(c^2 + d^2) &= (ac)^2 + (bd)^2 + (ad)^2 + (bc)^2\\
(ac + bd)^2 &= (ac)^2 + (bd)^2 + 2abcd\\
(a^2 + b^2)(c^2 + d^2) &- (ac + bd)^2 = (ad)^2 + (bc)^2 - 2abcd\\
(a^2 + b^2)(c^2 + d^2) &- (ac + bd)^2 = (ad - bc)^2 \geqslant 0\\
&\Rightarrow s^2 - e^2 \geqslant 0\\
&\Rightarrow s^2 \geqslant e^2\\
&\Rightarrow s \geqslant e
\end{align*}
\end{enumerate}

\section{Solution programme 4}
\label{sec:orgeba52a7}

\inputminted{python}{../code/prog_4.py}

\section{Solution du QCM d'auto-évaluation}
\label{sec:org63e08c9}

\begin{enumerate}
\item Une base orthonormée du plan est :

\begin{enumerate}
\item une station de lancement de fusée
\item un couple de vecteurs ayant des directions distinctes
\item \textbf{un couple de vecteurs ayant des directions orthogonales et
la même norme	(Bonne réponse)}
\item un couple de vecteurs ayant la même direction et la même
norme
\end{enumerate}
\item Quelles sont les coordonnées d'un vecteur \(\vec{u}\) dans une
base orthonormée \((\vec{i} , \vec{j})\) ?

\begin{enumerate}
\item \textbf{Le couple de nombres \((a ; b)\) tel que \(\vec{u} =
          a\vec{i} + b\vec{j}\). (Bonne réponse)}
\item Les coordonnées du point M obtenu par la translation de
vecteur \(\vec{u}\) à partir du point O.
\item La somme de celles des vecteurs \(\vec{i}\) et \(\vec{j}\).
\item Les coefficients de toute combinaison linéaire des vecteurs
de la base.
\end{enumerate}
\item Considérons le vecteur \(\vec{u} = a\vec{i} + b\vec{j}\) alors
l'expression de sa norme est :

\begin{enumerate}
\item \(\lvert\lvert\vec{u}\rvert\rvert = a + b\)
\item \(\lvert\lvert\vec{u}\rvert\rvert = a^2 - b^2\)
\item \(\lvert\lvert\vec{u}\rvert\rvert = a^2 + b^2\)
\item \textbf{\(\lvert\lvert\vec{u}\rvert\rvert = \sqrt{a^2 + b^2}\)
(Bonne réponse)}
\end{enumerate}
\end{enumerate}
\end{document}
