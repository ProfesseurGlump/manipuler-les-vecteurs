% Created 2025-12-19 Fri 10:15
% Intended LaTeX compiler: lualatex
\documentclass[a4paper,11pt]{book}
\usepackage[utf8]{inputenc}
\usepackage[T1]{fontenc}
\usepackage{graphicx}
\usepackage{longtable}
\usepackage{wrapfig}
\usepackage{rotating}
\usepackage[normalem]{ulem}
\usepackage{amsmath}
\usepackage{amssymb}
\usepackage{capt-of}
\usepackage{hyperref}
\usepackage[paperwidth=6in,paperheight=9in,margin=1in]{geometry}
\setlength{\headheight}{25pt}
\setcounter{tocdepth}{2} % Limiter la TdM aux subsections
\usepackage{setspace}
\onehalfspacing % Interligne 1,5
\usepackage[most]{tcolorbox}
\newtcolorbox{definition}{colback=blue!5,colframe=blue!50!black,title=Définition}
\newtcolorbox{methode}{colback=green!5,colframe=green!50!black,title=Méthode}
\usepackage{fontspec}
\setmainfont{Libertinus Serif}
\setsansfont{Helvetica Neue}
\setmonofont{Courier}
\usepackage{amsmath,amssymb}
\usepackage{polyglossia}
\setmainlanguage{french}
\setotherlanguage{english}
\usepackage{csquotes}
\MakeAutoQuote{«}{»}
\usepackage{hyperref}
\hypersetup{
colorlinks=true,
linkcolor=blue,
citecolor=green,
filecolor=magenta,
urlcolor=cyan,
pdftitle={Manipuler les vecteurs du plan},
pdfauthor={Laurent Garnier},
}
\usepackage{graphicx}
\graphicspath{{./img/}}
\usepackage{amsmath,amssymb}
\usepackage{enumitem}
\usepackage{tikz}
\usepackage{xcolor}
\definecolor{MyOrange}{HTML}{FF6600}
\usepackage{fancyhdr}
\pagestyle{fancy}
\fancyhf{}
\fancyhead[LE,RO]{\thepage}
\fancyhead[RE]{\leftmark}
\fancyhead[LO]{\rightmark}
\renewcommand{\headrulewidth}{0.4pt}
\usepackage[toc]{glossaries}
\makeglossaries
\author{Laurent Garnier}
\date{\today}
\title{Manipuler les vecteurs du plan}
\hypersetup{
 pdfauthor={Laurent Garnier},
 pdftitle={Manipuler les vecteurs du plan},
 pdfkeywords={},
 pdfsubject={Ce livre propose des exercices et explications pour manipuler les vecteurs},
 pdfcreator={Emacs 29.4 (Org mode 9.6.15)}, 
 pdflang={Fr}}
\begin{document}

\maketitle
\tableofcontents


\part{Introduction}
\label{sec:org22481e3}
\chapter{À qui s'adresse ce livre}
\label{sec:orgbe71f49}

Ce livre constitue une mise en oeuvre concrète et pratique de la
rubrique intitulée "Manipuler les vecteurs du plan" pages 8 et 9 du
programme officiel de mathématiques pour la classe de seconde
accessible à cette URL :
\url{https://cache.media.education.gouv.fr/file/SP1-MEN-22-1-2019/95/7/spe631\_annexe\_1062957.pdf}

Par conséquent, il intéressera :
\begin{itemize}
\item les élèves de fin de collège souhaitant prendre de l'avance et
préparer au mieux leur entrée au lycée
\item bien entendu les élèves de seconde qui sont les premiers concernés
\item les élèves de premières, terminales, et même du supérieur,
qui pourront (re)voir des notions pas toujours bien maîtrisées alors
que cet aspect fondamental des mathématiques modernes est crucial
\item enfin, toutes les personnes, amatrices ou professionnelles,
soucieuses d'apprendre ou de ré-apprendre à des fins personnelles
et/ou d'aide de personnes tierces (de façon bénévole ou rémunérée)
\end{itemize}


À l'heure où les deux lettres IA (pour Intelligence Artificielle)
sont sur toutes les lèvres il me semble important de bien faire
comprendre (ou de rappeler pour ceux qui l'avaient oublié) que les
modèles de langage tels que ChatGPT, ClaudeAI, DeepSeek, Gemini, Grok,
Kimi, Mistral et consort ne font que manipuler des vecteurs à chaque
prompt que vous leur écrivez ou dictez.

Oui, vous m'avez bien lu, les fameuses IAs supposément magiciennes
et "aptes" à "nous remplacer" selon certains (oui je pense très fort
à Laurent Alexandre), elles ne font QUE du calcul vectoriel.

L'espace dédié à cette introduction, et même le cadre global de cet
ouvrage, ne se prête pas aux explications générales et totales du
pourquoi et du comment mais je vous livre tout de même une ressource
très pédagogique qui vous permettra de mieux comprendre tous les
détails par vous même : la chaîne 3blue1brown (meilleure chaîne de
mathématiques au monde) a consacré une série de vidéos pour
expliquer précisément comment les modèles de langage fonctionnent,
voici l'URL : \url{https://youtu.be/LPZh9BOjkQs?si=QMzyEqxItcq7ZIR7}

En conclusion de cette introduction, je dirais donc qu'il n'a jamais
été aussi important de comprendre comment fonctionne le concept de
vecteur car une très grande partie de cette technologie qui apparaît
comme de la magie repose, expose et manipule quasi-exclusivement ce
concept si fécond.

Alors, cessez d'être spectateurs, devenez acteurs !

\chapter{Organisation de l'ouvrage}
\label{sec:org639c5c8}

Comme indiqué précédemment, cet ouvrage se base sur le programme
officiel de mathématiques en classe de seconde dans le système
scolaire français auquel vous pouvez accéder via cette URL :
  \url{https://cache.media.education.gouv.fr/file/SP1-MEN-22-1-2019/95/7/spe631\_annexe\_1062957.pdf}


Ainsi, les 7 contenus sont listés dans le même ordre avec un ou
deux exercices d'illustration à chaque fois et un QCM
d'auto-évaluation.

Ensuite viennent les 7 capacités attendues également illustrées par
au moins un exercice par capacité ainsi qu'un QCM
d'auto-évaluation.

Viennent ensuite des exercices de synthèse, puis quelques
démonstrations et enfin des exercices divers.

Les exercices divers se veulent (beaucoup) plus originaux (par
exemple analyse des déplacements du fou sur un échiquier) et
concrets.

Subséquemment vient une partie sur les erreurs fréquentes à éviter.

Par suite vous obtiendrez des explications sur la façon d'obtenir
les solutions qui ont volontairement été séparées de l'ouvrage afin
de limiter la tentation de recourir à la solution avant d'avoir
cherché à résoudre les exercices.

On ne le dira jamais assez mais c'est durant la phase de recherche
que vous évaluez votre niveau réel et que vous solidifiez les
apprentissages.

Enfin une dernière partie est consacrée aux remerciements et
feedback.

Bonne lecture, mais surtout, bonne action !

\part{Contenus}
\label{sec:orgc44a009}

\begin{itemize}
\item \(C_1\) : Vecteur \(\overrightarrow{MM'}\) associé à la translation
qui transforme M en M'. Direction, sens et norme.
\item \(C_2\) : Égalité de deux vecteurs. Notation \(\vec{u}\).
Vecteur nul.
\item \(C_3\) : Somme de deux vecteurs en lien avec l'enchaînement
des translations. Relation de Chasles.
\item \(C_4\) : Base orthonormée. Coordonnées d'un vecteur. Expression
de la norme d'un vecteur.
\item \(C_5\) : Expression des coordonnées de \(\overrightarrow{AB}\) en
fonction de celles de A et de B.
\item \(C_6\) : Produit d'un vecteur par un nombre réel. Colinéarité
de deux vecteurs.
\item \(C_7\) : Déterminant de deux vecteurs dans une base orthonormée,
critère de colinéarité. Application à l'alignement, au
parallélisme.
\end{itemize}

\chapter{Contenu 1}
\label{sec:org628dc0b}

\section{Objectifs pédagogiques}
\label{sec:org2e1da7e}

Dans cette partie les exercices ont pour but de mettre en
application le contenu 1.

\begin{itemize}
\item \(C_1\) : Vecteur \(\overrightarrow{MM'}\) associé à la
translation qui transforme M en M'. Direction, sens et norme.
\end{itemize}

\begin{center}
\includegraphics[width=.9\linewidth]{./img/C1.png}
\end{center}

\section{Exercice 1}
\label{sec:org789da19}

Soit les points A, B, C tels que :
\begin{itemize}
\item \(\vec{u} = \overrightarrow{AB}\)
\item \(\vec{v} = \overrightarrow{AC}\)
\end{itemize}

Voir la figure :

\begin{center}
\includegraphics[width=.9\linewidth]{./img/ABCvect.png}
\end{center}

\begin{enumerate}
\item Quelle est l'image du point B par la translation de vecteur
\(\vec{v}\) c'est-à-dire qui transforme A en C ? On appellera D
le point image.
\item Quelle est l'image du point C par la translation de vecteur
\(\vec{u}\) c'est-à-dire qui transforme A en B ? On appellera E
le point image.
\item Que peut-on déduire ?
\item Quelle est la nature du quadrilatère ABDC ?
\end{enumerate}


\section{Exercice 2}
\label{sec:org5c37067}

Soient A et B deux points distincts dans le plan.
\begin{enumerate}
\item On considère le point C image de B par la translation de
vecteur \(\vec{u} = \overrightarrow{AB}\). Quelle est la
direction du vecteur \(\overrightarrow{BC}\) ? Quelle est son
sens ? Quelle est sa norme ?
\item Que représente le point B pour le segment [AC] ?
\item Comparer les vecteurs \(\overrightarrow{AB}\) et
\(\overrightarrow{BC}\) entre eux et avec le vecteur
\(\overrightarrow{AC}\).
\end{enumerate}

\section{Programme 1}
\label{sec:org884234b}

Écrire un programme Python qui rappelle qu'un vecteur est défini
par 3 caractéristiques fondamentales :
\begin{enumerate}
\item Direction
\item Norme
\item Sens
\end{enumerate}

\section{QCM d'auto-évaluation}
\label{sec:orgf4714d5}

\textbf{Parmi les réponses proposées au moins une est la bonne. Cela signifie qu'il peut y avoir \emph{plusieurs} bonnes réponses.}

Lorsqu'on parle du vecteur \(\overrightarrow{MM'}\) associé à
la transalation qui transforme M en M'. 

\begin{enumerate}
\item Quelle est la direction ?
\begin{enumerate}[label=\alph*.]
\item Toute droite parallèle à l'axe des abscisses.
\item Uniquement la droite (MM').
\item Toute droite parallèle à l'axe des ordonnées.
\item Toute droite parallèle à la droite (MM').
\end{enumerate}
\item Quelle est le sens ?
\begin{enumerate}[label=\alph*.]
\item De O à M.
\item De M à M'.
\item De M' à M.
\item De M à O.
\end{enumerate}
\item Quelle est la norme ?
\begin{enumerate}[label=\alph*.]
\item La distance OM.
\item La distance OM'.
\item La distance MM'.
\item La distance M'M.
\end{enumerate}
\end{enumerate}

\chapter{Contenu 2}
\label{sec:orgde8aab4}
\section{Objectifs pédagogiques}
\label{sec:orgc6e742c}

Dans cette partie les exercices ont pour but de mettre en
application le contenu 2.

\begin{itemize}
\item \(C_2\) : Égalité de deux vecteurs. Notation
\(\vec{u}\). Vecteur nul.
\end{itemize}




\section{Exercice 3}
\label{sec:orgd72803c}

On reprend la figure de l'exercice 1 : 

\begin{center}
\includegraphics[width=.9\linewidth]{./img/ABCvect.png}
\end{center}


Soit D l'image du point C par la translation de vecteur \(\vec{u}
    = \overrightarrow{AB}\).

\begin{enumerate}
\item Comparer les vecteurs \(\vec{u} = \overrightarrow{AB}\) et
\(\overrightarrow{CD}\).
\item Soit le vecteur \(\vec{w} = \overrightarrow{AB} +
       \overrightarrow{DC}\).

Que remarquez-vous ?
\item Soit E l'image de D par la translation de vecteur
\[\vec{v} = \overrightarrow{AC}\]

Calculer
\[\overrightarrow{BD} + \overrightarrow{ED}\]
\end{enumerate}

\section{Programme 2}
\label{sec:org51d5edb}

Écrire un programme Python qui :
\begin{itemize}
\item demande à l'utilisateur si les vecteurs sont égaux
\item ensuite demande s'ils sont alignés
\item indique les déductions en fonction des réponses
\end{itemize}


\textbf{Indication} : réfléchir aux différents cas possibles.


\section{QCM d'auto-évaluation}
\label{sec:org28c31ae}

\textbf{Parmi les réponses proposées au moins une est la bonne. Cela
signifie qu'il peut y avoir \emph{plusieurs} bonnes réponses.}

\begin{enumerate}
\item Deux vecteurs \(\vec{u}\) et \(\vec{v}\) sont égaux si :
\begin{enumerate}[label=\alph*.]
\item Ils ont la même direction.
\item Ils ont la même direction et le même sens.
\item Ils ont la même direction et la même norme.
\item Ils ont la même direction, le même sens et la même norme.
\end{enumerate}
\item On dit qu'un vecteur est nul si :
\begin{enumerate}[label=\alph*.]
\item Sa direction est horizontale.
\item Sa direction est verticale.
\item Il va dans un sens puis dans l'autre.
\item Sa norme vaut zéro.
\end{enumerate}
\end{enumerate}


\chapter{Contenu 3}
\label{sec:org9ae8459}

\section{Objectifs pédagogiques}
\label{sec:org0d723e7}

Dans cette partie les exercices ont pour but de mettre en
application le contenu 3.

\begin{itemize}
\item \(C_3\) : Somme de deux vecteurs en lien avec l'enchaînement des
translations. Relation de Chasles.
\end{itemize}



\section{Exercice 4}
\label{sec:orgf77d057}

On reprend la figure de l'exercice 1 : 

\begin{center}
\includegraphics[width=.9\linewidth]{./img/ABCvect.png}
\end{center}


\begin{enumerate}
\item Calculer \(\overrightarrow{AB} + \overrightarrow{BC}\).

\textbf{Indication} : utiliser la relation de Chasles.
\item Construire le point D tel que \(\overrightarrow{AD} =
       \overrightarrow{AB} + \overrightarrow{AC}\).
\item Soit O l'intersection entre les segments [AD] et [BC]. Ecrire
le vecteur \(\overrightarrow{AO}\) comme la somme de deux
vecteurs.
\item Calculer \(\overrightarrow{AO} + \overrightarrow{BO} +
       \overrightarrow{CO} + \overrightarrow{DO}\).
\end{enumerate}


\section{Programme 3}
\label{sec:orgd5769ad}

Écrire un programme Python qui rappelle la relation de Chasles. 

\section{QCM d'auto-évaluation}
\label{sec:org6646736}

\textbf{Parmi les réponses proposées au moins une est la bonne. Cela
signifie qu'il peut y avoir \emph{plusieurs} bonnes réponses.}

\begin{enumerate}
\item Ajouter deux vecteurs revient à :
\begin{enumerate}[label=\alph*.]
\item enchaîner deux translations successives
\item faire une rotation
\item faire une symétrie
\item faire une homothétie
\end{enumerate}
\item La relation de Chasles :
\begin{enumerate}[label=\alph*.]
\item augmente la norme d'un vecteur
\item décompose un vecteur en sommes de vecteurs
\item consiste à passer un coup de fil à Michel
\item revient à faire une transformation géométrique sur un
vecteur
\end{enumerate}
\end{enumerate}


\chapter{Contenu 4}
\label{sec:orgdb8ce47}

\section{Objectifs pédagogiques}
\label{sec:org3b6aa9f}

Dans cette partie les exercices ont pour but de mettre en
application le contenu 4.

\begin{itemize}
\item \(C_4\) : Base orthonormée. Coordonnées d'un vecteur. Expression
de la norme d'un vecteur.
\end{itemize}



\section{Exercice 5}
\label{sec:org905f645}

On considère le carré ABCD sur la figure :

\begin{center}
\includegraphics[width=.9\linewidth]{./img/ABCDsquare.png}
\end{center}


\begin{enumerate}
\item Vérifier que les vecteurs \(\vec{u} = \overrightarrow{AB}\) et
\(\vec{v} = \overrightarrow{AD}\) forment bien une base
orthonormée.
\item Déterminer les coordonnées des 4 points A, B, C, D dans la base
\((\vec{u} ; \vec{v})\).
\item Calculer la norme du vecteur \(\overrightarrow{AC}\).
\end{enumerate}


\section{Exercice 5 bis}
\label{sec:org88bd4bd}

On se place dans le plan muni repère orthonormé \((O ; \vec{i} ;
    \vec{j})\). Avec :
\begin{align*}
\vec{i}&\begin{pmatrix}1\\0\end{pmatrix} && \vec{j}\begin{pmatrix}0\\1\end{pmatrix}
\end{align*}

Calculer les normes des vecteurs suivants :

\begin{enumerate}
\item \(n_1 = \lvert\lvert \vec{i} + \vec{j}\rvert\rvert\)
\item \(n_2 = \lvert\lvert \vec{i} - \vec{j}\rvert\rvert\)
\item \(n_3 = \lvert\lvert 3\vec{i}\rvert\rvert\)
\item \(n_4 = \lvert\lvert 3\vec{i} + 4\vec{j}\rvert\rvert\)
\item \(n_5 = \lvert\lvert -5\vec{j}\rvert\rvert\)
\end{enumerate}

\section{Exercice 5 ter}
\label{sec:org63048c0}

On se place dans le plan muni repère orthonormé \((O ; \vec{i} ;
    \vec{j})\). Avec :
\begin{align*}
\vec{i}&\begin{pmatrix}1\\0\end{pmatrix} && \vec{j}\begin{pmatrix}0\\1\end{pmatrix}
\end{align*}

Calculer et comparer les normes ajoutées séparément
\[s = \lvert\lvert\vec{u}\rvert\rvert +
    \lvert\lvert\vec{v}\rvert\rvert\]

avec celles des vecteurs sommes

\[e = \lvert\lvert\vec{u} + \vec{v}\rvert\rvert\]

\begin{enumerate}
\item \((\vec{u}, \vec{v}) = (\vec{i}, \vec{j})\)
\item \((\vec{u}, \vec{v}) = (\vec{i} + \vec{j}, \vec{i} - \vec{j})\)
\item \((\vec{u}, \vec{v}) = (a\vec{i} + b\vec{j}, c\vec{i} + d\vec{j})\)
\end{enumerate}



\section{Programme 4}
\label{sec:org333c12c}

Écrire un programme Python qui rappelle :
\begin{enumerate}
\item la définition d'une base orthonormée.
\item la relation vectorielle entre les cordonnées d'un vecteur et
les vecteurs de la base.
\item la formule de calcul de la norme d'un vecteur et qui effectue
le calcul.
\end{enumerate}


\section{QCM d'auto-évaluation}
\label{sec:org5ac697c}

\textbf{Parmi les réponses proposées au moins une est la bonne. Cela
signifie qu'il peut y avoir \emph{plusieurs} bonnes réponses.}

\begin{enumerate}
\item Une base orthonormée du plan est :
\begin{enumerate}[label=\alph*.]
\item une station de lancement de fusée
\item un couple de vecteurs ayant des directions distinctes
\item un couple de vecteurs ayant des directions orthogonales et
la même norme
\item un couple de vecteurs ayant la même direction et la même
norme
\end{enumerate}
\item Quelles sont les coordonnées d'un vecteur \(\vec{u}\) dans une
base orthonormée \((\vec{i} , \vec{j})\) ?
\begin{enumerate}[label=\alph*.]
\item Le couple de nombres \((a ; b)\) tel que \(\vec{u} =
          a\vec{i} + b\vec{j}\).
\item Les coordonnées du point M obtenu par la translation de
vecteur \(\vec{u}\) à partir du point O.
\item La somme de celles des vecteurs \(\vec{i}\) et \(\vec{j}\).
\item Les coefficients de toute combinaison linéaire des vecteurs
de la base.
\end{enumerate}
\item Considérons le vecteur \(\vec{u} = a\vec{i} + b\vec{j}\) alors
l'expression de sa norme est :
\begin{enumerate}[label=\alph*.]
\item \(\lvert\lvert\vec{u}\rvert\rvert = a + b\)
\item \(\lvert\lvert\vec{u}\rvert\rvert = a^2 - b^2\)
\item \(\lvert\lvert\vec{u}\rvert\rvert = a^2 + b^2\)
\item \(\lvert\lvert\vec{u}\rvert\rvert = \sqrt{a^2 + b^2}\)
\end{enumerate}
\end{enumerate}

\chapter{Contenu 5}
\label{sec:org1a869d8}

\section{Objectifs pédagogiques}
\label{sec:orgc5fb796}

Dans cette partie les exercices ont pour but de mettre en
application le contenu 5.

\begin{itemize}
\item \(C_5\) : Expression des coordonnées de \(\overrightarrow{AB}\)
en fonction de celles de A et de B.
\end{itemize}



\section{Exercice 6}
\label{sec:orga6748db}

On se place dans le plan muni repère orthonormé \((O ; \vec{i} ;
    \vec{j})\). Avec :
\begin{align*}
\vec{i}&\begin{pmatrix}1\\0\end{pmatrix} && \vec{j}\begin{pmatrix}0\\1\end{pmatrix}
\end{align*}


Soient les points \(A(x_A ; y_A)\) et \(B(x_B ; y_B)\) deux points
distincts dans le plan. 

\begin{enumerate}
\item Exprimer le vecteur \(\overrightarrow{OA}\) en fonction des
vecteurs de la base.
\item Exprimer le vecteur \(\overrightarrow{OB}\) en fonction des
vecteurs de la base.
\item Exprimer le vecteur \(\overrightarrow{AB}\) en fonction des
vecteurs \(\overrightarrow{OA}\) et \(\overrightarrow{OB}\).
\item En déduire l'expression du vecteur \(\overrightarrow{AB}\) en
fonction des vecteurs de la base.
\item En déduire l'expression des coordonnées du vecteur
\(\overrightarrow{AB}\) en fonction de celles de A et de B.
\end{enumerate}


\section{Programme 5}
\label{sec:org529931f}

Écrire un programme Python qui rappelle l'expression des
coordonnées du vecteur \(\overrightarrow{AB}\) en fonction de
celles de A et B et qui les calcule. 

\section{QCM d'auto-évaluation}
\label{sec:org0700b1e}

\textbf{Parmi les réponses proposées au moins une est la bonne. Cela
signifie qu'il peut y avoir \emph{plusieurs} bonnes réponses.}

On considère le vecteur \(\overrightarrow{AB}\) dans le plan muni
du repère orthonormé \((O ; \vec{i}, \vec{j})\).

\begin{enumerate}
\item En utilisant Chasles on peut écrire :
\begin{enumerate}[label=\alph*.]
\item \(\overrightarrow{AB} = \overrightarrow{OA} +
          \overrightarrow{OB}\)
\item \(\overrightarrow{AB} = \overrightarrow{OA} - \overrightarrow{OB}\)
\item \(\overrightarrow{AB} = \overrightarrow{AO} +
          \overrightarrow{BO}\)
\item \(\overrightarrow{AB} = \overrightarrow{AO} + \overrightarrow{OB}\)
\end{enumerate}
\item En utilisant les coordonnées des points A et B on a :
\begin{enumerate}[label=\alph*.]
\item \(\overrightarrow{AB}\begin{pmatrix}x_A + x_B\\y_A +
          y_B\end{pmatrix}\)
\item \(\overrightarrow{AB}\begin{pmatrix}x_A - x_B\\y_A -
          y_B\end{pmatrix}\)
\item \(\overrightarrow{AB}\begin{pmatrix}x_B - x_A\\y_B -
          y_A\end{pmatrix}\)
\item \(\overrightarrow{AB}\begin{pmatrix}x_A \times x_B\\y_A
          \times y_B\end{pmatrix}\)
\end{enumerate}
\end{enumerate}


\chapter{Contenu 6}
\label{sec:org23634d9}

\section{Objectifs pédagogiques}
\label{sec:orgbe7623d}

Dans cette partie les exercices ont pour but de mettre en
application le contenu 6.

\begin{itemize}
\item \(C_6\) : Produit d'un vecteur par un nombre réel. Colinéarité
de deux vecteurs.
\end{itemize}



\section{Exercice 7}
\label{sec:org09619a6}

On se place dans le plan muni repère orthonormé \((O ; \vec{i} ;
    \vec{j})\). Avec :
\begin{align*}
\vec{i}&\begin{pmatrix}1\\0\end{pmatrix} && \vec{j}\begin{pmatrix}0\\1\end{pmatrix} \\
\vec{u} &= 3\vec{i} + 2\vec{j}              && \vec{v} = 2\vec{i} + 3\vec{j}
\end{align*}

Voir figure :

\begin{center}
\includegraphics[width=.9\linewidth]{./img/Oijorthonormaluv.png}
\end{center}

\begin{enumerate}
\item Construire les points A, B, C et D tels que :
\begin{align*}
\overrightarrow{OA} &= 2\vec{i} && \overrightarrow{OB} =
3\vec{i}\\
\overrightarrow{OC} &= 2\vec{j} && \overrightarrow{OD} =
3\vec{j}
\end{align*}
\item Vérifier que les vecteurs \(\overrightarrow{OA}\) et
\(\overrightarrow{OB}\) sont colinéaires.
\item Vérifier que les vecteurs \(\overrightarrow{OC}\) et
\(\overrightarrow{OD}\) sont colinéaires.
\item Construire les points E et F tels que :
\begin{align*}
\overrightarrow{OE} = \vec{u} && \overrightarrow{OF} = \vec{v}
\end{align*}
\item Vérifier que les vecteurs \(\overrightarrow{DE}\) et
\(\overrightarrow{CF}\) sont colinéaires.
\end{enumerate}


\section{Programme 6}
\label{sec:orgb3ebe61}

Écrire un programme Python indique si deux vecteurs sont
colinéaires ou pas. 

\section{QCM d'auto-évaluation}
\label{sec:org7c65e2a}

\textbf{Parmi les réponses proposées au moins une est la bonne. Cela
signifie qu'il peut y avoir \emph{plusieurs} bonnes réponses.}

\begin{enumerate}
\item Si on multiplie un vecteur par un nombre réel supérieur à 1
alors :
\begin{enumerate}[label=\alph*.]
\item Le vecteur change de direction.
\item Le vecteur augmente sa norme.
\item Le vecteur change de sens.
\item Le vecteur reste identique.
\end{enumerate}
\item Si on multiplie un vecteur par un nombre réel inférieur à -1
alors :
\begin{enumerate}[label=\alph*.]
\item Le vecteur change de direction.
\item Le vecteur augmente sa norme.
\item Le vecteur change de sens.
\item Le vecteur reste identique.
\end{enumerate}
\item Si on multiplie un vecteur par un nombre réel supérieur à -1 et
inférieur à 1 alors :
\begin{enumerate}[label=\alph*.]
\item Le vecteur change de direction.
\item Le vecteur diminue sa norme.
\item Le vecteur change de sens.
\item Le vecteur reste identique.
\end{enumerate}
\item Si on multiplie un vecteur par un nombre réel alors :
\begin{enumerate}[label=\alph*.]
\item Le vecteur obtenu n'est pas colinéaire au vecteur initial.
\item Le vecteur obtenu est colinéaire au vecteur initial.
\end{enumerate}
\end{enumerate}


\chapter{Contenu 7}
\label{sec:org0736196}

\section{Objectifs pédagogiques}
\label{sec:orgcc659ad}

Dans cette partie les exercices ont pour but de mettre en
application le contenu 7.

\begin{itemize}
\item \(C_7\) : Déterminant de deux vecteurs dans une base
orthonormée, critère de colinéarité. Application à l'alignement,
au parallélisme.
\end{itemize}


\section{Exercice 8}
\label{sec:org6798f7e}

On reprend la configuration finale de l'exercice 7.

Voir figure :

\begin{center}
\includegraphics[width=.9\linewidth]{./img/figexo8.png}
\end{center}

\begin{enumerate}
\item Calculer les déterminants suivants :
\begin{align*}
d_1 = det(\vec{i}, \vec{j}) && d_2 = det(\vec{u}, \vec{v})\\
d_3 = det(\overrightarrow{OA}, \overrightarrow{OB}) && d_4 = det(\overrightarrow{OC}, \overrightarrow{OD}) \\
d_5 = det(\overrightarrow{OA}, \overrightarrow{OC}) && d_6 = det(\overrightarrow{OB}, \overrightarrow{OD})
\end{align*}
\item En utilisant le déterminant montrer que les vecteurs
\(\overrightarrow{DF}\) et \(\overrightarrow{CE}\) sont
colinéaires. En déduire la nature du quadrilatère DCEF.
\item Même question pour \(\overrightarrow{AF}\) et
\(\overrightarrow{BE}\) et le quadrilatère ABEF.
\item Calculer les normes des vecteurs \(\overrightarrow{IF}\) et
\(\overrightarrow{JE}\).
\item Comparer les vecteurs \(\overrightarrow{IJ}\) et
\(\overrightarrow{EF}\). En déduire la nature du quadrilatère
IEFJ.
\item On nommera G l'intersection des segments [IF] et
[JE]. Déterminer ses coordonnées. En déduire celle du point H
tel que \(\overrightarrow{OG} = \overrightarrow{GH}\). En
déduire la nature du quadrilatère OBHD.
\item Quelle est la nature du triangle OFE ?
\item Quelle est l'image du point G par la translation de vecteur
\(\overrightarrow{BG}\) ?
\item Déterminer les coordonnées du point K tel que
\(\overrightarrow{BK} = \vec{j}\).
\item Déterminer les coordonnées du point L tel que
\(\overrightarrow{DL} = \vec{i}\).
\end{enumerate}


\section{Programme 7}
\label{sec:org006f2cd}

Écrire un programme Python qui indique si 3 points sont alignés ou pas.

\section{QCM d'auto-évaluation}
\label{sec:org68697e2}

\textbf{Parmi les réponses proposées au moins une est la bonne. Cela
signifie qu'il peut y avoir \emph{plusieurs} bonnes réponses.}

\begin{enumerate}
\item Considérons les vecteurs
\[\vec{u}_1\begin{pmatrix}x_1\\y_1\end{pmatrix}\] et
\[\vec{u}_2\begin{pmatrix}x_2\\y_2\end{pmatrix}\] alors :
\begin{enumerate}[label=\alph*.]
\item \[det(\vec{u}_1 , \vec{u}_2) = \begin{vmatrix}x_1&x_2\\y_1&y_2\end{vmatrix}
          = x_1x_2 + y_1y_2\]
\item \[det(\vec{u}_1 , \vec{u}_2) = \begin{vmatrix}x_1&x_2\\y_1&y_2\end{vmatrix}
          = x_1x_2 - y_1y_2\]
\item \[det(\vec{u}_1 , \vec{u}_2) = \begin{vmatrix}x_1&x_2\\y_1&y_2\end{vmatrix}
          = x_1y_2 + y_1x_2\]
\item \[det(\vec{u}_1 , \vec{u}_2) = \begin{vmatrix}x_1&x_2\\y_1&y_2\end{vmatrix}
          = x_1y_2 - y_1x_2\]
\end{enumerate}
\item Considérons les mêmes vecteurs que précédemment.

On dira que \(\vec{u}_1\) et \(\vec{u}_2\) sont colinéaires
si :

\begin{enumerate}[label=\alph*.]
\item \(det(\vec{u}_1 , \vec{u}_2) = 1\)
\item \(det(\vec{u}_1 , \vec{u}_2) = 0\)
\item il existe un réel \(k\) tel que \(x_1 = kx_2\) et \(y_1 =
          ky_2\)
\item \(\dfrac{x_1}{x_2} = \dfrac{y_1}{y_2}\)
\end{enumerate}
\end{enumerate}

\part{Capacités attendues}
\label{sec:org4c8aade}

\begin{itemize}
\item \(Ca_{1}\) : Représenter géométriquement des vecteurs.
\item \(Ca_{2}\) : Construire géométriquement la somme de deux vecteurs.
\item \(Ca_{3}\) : Représenter un vecteur dont on connaît les
coordonnées. Lire les coordonnées d'un vecteur.
\item \(Ca_{4}\) : Calculer les coordonnées d'une somme de vecteurs, d'un
produit d'un vecteur par un nombre réel.
\item \(Ca_{5}\) : Calculer la distance entre deux points. Calculer les
coordonnées du milieu d'un segment.
\item \(Ca_{6}\) : Caractériser alignement et parallélisme par la
colinéarité de vecteurs.
\item \(Ca_{7}\) : Résoudre des problèmes en utilisant la représentation la
plus adaptée des vecteurs.
\end{itemize}


\chapter{Capacité attendue 1}
\label{sec:orgbd073aa}
\section{Objectifs pédagogiques}
\label{sec:org75069d1}

Dans cette partie le but est de mettre en application la capacité
attendue 1.

\begin{itemize}
\item \(Ca_{1}\) : Représenter géométriquement des vecteurs.
\end{itemize}

\section{Exercice 9}
\label{sec:orgae03f16}

On considère le triangle ABC représenté sur la figure avec le
quadrillage :

\begin{center}
\includegraphics[width=.9\linewidth]{./img/ABCtrig.png}
\end{center}

\begin{enumerate}
\item Construire le point D tel que \(\overrightarrow{BD} =
       \overrightarrow{AB}\).
\item Construire le point E tel que \(\overrightarrow{CE} =
       \overrightarrow{AB}\).
\item Comparer les vecteurs \(\overrightarrow{BE}\) et
\(\overrightarrow{AC}\).
\end{enumerate}



\section{QCM d'auto-évaluation}
\label{sec:org5329a52}

\textbf{Parmi les réponses proposées au moins une est la bonne. Cela
    signifie qu'il peut y avoir \emph{plusieurs} bonnes réponses.}

\begin{enumerate}
\item Un vecteur est caractérisé par :
\begin{enumerate}[label=\alph*.]
\item Sa longueur uniquement
\item Sa direction et son sens uniquement
\item Sa direction, son sens et sa norme
\item Son origine et son extrémité
\end{enumerate}

\item Les vecteurs \(\overrightarrow{AB}\) et \(\overrightarrow{CD}\) 
sont égaux si :
\begin{enumerate}[label=\alph*.]
\item A = C et B = D
\item ABDC est un parallélogramme
\item AB = CD (distances égales)
\item Les segments [AB] et [CD] sont parallèles
\end{enumerate}
\end{enumerate}

\chapter{Capacité attendue 2}
\label{sec:org5384ac3}
\section{Objectifs pédagogiques}
\label{sec:org592adff}

Dans cette partie le but est de mettre en application la capacité
attendue 2.

\begin{itemize}
\item \(Ca_{2}\) : Construire géométriquement la somme de deux vecteurs.
\end{itemize}


\section{Exercice 10}
\label{sec:org0b4c56a}

On considère la configuration initiale de l'exercice 9 voir
figure :

\begin{center}
\includegraphics[width=.9\linewidth]{./img/ABCtrig.png}
\end{center}

\begin{enumerate}
\item Construire le point D tel que \(\overrightarrow{BD} =
       \overrightarrow{BC} + \overrightarrow{BA}\).
\item En déduire la nature du quadrilatère ABCD.
\item Construire le point E tel que \(\overrightarrow{AE} =
       \overrightarrow{AB} + \overrightarrow{AC}\).
\item En déduire la nature du quadrilatère ABEC.
\end{enumerate}


\section{QCM d'auto-évaluation}
\label{sec:orgacfbcf1}

\textbf{Parmi les réponses proposées au moins une est la bonne. Cela
signifie qu'il peut y avoir \emph{plusieurs} bonnes réponses.}

Considérons deux vecteurs \(\vec{u}\) et \(\vec{v}\).

\begin{enumerate}
\item Pour construire géométriquement la somme il faut :
\begin{enumerate}[label=\alph*.]
\item partir de l'origine du vecteur \(\vec{u}\) puis, arrivé à
son extrémité appliquer le vecteur \(\vec{v}\)
\item partir de l'origine du vecteur \(\vec{v}\) puis, arrivé à
son extrémité appliquer le vecteur \(\vec{u}\)
\item les deux propositions précédentes aboutissent au même
résultat
\end{enumerate}
\item Si \(\vec{u}\) et \(\vec{v}\) ne sont pas colinéaires alors
\(\vec{w} = \vec{u} + \vec{v}\) :
\begin{enumerate}[label=\alph*.]
\item est un vecteur ayant même direction que \(\vec{u}\) et
\(\vec{v}\)
\item représente la diagonale du parallélogramme obtenu en
faisant partir \(\vec{u}\) et \(\vec{v}\) de la même origine
\end{enumerate}
\end{enumerate}

\chapter{Capacité attendue 3}
\label{sec:org5484cda}
\section{Objectifs pédagogiques}
\label{sec:org97cef35}

Dans cette partie le but est de mettre en application la capacité
attendue 3.

\begin{itemize}
\item \(Ca_{3}\) : Représenter un vecteur dont on connaît les
coordonnées. Lire les coordonnées d'un vecteur.
\end{itemize}


\section{Exercice 11}
\label{sec:orgbfa83c0}

On se place dans le plan muni repère orthonormé \((O ; \vec{i} ;
    \vec{j})\). Avec :
\begin{align*}
\vec{i}&\begin{pmatrix}1\\0\end{pmatrix} && \vec{j}\begin{pmatrix}0\\1\end{pmatrix}
\end{align*}

\begin{enumerate}
\item Représenter le vecteur
\(\vec{u}\begin{pmatrix}3\\ 5\end{pmatrix}\).
\item Lire les coordonnées du vecteur \(\vec{v}\) sur la figure :
\begin{center}
\includegraphics[width=.9\linewidth]{./img/figexo11.png}
\end{center}
\end{enumerate}


\section{QCM d'auto-évaluation}
\label{sec:orgb5fa55a}

\textbf{Parmi les réponses proposées au moins une est la bonne. Cela
signifie qu'il peut y avoir \emph{plusieurs} bonnes réponses.}

Considérons le vecteur
\(\vec{u}\begin{pmatrix}a\\b\end{pmatrix}\). Alors :

\begin{enumerate}
\item Pour le représenter dans le repère \((O ; \vec{i} , \vec{j})\) :
\begin{enumerate}[label=\alph*.]
\item on se place au point M de coordonnées (a ; b) et on trace le
vecteur en se déplaçant de a unités sur l'axe horizontal et
b unités sur l'axe vertical.
\item partant de l'origine du repère on se déplace de a unités sur
l'axe horizontal et b unités sur l'axe vertical.
\end{enumerate}
\item Pour lire les cordonnées d'un vecteur \(\vec{v}\) :
\begin{enumerate}[label=\alph*.]
\item on se place à son origine et on reporte les coordonnées du
point
\item on trace un représentant du vecteur en partant de l'origine
du repère et on lit les coordonnées de son extrémité
\end{enumerate}
\end{enumerate}

\chapter{Capacité attendue 4}
\label{sec:org6ed8f8b}
\section{Objectifs pédagogiques}
\label{sec:orgb395a70}

Dans cette partie le but est de mettre en application la capacité
attendue 4.

\begin{itemize}
\item \(Ca_{4}\) : Calculer les coordonnées d'une somme de vecteurs,
d'un produit d'un vecteur par un nombre réel.
\end{itemize}



\section{Exercice 12}
\label{sec:org65da53f}


On se place dans le plan muni repère orthonormé \((O ; \vec{i} ;
    \vec{j})\). Avec :
\begin{align*}
\vec{i}&\begin{pmatrix}1\\0\end{pmatrix} && \vec{j}\begin{pmatrix}0\\1\end{pmatrix}
\end{align*}

Calculer les coordonnées des vecteurs :
\begin{align*}
\vec{s} = \vec{i} + \vec{j} && \vec{d} = \vec{i} - \vec{j}\\
\vec{p} = 3\vec{d} && \vec{c} = 2\vec{s} - 5\vec{d}
\end{align*}

\section{QCM d'auto-évaluation}
\label{sec:org527707e}

\textbf{Parmi les réponses proposées au moins une est la bonne. Cela
signifie qu'il peut y avoir \emph{plusieurs} bonnes réponses.}

\begin{enumerate}
\item Soient \[\vec{u}_1\begin{pmatrix}x_1\\y_1\end{pmatrix}\] et
\[\vec{u}_2\begin{pmatrix}x_2\\y_2\end{pmatrix}\] deux vecteurs
alors \[\vec{u}_3 = \vec{u}_1 + \vec{u}_2\] a pour
coordonnées : 
\begin{enumerate}[label=\alph*.]
\item \(\vec{u}_3\begin{pmatrix}x_1x_2\\y_1y_2\end{pmatrix}\)
\item \(\vec{u}_3\begin{pmatrix}x_1 - x_2\\y_1 - y_2\end{pmatrix}\)
\item \(\vec{u}_3\begin{pmatrix}x_1 + x_2\\y_1 +
          y_2\end{pmatrix}\)
\item \(\vec{u}_3\begin{pmatrix}\dfrac{x_1}{x_2}\\\dfrac{y_1}{y_2}\end{pmatrix}\)
\end{enumerate}
\item Soient un vecteur \(\vec{u}\begin{pmatrix}x\\y\end{pmatrix}\)
et un réel k alors \(\vec{v} = ku\) vérifie :
\begin{enumerate}[label=\alph*.]
\item \(\vec{v}\begin{pmatrix}kx\\y\end{pmatrix}\)
\item \(\vec{v}\begin{pmatrix}x\\ky\end{pmatrix}\)
\item \(\vec{v}\begin{pmatrix}kx\\ky\end{pmatrix}\)
\item \(\vec{v}\begin{pmatrix}\frac{1}{k}x\\ \frac{1}{k}y\end{pmatrix}\)
\end{enumerate}
\end{enumerate}

\chapter{Capacité attendue 5}
\label{sec:org7380515}
\section{Objectifs pédagogiques}
\label{sec:org82fb6f9}

Dans cette partie le but est de mettre en application la capacité
attendue 5.

\begin{itemize}
\item \(Ca_{5}\) : Calculer la distance entre deux points. Calculer
les coordonnées du milieu d'un segment.
\end{itemize}

\section{Exercice 13}
\label{sec:orgeacd002}


On se place dans le plan muni repère orthonormé \((O ; \vec{i} ;
    \vec{j})\). Avec :
\begin{align*}
\vec{i}&\begin{pmatrix}1\\0\end{pmatrix} && \vec{j}\begin{pmatrix}0\\1\end{pmatrix}
\end{align*}

On considère les points A(2 ; 3), B(-3 ; 2), C(-2 ; -3) et D(3 ;
-2) tels que sur la figure :

\begin{center}
\includegraphics[width=.9\linewidth]{./img/figexo13.png}
\end{center}

\begin{enumerate}
\item Calculer les distances AB, AC, AD, BC, BD, CD.
\item Calculer les coordonnées des milieux des segments [AB], [AC],
[AD], [BC], [BD], [CD].
\end{enumerate}


\section{QCM d'auto-évaluation}
\label{sec:org817bc9f}

\textbf{Parmi les réponses proposées au moins une est la bonne. Cela
signifie qu'il peut y avoir \emph{plusieurs} bonnes réponses.}

\begin{enumerate}
\item La distance entre \(A(x_A ; y_A)\) et \(B(x_B ; y_B)\) vaut :
\begin{enumerate}[label=\alph*.]
\item \(AB = x_Ax_B + y_Ay_B\)
\item \(AB = (x_A + x_B)^2 + (y_A + y_B)^2\)
\item \(AB = (x_A - x_B)^2 + (y_A - y_B)^2\)
\item \(AB = \sqrt{(x_A - x_B)^2 + (y_A - y_B)^2}\)
\end{enumerate}
\item Le milieu \(M(x_M ; y_M)\) du segment [AB] vérifie :
\begin{enumerate}[label=\alph*.]
\item \((x_M ; y_M) = \left(\dfrac{x_A + x_B}{2} ; \dfrac{y_A +
          y_B}{2}\right)\)
\item \((x_M ; y_M) = \left(\dfrac{x_A - x_B}{2} ; \dfrac{y_A -
          y_B}{2}\right)\)
\item \((x_M ; y_M) = \left(\dfrac{x_A \times x_B}{2} ; \dfrac{y_A \times
          y_B}{2}\right)\)
\item \((x_M ; y_M) = \left(\dfrac{x_A \div x_B}{2} ; \dfrac{y_A \div
          y_B}{2}\right)\)
\end{enumerate}
\end{enumerate}

\chapter{Capacité attendue 6}
\label{sec:org8b41311}
\section{Objectifs pédagogiques}
\label{sec:orgae974db}

Dans cette partie le but est de mettre en application la capacité
attendue 6.

\begin{itemize}
\item \(Ca_{6}\) : Caractériser alignement et parallélisme par la
colinéarité de vecteurs.
\end{itemize}


\section{Exercice 14}
\label{sec:orge48e486}

On reprend la configuration de l'exercice précédent.

Voir la figure :

\begin{center}
\includegraphics[width=.9\linewidth]{./img/figexo13.png}
\end{center}

\begin{enumerate}
\item Les points A, O et C sont-ils alignés ?
\item Calculer OA et OC. Qu'en déduisez-vous ?
\item Calculer le déterminant \(det(\overrightarrow{OD} ,
       \overrightarrow{OB})\). Quelle interprétation géométrique en
déduisez-vous ?
\item Comparer OB et OD. Qu'en déduisez-vous ?
\item Les vecteurs \(\overrightarrow{AB}\) et \(\overrightarrow{CD}\)
sont-ils colinéaires ?
\item Comparer AB et AD.
\item Quelle est la nature du quadrilatère ABCD ?
\end{enumerate}


\section{QCM d'auto-évaluation}
\label{sec:org56c5020}

\textbf{Parmi les réponses proposées au moins une est la bonne. Cela
signifie qu'il peut y avoir \emph{plusieurs} bonnes réponses.}

\begin{enumerate}
\item Pour montrer que A, B et C sont alignés il faut :
\begin{enumerate}[label=\alph*.]
\item que \(\overrightarrow{AB} = \overrightarrow{AC}\)
\item qu'il existe un réel k tel que \(\overrightarrow{AB} =
          k\overrightarrow{AC}\)
\item vérifier que \(\overrightarrow{AB} + \overrightarrow{BC} =
          \overrightarrow{AC}\)
\item vérifier que \(det(\overrightarrow{AB} ,
          \overrightarrow{AC}) = 0\)
\end{enumerate}
\item Pour montrer que les droites (AB) et (CD) sont parallèles il faut :
\begin{enumerate}[label=\alph*.]
\item montrer que les vecteurs \(\overrightarrow{AB}\) et
\(\overrightarrow{CD}\) sont colinéaires
\item vérifier que \(det(\overrightarrow{AB} ,
          \overrightarrow{CD}) = 0\)
\item montrer que \(\overrightarrow{AB} = \overrightarrow{CD}\)
\item vérifier que \(det(\overrightarrow{AB} ,
          \overrightarrow{CD}) \neq 0\)
\end{enumerate}
\end{enumerate}


\chapter{Capacité attendue 7}
\label{sec:org590e81b}
\section{Objectifs pédagogiques}
\label{sec:orgaa305d8}

Dans cette partie le but est de mettre en application la capacité
attendue 7.

\begin{itemize}
\item \(Ca_{7}\) : Résoudre des problèmes en utilisant la
représentation la plus adaptée des vecteurs.
\end{itemize}



\section{Exercice 15}
\label{sec:orgd86dde7}

Pour chacune des situations suivantes, indiquez comment la
résoudre selon la représentation vectorielle parmi :

\begin{itemize}
\item Analytique (coordonnées, calculs algébriques)
\item Colinéarité (proportionnalité, déterminant)
\item Géométrique (relation de Chasles, parallélogramme)
\end{itemize}


\begin{enumerate}
\item Démontrer que les points A, B, C sont alignés
\item Calculer la distance entre deux points A(2;3) et B(5;7)
\item Montrer que ABCD est un parallélogramme
\item Trouver les coordonnées du point M tel que :
\[\overrightarrow{AM} = 2\overrightarrow{AB} +
       3\overrightarrow{AC}\]
\item Vérifier si deux droites (AB) et (CD) sont parallèles
\end{enumerate}


\section{Programme}
\label{sec:org13bfde3}

Écrire un programme Python qui résout le problème
\[\overrightarrow{AM} = a\overrightarrow{AB} +
    b\overrightarrow{AC}\]
C'est-à-dire un programme qui permet d'exprimer les coordonnées du
point M en fonction des paramètres a et b et des coordonnées des
points déjà existants A, B, et C.

\section{QCM d'auto-évaluation}
\label{sec:org40b972b}

\textbf{Parmi les réponses proposées au moins une est la bonne. Cela
signifie qu'il peut y avoir \emph{plusieurs} bonnes réponses.}

\begin{enumerate}
\item Si ABC est un triangle et que D est un 4ème point qui vérifie
l'égalité \[\overrightarrow{AD} = \overrightarrow{AB} +
       \overrightarrow{AC}\]
alors on peut en déduire que :
\begin{enumerate}[label=\alph*.]
\item \(\overrightarrow{AB} + \overrightarrow{BC} = \overrightarrow{AC}\)
\item ABCD est un parallélogramme
\item ABDC est un parallélogramme
\item \(\overrightarrow{AB} = \overrightarrow{CD}\) et
\(\overrightarrow{AC} = \overrightarrow{BD}\) les deux
égalités sont vraies
\item \(\overrightarrow{AB} = \overrightarrow{CD}\) ou
\(\overrightarrow{AC} = \overrightarrow{BD}\) une seule des
deux égalités est vraie
\item \(\overrightarrow{AB} \neq \overrightarrow{CD}\) et
\(\overrightarrow{AC} \neq \overrightarrow{BD}\) aucune
des égalités n'est vraie
\item le point D est à l'intérieur du triangle ABC
\item le point D est l'image du point A par la symétrie de
centre le milieu du segment [BC]
\item le point D est à l'extérieur du triangle ABC
\item le point D est l'image du point I par la translation de
vecteur \(\overrightarrow{AI}\) où I est le milieu du
segment [BC]
\end{enumerate}
\item Si ABCD est un carré alors : 
\begin{enumerate}[label=\alph*.]
\item Les vecteurs \(\overrightarrow{AB}\) et
\(\overrightarrow{AC}\) forment une base orthornomée
\item Les vecteurs \(\overrightarrow{AB}\) et
\(\overrightarrow{AD}\) forment une base orthornomée
\item \(det(\overrightarrow{AB} , \overrightarrow{AC}) = 1\)
\item \(det(\overrightarrow{AB} , \overrightarrow{AD}) = 0\)
\item \(det(\overrightarrow{AB} , \overrightarrow{AD}) = 1\)
\item \(det(\overrightarrow{AB} , \overrightarrow{AC}) = 0\)
\item \(AC^2 = AB^2 + BC^2\)
\item \(\overrightarrow{AB} + \overrightarrow{CD} =
          2\overrightarrow{AB}\)
\item \(\overrightarrow{AB} + \overrightarrow{CD} = \vec{0}\)
\item Le centre O du carré vérifie
\[\overrightarrow{OA} + \overrightarrow{OB} +
           \overrightarrow{OC} + \overrightarrow{OD} = \vec{0}\]
\end{enumerate}
\item Si ABCD est un rectangle et O l'intersection des droites (AC)
et (BD) alors : 
\begin{enumerate}[label=\alph*.]
\item \(\overrightarrow{OA} + \overrightarrow{OB} +
          \overrightarrow{OC} + \overrightarrow{OD} = \vec{0}\)
\item \(\overrightarrow{AO} = \overrightarrow{OC} = \dfrac{1}{2}\overrightarrow{AC}\)
\item \(det(\overrightarrow{AB} , \overrightarrow{AC}) = 1\)
\item \(det(\overrightarrow{AB} , \overrightarrow{AD}) = 0\)
\item \(det(\overrightarrow{AB} , \overrightarrow{AD}) = 1\)
\item \(det(\overrightarrow{AB} , \overrightarrow{AC}) = 0\)
\item \(AC > AB + BC\)
\item \(AC < AB + AD\)
\item \(AC = BD\)
\item \(AC\neq BD\)
\end{enumerate}
\item Soient A et B deux points distincts du plan. Si C est l'image
de B par la translation de vecteur \(\overrightarrow{AB}\)
alors : 	 
\begin{enumerate}[label=\alph*.]
\item B est le milieu du segment [AC]
\item \(\overrightarrow{AC} = 2\overrightarrow{AB}\)
\item \(\overrightarrow{OC} = \overrightarrow{OA} +
          2\overrightarrow{OB}\)
\item les coordonnées de C vérifient :
\begin{align*}
x_C &= 2x_B - x_A\\
y_C &= 2y_B - y_A
\end{align*}
\item les coordonnées de C vérifient :
\begin{align*}
x_C &= 2x_B + x_A\\
y_C &= 2y_B + y_A
\end{align*}
\item C est l'image de A par la symétrie de centre B.
\item C est le milieu du segment [AB].
\item \(\overrightarrow{AB} = \overrightarrow{BC}\)
\item \(\overrightarrow{AB} + \overrightarrow{BC} =
          2\overrightarrow{AB}\)
\item \(det(\overrightarrow{AB} , \overrightarrow{AC}) = 0\)
\end{enumerate}
\item Si on a \(det(\overrightarrow{AB}, \overrightarrow{AD}) \neq 0\)
et \(det(\overrightarrow{AB}, \overrightarrow{DC}) = 0\) alors :
\begin{enumerate}[label=\alph*.]
\item ABCD ou ABDC est un trapèze.
\item Si AB = DC alors ABCD ou ABDC est un parallélogramme.
\item Si AB = AD = DC alors ABCD est un losange.
\item Si AB = AC = CD alors ABDC est un losange.
\item Si AB = DC et CA = BD alors ABDC est un rectangle.
\item Si AB = DC et AD = BC alors ABDC est un rectangle.
\item Si AB = DC et AC = BD alors ABCD est un rectangle.
\item Si AB = DC et AD = BC alors ABCD est un rectangle.
\item Si AB = DC et AC = BD alors ABCD est un losange.
\item Si AB = DC et AD = BC alors ABCD est un losange.
\end{enumerate}
\item Soient \(\vec{u}\), \(\vec{v}\) et \(\vec{w}\) trois vecteurs
tels que \(\vec{w} = \vec{u} + \vec{v}\).
\begin{enumerate}[label=\alph*.]
\item Si \(\lvert\lvert\vec{u}\rvert\rvert +
          \lvert\lvert\vec{v}\rvert\rvert =
          \lvert\lvert\vec{w}\rvert\rvert\) alors les trois vecteurs
sont colinéaires.
\item Il est impossible que les trois vecteurs soient colinéaires.
\item Si \(det(\vec{u}, \vec{v}) = 0\) alors les trois vecteurs
sont colinéaires.
\item Si \(det(\vec{u}, \vec{v}) = 0\) alors \(det(\vec{v},
          \vec{w}) = 0\).
\item Si \(det(\vec{w}, \vec{u}) = 0\) alors \(det(\vec{u},
          \vec{v}) = 0\).
\item \(\lvert\lvert\vec{u}\rvert\rvert +
          \lvert\lvert\vec{v}\rvert\rvert >
          \lvert\lvert\vec{w}\rvert\rvert\)
\item \(\lvert\lvert\vec{u}\rvert\rvert +
          \lvert\lvert\vec{v}\rvert\rvert <
          \lvert\lvert\vec{w}\rvert\rvert\)
\item Si \(det(\vec{u}, \vec{v}) = 0\) alors soit \(\vec{u} =
          \vec{0}\) soit \(\vec{v} = \vec{0}\)
\item Si \(det(\vec{u}, \vec{v}) = 0\) alors soit \(\vec{w} =
          \vec{u}\) soit \(\vec{w} = \vec{v}\)
\item Si \(det(\vec{w}, \vec{u}) = 0\) alors soit \(\vec{u} =
          -\vec{v}\) soit \(\vec{w} = \vec{0}\)
\end{enumerate}
\item Soit un vecteur \(\vec{u}\) de norme
\(\lvert\lvert\vec{u}\rvert\rvert = 5\).
\begin{enumerate}[label=\alph*.]
\item Si \(x_{\vec{u}} = 3\) alors \(y_{\vec{u}} = 4\).
\item Si \(x_{\vec{u}} = 3\) alors \(y_{\vec{u}} = -4\).
\item Si \(x_{\vec{u}} = -3\) alors \(y_{\vec{u}} = -4\).
\item Si \(x_{\vec{u}} = -3\) alors \(y_{\vec{u}} = 4\).
\item Si \(x_{\vec{u}} = \pm 3\) alors \(y_{\vec{u}} = \pm 4\).
\item Si \(x_{\vec{u}} = -4\) alors \(y_{\vec{u}} = 3\).
\item Si \(x_{\vec{u}} = -4\) alors \(y_{\vec{u}} = -3\).
\item Si \(x_{\vec{u}} = 4\) alors \(y_{\vec{u}} = -3\).
\item Si \(x_{\vec{u}} = 4\) alors \(y_{\vec{u}} = 3\).
\item Si \(x_{\vec{u}} = \pm 4\) alors \(y_{\vec{u}} = \pm 3\).
\end{enumerate}
\item Dans un repère orthonormée \((O ; \vec{i} , \vec{j})\) on
considère les points A(3 ; 2), B(3 ; -2), C(-3 ; -2), D(3 ;
-1), E(-3 ; -1), F(-1 ; -1), G(1 ; 1), H(1 ; 2), I(-1 ; 2).       
\begin{enumerate}[label=\alph*.]
\item Les vecteurs \(\overrightarrow{AB}\) et
\(\overrightarrow{AC}\) sont colinéaires car
\[det(\overrightarrow{AB} , \overrightarrow{AC}) = 0\]
\item Les vecteurs \(\overrightarrow{AH}\) et
\(\overrightarrow{AI}\) sont colinéaires car
\[det(\overrightarrow{AH} , \overrightarrow{AI}) = 0\]
\item Les points A, B et C sont alignés.
\item Les points D, E et F sont alignés.
\item ABC est un triangle rectangle en B.
\item ABD est un triangle rectangle en B.
\item GDF est un triangle rectangle en G.
\item GDF est un triangle isocèle en G.
\item BCHA est un trapèze.
\item Les vecteurs \(\overrightarrow{BC}\) et
\(\overrightarrow{AH}\) sont colinéaires.
\end{enumerate}
\item Dans un repère orthonormée \((O ; \vec{i} , \vec{j})\) on
considère les points A(2 ; 3), B(-3 ; 2), C(-2 ; -3); D(3 ;
-2).
\begin{enumerate}[label=\alph*.]
\item Le triangle BOA est isocèle en A.
\item Le triangle BOA est isocèle en B.
\item Le triangle BOA est isocèle en O.
\item Le triangle BOA est rectangle en A.
\item Le triangle BOA est rectangle en B.
\item Le triangle BOA est rectangle en O.
\item C est l'image de O par la translation de vecteur
\(\overrightarrow{AO}\).
\item D est l'image de O par la translation de vecteur
\(\overrightarrow{BO}\).
\item ABDC est un carré.
\item ABCD est un carré.
\end{enumerate}
\end{enumerate}

\part{Exercices complémentaires}
\label{sec:org34471a4}
\chapter{Pour s'exercer davantage}
\label{sec:org71b137c}
\section{Exercice 16}
\label{sec:org4c2b33a}

On se place dans le plan muni du repère orthonormé \((O ; \vec{i},
    \vec{j})\). 

\begin{enumerate}
\item Placer les points O(0 ; 0), I(1 ; 0), J(0 ; 1) et A tel que
\[\overrightarrow{OA} = \vec{i} + \vec{j}\]
Quelle est la nature du triangle OIA ?
\item Placer le point B tel que
\[\overrightarrow{AB} = \vec{j} - \vec{i}\]
Trouver une égalité vectorielle plus simple pour placer le
point B. Comparer OA, AB et OB. En déduire la nature du
triangle OAB.
\item Placer le point C tel que
\[\overrightarrow{OC} = 2\overrightarrow{AB}\]
Trouver une égalité vectorielle plus simple pour placer le
point C. Comparer OB, BC et OC. En déduire la nature du
triangle OBC.
\item Placer le point D tel que
\[\overrightarrow{OD} = -4\vec{i}\]
Quelle est la nature du triangle COD ?
\item Placer le point E tel que
\[\overrightarrow{DE} = -4\vec{j}\]
Quelle est la nature du triangle ODE ? Les vecteurs
\(\overrightarrow{OA}\) et \(\overrightarrow{OE}\) sont-ils
colinéaires ?
\item Placer le point F tel que
\[\overrightarrow{OF} = -4\overrightarrow{OB}\]
Quelle est la nature du triangle EOF ?
\item Placer le point G tel que
\[\overrightarrow{FG} = 8\vec{i}\]
Quelle est la nature du triangle FOG ?
\item Calculer les coordonnées des points H, K, L, M, N, P milieux
respectifs des segments [OA], [OC], [OD], [OE], [OF], [OG].

\textbf{Rappel} : le point J est le milieu du segment [OB].
\item Tracer les segments [HJ], [JK], [KL], [LM], [MN] et [NP].
\item Tracer les cercles de centres H, J, K, L, M, N, P de rayons
respectifs HJ, JK, KL, LM, MN, NP.
Que remarquez-vous ?
\end{enumerate}


\section{Exercice 17}
\label{sec:orgc59b200}

On se place dans le plan muni du repère orthonormé \((O ; \vec{i},
    \vec{j})\). Pour la suite de l'exercice on considère les points
\(I(1 ; 0)\) et \(J(0 ; 1)\) ainsi que le vecteur
\[\vec{u} = 2\vec{j} - 3\vec{i}\]

\begin{enumerate}
\item Quel est l'ensemble des points \(M(x ; y)\) du plan tels que
\(det(\vec{i} , \overrightarrow{OM}) = 0\) ?
\item Quel est l'ensemble des points \(M(x ; y)\) du plan tels que
\(det(\vec{j} , \overrightarrow{OM}) = 0\) ?
\item Quel est l'ensemble des points \(M(x ; y)\) du plan tels que       
\(det(\overrightarrow{OM} , \vec{u}) = 0\) ?
\item Quel est l'ensemble des points \(M(x ; y)\)
du plan tels que       
\(det(\vec{i} , \overrightarrow{IM}) = 0\) ?
\item Quel est l'ensemble des points \(M(x ; y)\)
du plan tels que       
\(det(\vec{j} , \overrightarrow{JM}) = 0\) ?
\item Quel est l'ensemble des points \(M(x ; y)\)
du plan tels que       
\(det(\vec{i} , \overrightarrow{JM}) = 0\) ?
\item Quel est l'ensemble des points \(M(x ; y)\)
du plan tels que       
\(det(\vec{j} , \overrightarrow{IM}) = 0\) ?
\item Quel est l'ensemble des points \(M(x ; y)\)
du plan tels que       
\(det(\overrightarrow{IM}, \vec{u}) = 0\) ?
\item Quel est l'ensemble des points \(M(x ; y)\)
du plan tels que       
\(det(\overrightarrow{JM}, \vec{u}) = 0\) ?
\item Soient un point \(P(x_P ; y_P)\) et un vecteur \(\vec{v} =
        a\vec{j} - b\vec{i}\). Quel est l'ensemble des points \(M(x ; y)\)
du plan tels que       
\(det(\vec{v} , \overrightarrow{PM}) = 0\) ?
\end{enumerate}

\section{Exercice 18}
\label{sec:org9618643}

Soient A et B deux points distincts dans le plan.

Déterminer l'ensemble des points M tels que MA = MB.


\section{Exercice 19}
\label{sec:org1daec18}

On considère deux points A et B distincts.

\begin{enumerate}
\item On souhaite étudier les configurations possibles pour ajouter
un troisième point C de sorte qu'ils soient tous les trois
alignés. En l'absence de repère, quelle égalité vectorielle
doit être vérifiée pour que les trois points soient alignés ? 

\textbf{Rappel} : sans repère vous ne pouvez \emph{pas} utiliser de
coordonnées.
\item Combien existe-t-il de configurations différentes pour que les
points A, B et C soient alignés ?

\textbf{Indication} : faire des dessins.
\item Désormais on ajoute un repère orthonormé \((O ; \vec{i} ,
       \vec{j})\) et par conséquent les points ont les coordonnées
respectives :
\[A(x_A ; y_A), B(x_B ; y_B), C(x_C ; y_C) \]

Comment déterminer si les points sont alignés ou pas ?
\item Écrire un programme Python qui permet de déterminer si 3 points
sont alignés ou non connaissant leurs coordonnées. Dans le cas
où les points sont alignés il indiquera la nature de la
configuration.
\end{enumerate}

\section{Exercice 20}
\label{sec:orgb1e079b}

On se place dans le plan muni d'un repère orthonormé \((O ;
    \vec{i} , \vec{j})\).

\begin{enumerate}
\item Construire le vecteur
\[\vec{u} = 3\vec{i} + 4\vec{j}\]
et placer le point A tel que
\[\overrightarrow{OA} = \vec{u}\]
\item Placer le point H tel que
\[\overrightarrow{OH} = 3\vec{i}\]
et le point K tel que
\[\overrightarrow{OK} = 4\vec{j}\]
En déduire une égalité vectorielle liant \(\overrightarrow{OA},
       \overrightarrow{OH}\) et \(\overrightarrow{OK}\).
\item Quelle est la nature du triangle AHO ? AHI ?
\item Soit M un point quelconque sur l'axe des abscisses. Comparer AM
et AH. Est-il possible de trouver un point M tel que AM < AH ?
\item Quel est le point M de l'axe des abscisses tel que la
distance AM soit minimale ? On appellera cette distance entre A et
l'axe des abscisses :
\[d(A ; (OI)) = \min_{M\in (OI)}(AM)\]
En déduire sa valeur.
\item Quelle est la nature du triangle AKO ? AKJ ?
\item Soit N un point quelconque sur l'axe des ordonnées. Comparer AN
et AK. Est-il possible de trouver un point N tel que AN < AK ?
\item Quel est le point N de l'axe des ordonnées tel que la
distance AN soit minimale ? On appellera cette distance entre A et
l'axe des ordonnées :
\[d(A ; (OJ)) = \min_{N\in (OJ)}(AN)\]
En déduire sa valeur.
\end{enumerate}




\section{Exercice 21}
\label{sec:org60b522b}

On se place dans le plan muni d'un repère orthonormé \((O ;
    \vec{i} , \vec{j})\).

\begin{enumerate}
\item Placer les points I, J et K tels que :
\begin{align*}
OI &= \vec{i}\\
OJ &= \vec{j}\\
OK &= \vec{u} = \vec{i} + \vec{j}\\
\end{align*}
Calculer les coordonnées du point L milieu du segment [OK]. En
déduire la nature des triangles LOI et JOL.
\item Comparer IK et IL. En déduire la distance entre le point I et
la droite (OK).
\item Comparer JK et JL. En déduire la distance entre le point J et
la droite (OK).
\item Construire les points A et B tels que :
\begin{align*}
\overrightarrow{IA} &= \vec{u}\\
\overrightarrow{JB} &= \vec{u}
\end{align*}
En déduire la nature du quadrilatère IABJ.
\item Quelle est la distance entre le point A et la droite (IJ) ?
\end{enumerate}


\section{Exercice 22}
\label{sec:orgc1e738d}

Soit ABC un triangle. Le point G vérifie :
\[\overrightarrow{GA} + \overrightarrow{GB} + \overrightarrow{GC}
    = \vec{0}\]

\begin{enumerate}
\item Exprimer le vecteur \(\overrightarrow{AG}\) en fonction des
vecteurs \(\overrightarrow{AB}\) et \(\overrightarrow{AC}\).
\item Exprimer le vecteur \(\overrightarrow{BG}\) en fonction des
vecteurs \(\overrightarrow{BA}\) et \(\overrightarrow{BC}\).
\item Exprimer le vecteur \(\overrightarrow{CG}\) en fonction des
vecteurs \(\overrightarrow{CB}\) et \(\overrightarrow{CA}\).
\item Déterminer les coordonnées des points I, J et K milieux
respectifs des segments [AB], [BC] et [CA].
\item Exprimer le vecteur \(\overrightarrow{AJ}\) en fonction des
vecteurs \(\overrightarrow{AB}\) et \(\overrightarrow{AC}\). En
déduire une comparaison avec \(\overrightarrow{AG}\).
\item Exprimer le vecteur \(\overrightarrow{BK}\) en fonction des
vecteurs \(\overrightarrow{BA}\) et \(\overrightarrow{BC}\). En
déduire une comparaison avec \(\overrightarrow{BG}\).
\item Exprimer le vecteur \(\overrightarrow{CI}\) en fonction des
vecteurs \(\overrightarrow{CA}\) et \(\overrightarrow{CB}\). En
déduire une comparaison avec \(\overrightarrow{CG}\).
\item En utilisant les éléments précédents (sélectionnez les
informations nécessaires) établir le calcul des coordonnées du
point G en fonction de celles des points A, B, C.
\item En utilisant les éléments précédents (sélectionnez les
informations nécessaires) établir le calcul des coordonnées du
point G en fonction de celles des points I, J, K.
\item Appliquons les résultats précédents dans un cas concret. Dans
un repère orthonormé \((O ; \vec{i}, \vec{j})\) placer les
points \(A(-4 ; 4), B(5 ; 4), C(-1 ; -2)\). Tracer le
triangle, calculer les coordonnées des milieux I, J, K, les
placer, tracer les médianes, calculer les coordonnées du point
G. Que remarquez-vous ?
\end{enumerate}


\section{Exercice 23}
\label{sec:orge0e2170}

On considère la transformation qui à tout point M associe le point
M' tel que :

\[\overrightarrow{OM'} = 2\overrightarrow{OM} + \vec{u}\]

où \(\vec{u}\begin{pmatrix}3\\-1\end{pmatrix}\)

\begin{enumerate}
\item Approche analytique : Si \(M(x;y)\), exprimer les coordonnées de
\(M'(x';y')\)
\item Cas particulier : Trouver le(s) point(s) invariant(s) par cette
transformation
\item Interprétation géométrique : Décrire cette transformation en
termes simples.

\textbf{Indication} : on commencera par déterminer l'image des points
O, I, J.
\item Trouver la transformation réciproque.
\end{enumerate}



\section{Exercice 24}
\label{sec:orge81d35a}

On se place dans le plan muni d'un repère orthonormé \((O ;
    \vec{i} , \vec{j})\) :

\begin{enumerate}
\item Quelle est la nature du triangle OIJ ?
\item Soit A le milieu du segment [IJ]. Exprimer le vecteur
\(\overrightarrow{OA}\) en fonction de \(\vec{i}\) et
\(\vec{j}\).
\item Soit B l'image de A par la translation de vecteur
\(\overrightarrow{OA}\). Calculer OB.
\item Quelle est la nature du quadrilatère OIBJ ?
\item Soit C le point tel que \(\overrightarrow{IC} =
       \vec{i}\). Quelle est la nature du triangle OCB ?
\item Soit K le point tel que \(\overrightarrow{JK} = \vec{i} +
       \vec{j}\) et L le point tel que \(\overrightarrow{IL} =
       \overrightarrow{OB}\).
\item Soit D le point tel que \(\overrightarrow{IC} =
       -\vec{j}\). Quelle est la nature du quadrilatère OBCD ?
\end{enumerate}




\section{Exercice 25}
\label{sec:orgab2ba98}

On se place dans le plan muni d'un repère orthonormé \((O ;
    \vec{i} , \vec{j})\) :

\begin{enumerate}
\item Placer I' image de I par la translation de vecteur
\(-2\vec{i}\) et J' image de J par la translation de vecteur
\(-2\vec{j}\). Quel est le point commun entre les vecteurs
\(\vec{i}, \vec{j}, \overrightarrow{OI'},
       \overrightarrow{OJ'}\) ?
\item Quel est l'ensemble des points \(M(x ; y)\) du plan tels que
\(\lvert\lvert\overrightarrow{OM}\rvert\rvert =
       \lvert\lvert\vec{i}\rvert\rvert\) ?
\item Tracer le cercle \(C_1\) de centre O et de rayon 1. Puis tracer le
cercle \(C_2\) de centre I passant par I' et celui \(C_3\) de
centre I' passant par I. Ces deux derniers cercles (\(C_2\) et
\(C_3\)) se coupent deux fois. Appelons A le point
d'intersection d'ordonnée positive. Calculer IA.
\item Quelle est la nature du triangle IAI' ? On appellera B
l'intersection du segment [I'A] avec le cercle \(C_1\) et C
l'intersection du segment [IA] avec le cercle \(C_1\). Quelle
est la nature du triangle OCB ? Et celle de COI ? En déduire
celle du quadrilatère OICB.
\item Exprimer le vecteur \(\overrightarrow{AB}\) en fonction du
vecteur \(\overrightarrow{AI'}\). Qu'en déduire pour B par
rapport au segment [AI'] ?
\item Exprimer le vecteur \(\overrightarrow{AC}\) en fonction du
vecteur \(\overrightarrow{AI}\). Qu'en déduire pour C par
rapport au segment [AI] ?
\item Placer C' image de C par la translation de vecteur
\(2\overrightarrow{CO}\) et B' image de B par la translation de
vecteur \(2\overrightarrow{BO}\). Quelle est la nature du
polygone ICBI'C'B' ?
\end{enumerate}


\section{Exercice 26}
\label{sec:orga5dc793}

On se place dans le plan muni d'un repère orthonormé \((O ;
    \vec{i} ; \vec{j})\).

Posons \[\vec{u} = \vec{i} + \vec{j}\] pour la suite de
l'exercice. 

\begin{enumerate}
\item Placer le point A tel que \(\overrightarrow{IA} =
       \vec{i}\). Placer le point B tel que \(\overrightarrow{AB} =
       \vec{j}\). Placer le point C tel que \(\overrightarrow{JC} =
       \vec{i}\). Vérifier que le quadrilatère IABC est un carré.
\item Placer le point D tel que
\[\overrightarrow{ID} = \dfrac{IB}{2}(\vec{i}+\vec{j})\]
en déduire ses coordonnées.
\item Placer le point E tel que \(\overrightarrow{DE} = \vec{j}\). En
déduire ses coordonnées.
\item Placer le point F tel que
\[\overrightarrow{EF} = \dfrac{\lvert\lvert\vec{u}\rvert\rvert}{2}(\vec{j}-\vec{i})\]
en déduire ses coordonnées.
\item Placer le point G tel que \(\overrightarrow{FG} = -\vec{i}\).
\item Placer le point H tel que
\[\overrightarrow{GH} = -\dfrac{\lvert\lvert\vec{u}\rvert\rvert}{2}(\vec{u})\]
en déduire ses coordonnées.
\item Placer le point L tel que \(\overrightarrow{HL} =
       -\vec{j}\) en déduire ses coordonnées.
\item Tracer les vecteurs \(\vec{i}, \overrightarrow{ID},
       \overrightarrow{DE}, \overrightarrow{EF}, \overrightarrow{FG},
       \overrightarrow{GH}, \overrightarrow{HL}, \overrightarrow{LO}\).

Vérifier que vous avez bien obtenu le polygone obtenu est un
octogone régulier de côté 1.
\item Pour compléter la figure tracer \(\vec{j}, \overrightarrow{JC},
       \overrightarrow{CI}\) et vérifier que OICJ est bien un carré.
\item Pour ajouter une touche finale tracer les vecteurs :
\begin{align*}
\overrightarrow{OM} = \vec{j} - \vec{i} && \overrightarrow{MN}
= -\vec{u}\\
\overrightarrow{NP} = \vec{i} - \vec{j} && \overrightarrow{PO}
= \vec{u}\\
\overrightarrow{IB} = \vec{u} && \overrightarrow{BQ} = \vec{i} - \vec{j} \\
\overrightarrow{QR} = -\vec{u} && \overrightarrow{RI} = \vec{j} - \vec{i} 
\end{align*}

Vérifier que vous obtenez bien les carrés OMNP et OBQR.
\end{enumerate}


\textbf{Remarque} : avec un peu d'imagination les deux carrés OMNP et
OBQR représenteraient des moteurs, le carré OICJ l'entrée et le
polygone OIDEFGHL le vaisseau.



\part{Exercices de synthèse}
\label{sec:orgb7200a8}
\chapter{Représentation par coordonnées (dans un repère)}
\label{sec:org4eb5641}
\section{Exercice 27}
\label{sec:org3205928}

Dans un repère orthonormé \((O ; \vec{i} ; \vec{j})\),
on donne les points A(2,3), B(-1, 4), et C(5,−2).


\begin{enumerate}
\item Déterminer les coordonnées des vecteurs
\(\overrightarrow{AB}\) et \(\overrightarrow{AC}\).
\item Montrer que les points A, B, et C ne sont pas alignés en
utilisant les vecteurs.
\item Trouver les coordonnées du point D tel que
\[\overrightarrow{AD} = 2\overrightarrow{AB} -
       \overrightarrow{AC}\]
\end{enumerate}

Objectif : Utiliser les coordonnées pour calculer des vecteurs,
vérifier l’alignement, et résoudre des équations vectorielles.

\chapter{Représentation géométrique (sans repère)}
\label{sec:orgc3dd9cf}
\section{Exercice 28}
\label{sec:org948b17c}

Soit un parallélogramme ABCD de centre O.

\begin{enumerate}
\item Exprimer \(\overrightarrow{AO}\) en fonction de
\(\overrightarrow{AB}\) et \(\overrightarrow{AD}\).
\item Montrer que pour tout point M du plan, on a :
\[\overrightarrow{MA} + \overrightarrow{MC} =
       \overrightarrow{MB} + \overrightarrow{MD}\]
\item En déduire l’ensemble des points M tels que
\[\lvert\lvert \overrightarrow{MA} +
       \overrightarrow{MC}\rvert\rvert = 6\]
\end{enumerate}

Objectif : Manipuler les vecteurs avec des relations géométriques
(parallélogramme, milieu) et utiliser la norme pour caractériser
un ensemble de points.


\section{Exercice 29}
\label{sec:org5577b24}

Soit ABCD un carré de centre O.

Déterminer l'ensemble des points M(x, y) du plan tels que OM = OA.

\chapter{Colinéarité et alignement}
\label{sec:org8b114bf}

\section{Exercice 30}
\label{sec:orgacf6529}

Soit les vecteurs \(\vec{u}\begin{pmatrix}3\\-2\end{pmatrix}\) et
\(\vec{v}\begin{pmatrix}-6\\4\end{pmatrix}\).

\begin{enumerate}
\item Montrer que \(\vec{u}\) et \(\vec{v}\) sont colinéaires.
\item Trouver tous les points M(x,y) tels que le vecteur
\(\overrightarrow{OM}\) soit colinéaire à \(\vec{u}\).
\item Les points A(1,0), B(4,−2), et C(−2,1) sont-ils alignés ?
Justifier avec les vecteurs.
\end{enumerate}

Objectif : Utiliser la colinéarité pour caractériser l’alignement
ou trouver des ensembles de points.

\section{Exercice 31}
\label{sec:org4b4d06f}

Dans le plan muni d'un repère orthonormé \((O ; \vec{i} ;
       \vec{j})\) :
\begin{enumerate}
\item Déterminer l'ensemble des points M(x ; y) tels que
\(\overrightarrow{OM} = k\vec{i}\) avec \(k\in\mathbb{R}\).
\item Déterminer l'ensemble des points M(x ; y) tels que
\(\overrightarrow{OM} = k\vec{j}\) avec \(k\in\mathbb{R}\).
\item Posons \(\vec{u} = \vec{i} + \vec{j}\).
Déterminer l'ensemble des points M(x ; y) tels que
\(\overrightarrow{OM} = k\vec{u}\) avec \(k\in\mathbb{R}\).
\item Posons \(\vec{v} = \vec{i} - \vec{j}\).
Déterminer l'ensemble des points M(x ; y) tels que
\(\overrightarrow{OM} = k\vec{v}\) avec \(k\in\mathbb{R}\).
\item Posons \(\vec{w} = -\vec{i} - \vec{j}\).
Déterminer l'ensemble des points M(x ; y) tels que
\(\overrightarrow{OM} = k\vec{w}\) avec \(k\in\mathbb{R}\).
\end{enumerate}



\part{Démonstrations}
\label{sec:org400f8a1}
\chapter{Démonstrations}
\label{sec:orgc09f04b}
\section{Exercice 32}
\label{sec:org6ea90ed}

Démontrer que deux vecteurs sont colinéaires si et seulement si
leur déterminant est nul.

\section{Exercice 32bis}
\label{sec:org4fb03f3}

 Démontrer que pour tous points A, B, C : 
\(\overrightarrow{AB} + \overrightarrow{BC} = \overrightarrow{AC}\)

\section{Exercice 32ter}
\label{sec:orga96d748}

Démontrer que si I est le milieu de [AB], alors :
\(\overrightarrow{AI} = \frac{1}{2}\overrightarrow{AB}\)   

\part{Exercices divers}
\label{sec:org32ba5b4}
\chapter{Alphabet}
\label{sec:org92f9c1a}
\section{Introduction}
\label{sec:orgac1658e}

Dans cette section, vous allez apprendre à maîtriser les vecteurs en 
traçant toutes les lettres de l'alphabet latin. Chaque lettre est une 
construction géométrique précise utilisant les opérations vectorielles.

\textbf{Objectifs} :
\begin{itemize}
\item Renforcer la manipulation des vecteurs de base \(\vec{i}\) et \(\vec{j}\)
\item Visualiser concrètement les combinaisons linéaires
\item Développer la précision du tracé géométrique
\item S'amuser avec les mathématiques !
\end{itemize}

\textbf{Consignes générales} :
\begin{enumerate}
\item Utilisez une règle et un compas pour les tracés
\item Respectez scrupuleusement l'échelle (1 carreau = 1 unité recommandé)
\item Vérifiez visuellement que vous obtenez bien la lettre attendue
\item Si nécessaire, recommencez le tracé pour gagner en précision
\end{enumerate}

\section{Exercice 33}
\label{sec:org24f8688}

Dans le plan muni d'un repère orthonormé \((O ; \vec{i} ;
       \vec{j})\) :

\begin{enumerate}
\item Placer le point A tel que \(\overrightarrow{OA} = 2\vec{i} +
       3\vec{j}\)
\item Construire l'image A' du point A par la translation de vecteur
\(\vec{v} = 2\vec{i} - 3\vec{j}\).
\item Déterminer les coordonnées de milieux des segments [OA] et
[AA'] qu'on nommera H et H' respectivement.
\item Exprimer le vecteur \(\overrightarrow{HH'}\) en fonction des
vecteurs de la base.
\item À quelle lettre de l'alphabet latin peut-on penser lorsqu'on
relie O, A, A' puis H et H' ?
\end{enumerate}


\section{Exercice 34}
\label{sec:org77ae6dd}

Dans le plan muni d'un repère orthonormé \((O ; \vec{i} ;
       \vec{j})\) :

\begin{enumerate}
\item Construire le point A tel que \(\overrightarrow{OA} =
       -3\vec{j}\). En déduire les coordonnées du point A.
\item Construire le point B tel que \(\overrightarrow{AB} =
       2\vec{i}\). En déduire les coordonnées du point B.
\item Construire le point C tel que \(\overrightarrow{BC} =
       \vec{i} + \vec{j}\). En déduire les coordonnées du point C.
\item Construire le point D tel que \(\overrightarrow{CD} =
        \vec{j}\). En déduire les coordonnées du point D.
\item Construire le point E tel que \(\overrightarrow{DE} =
       \vec{j} - \vec{i}\). En déduire les coordonnées du point E.
\item Construire le point F tel que \(\overrightarrow{EF} =
       \vec{i} + \vec{j}\). En déduire les coordonnées du point F.
\item Construire le point G tel que \(\overrightarrow{FG} =
       \vec{j}\). En déduire les coordonnées du point G.
\item Construire le point H tel que \(\overrightarrow{GH} =
       \vec{j} - \vec{i}\). En déduire les coordonnées du point H.
\item Construire le point K tel que \(\overrightarrow{HK} =
       -2\vec{i}\). En déduire les coordonnées du point K.
\item Tracer les vecteurs \(\overrightarrow{OA}, \overrightarrow{AB},
	\overrightarrow{BC}, \overrightarrow{CD}, \overrightarrow{DE},
	\overrightarrow{EF}, \overrightarrow{FG}, \overrightarrow{GH},
	\overrightarrow{HK}, \overrightarrow{KO}, \overrightarrow{OE}\)
et vérifier que vous avez bien obtenu la lettre B.
\end{enumerate}



\section{Exercice 35}
\label{sec:org7dbb118}

Dans le plan muni d'un repère orthonormé \((O ; \vec{i} ;
       \vec{j})\) :

\begin{enumerate}
\item Construire le point A tel que \(\overrightarrow{OA} =
       -\vec{j}\). En déduire les coordonnées du point A.
\item Construire le point B tel que \(\overrightarrow{AB} =
       \vec{i}-\vec{j}\). En déduire les coordonnées du point B.
\item Construire le point C tel que \(\overrightarrow{BC} =
       \vec{i}\). En déduire les coordonnées du point C.
\item Construire le point B' tel que \(\overrightarrow{JB'} =
       \vec{i} + \vec{j}\). En déduire les coordonnées du point B'.
\item Construire le point C' tel que \(\overrightarrow{B'C'} =
       \vec{i}\). En déduire les coordonnées du point C'.
\item Tracer les vecteurs \(\overrightarrow{OA}, \overrightarrow{AB},
	\overrightarrow{BC}, \overrightarrow{OJ}, \overrightarrow{JB'},
	\overrightarrow{B'C'}\) et vérifier que vous avez bien obtenu
la lettre C.
\end{enumerate}



\section{Exercice 36}
\label{sec:orgf5b8f04}

Dans le plan muni d'un repère orthonormé \((O ; \vec{i} ;
       \vec{j})\) :

\begin{enumerate}
\item Construire le point A tel que \(\overrightarrow{OA} =
       -3\vec{j}\). En déduire les coordonnées du point A.
\item Construire le point B tel que \(\overrightarrow{AB} =
       2\vec{i}\). En déduire les coordonnées du point B.
\item Construire le point C tel que \(\overrightarrow{BC} =
       \vec{i} + \vec{j}\). En déduire les coordonnées du point C.
\item Construire le point D tel que \(\overrightarrow{CD} =
        4\vec{j}\). En déduire les coordonnées du point D.
\item Construire le point E tel que \(\overrightarrow{DE} =
       \vec{j} - \vec{i}\). En déduire les coordonnées du point E.
\item Construire le point F tel que \(\overrightarrow{EF} =
       -2\vec{i}\). En déduire les coordonnées du point F.
\item Tracer les vecteurs \(\overrightarrow{OA}, \overrightarrow{AB},
	\overrightarrow{BC}, \overrightarrow{CD}, \overrightarrow{DE},
	\overrightarrow{EF}, \overrightarrow{FO}\)
et vérifier que vous avez bien obtenu la lettre D.
\end{enumerate}



\section{Exercice 37}
\label{sec:orgb50d632}

Dans le plan muni d'un repère orthonormé \((O ; \vec{i} ;
       \vec{j})\) :

\begin{enumerate}
\item Construire le point A tel que \(\overrightarrow{OA} =
       -\vec{j}\). En déduire les coordonnées du point A.
\item Construire le point B tel que \(\overrightarrow{AB} =
       \vec{i}\). En déduire les coordonnées du point B.
\item Construire le point B' tel que \(\overrightarrow{JB'} =
       \vec{i}\). En déduire les coordonnées du point B'.
\item Tracer les vecteurs \(\overrightarrow{OA}, \overrightarrow{AB},
	\overrightarrow{OJ}, \overrightarrow{JB'}\) et vérifier que
vous avez bien obtenu la lettre E.
\end{enumerate}


\section{Exercice 38}
\label{sec:orgccf1fb1}

Dans le plan muni d'un repère orthonormé \((O ; \vec{i} ;
       \vec{j})\) :

\begin{enumerate}
\item Construire le point A tel que \(\overrightarrow{OA} =
       -\vec{j}\). En déduire les coordonnées du point A.
\item Construire le point B tel que \(\overrightarrow{JB} =
       \vec{i}\). En déduire les coordonnées du point B.
\item Tracer les vecteurs \(\overrightarrow{OA}, \overrightarrow{OJ},
       \overrightarrow{JB}\) et vérifier que vous avez bien obtenu la
lettre F.
\end{enumerate}




\section{Exercice 39}
\label{sec:orgaa4959d}

Dans le plan muni d'un repère orthonormé \((O ; \vec{i} ;
       \vec{j})\) :

\begin{enumerate}
\item Construire le point A tel que \(\overrightarrow{IA} =
       \vec{j}\). En déduire les coordonnées du point A.
\item Construire le point B tel que \(\overrightarrow{AB} =
       -\dfrac{1}{2}\vec{i}\). En déduire les coordonnées du point B.
\item Construire le point C tel que \(\overrightarrow{JC} =
       \vec{j}\). En déduire les coordonnées du point C.
\item Construire le point D tel que \(\overrightarrow{JD} =
       \vec{i}\). En déduire les coordonnées du point D.
\item Construire le point E tel que \(\overrightarrow{DE} =
       -\dfrac{1}{2}\vec{j}\). En déduire les coordonnées du point E.
\item Tracer les vecteurs \(\overrightarrow{OI}, \overrightarrow{IA},
       \overrightarrow{OJ}, \overrightarrow{JC}, \overrightarrow{CD},
       \overrightarrow{DE}\) et vérifier que vous avez bien obtenu la
lettre G.
\end{enumerate}


\section{Exercice 40}
\label{sec:org5d1a8c8}

Dans le plan muni d'un repère orthonormé \((O ; \vec{i} ;
       \vec{j})\) :


\begin{enumerate}
\item Construire le point A tel que \(\overrightarrow{OA} =
       -\vec{j}\). En déduire les coordonnées du point A.
\item Construire le point B tel que \(\overrightarrow{IB} =
       -\vec{j}\). En déduire les coordonnées du point B.
\item Construire le point C tel que \(\overrightarrow{IC} =
       \vec{j}\). En déduire les coordonnées du point C.
\item Tracer les vecteurs \(\overrightarrow{OA}, \overrightarrow{OI},
       \overrightarrow{IB}, \overrightarrow{IC}, \overrightarrow{OJ}\)
et vérifier que vous avez bien obtenu la lettre H.
\end{enumerate}



\section{Exercice 41}
\label{sec:org3ae88f0}

Dans le plan muni d'un repère orthonormé \((O ; \vec{i} ;
       \vec{j})\) :

\begin{enumerate}
\item Construire le point A tel que \(\overrightarrow{OA} =
       -\vec{i}\). En déduire les coordonnées du point A.
\item Construire le point B tel que \(\overrightarrow{JB} =
       \vec{j}\). En déduire les coordonnées du point B.
\item Construire le point C tel que \(\overrightarrow{BC} =
       -\vec{i}\). En déduire les coordonnées du point C.
\item Construire le point D tel que \(\overrightarrow{BD} =
       \vec{i}\). En déduire les coordonnées du point D.
\item Tracer les vecteurs \(\overrightarrow{OA}, \overrightarrow{OI},
       \overrightarrow{OJ}, \overrightarrow{JB}, \overrightarrow{BC},
       \overrightarrow{BD}\) et vérifier que vous avez bien obtenu la
lettre I.
\end{enumerate}



\section{Exercice 42}
\label{sec:orge40cfcc}

Dans le plan muni d'un repère orthonormé \((O ; \vec{i} ;
       \vec{j})\) :

\begin{enumerate}
\item Construire le point A tel que \(\overrightarrow{OA} =
       -\vec{i}\). En déduire les coordonnées du point A.
\item Construire le point B tel que \(\overrightarrow{JB} =
       \vec{j}\). En déduire les coordonnées du point B.
\item Construire le point C tel que \(\overrightarrow{BC} =
       -\vec{i}\). En déduire les coordonnées du point C.
\item Construire le point D tel que \(\overrightarrow{BD} =
       \vec{i}\). En déduire les coordonnées du point D.
\item Tracer les vecteurs \(\overrightarrow{OA}, \overrightarrow{OJ},
       \overrightarrow{JB}, \overrightarrow{BC}, \overrightarrow{BD}\)
et vérifier que vous avez bien obtenu la lettre J.
\end{enumerate}


\section{Exercice 43}
\label{sec:orgcc57908}

Dans le plan muni d'un repère orthonormé \((O ; \vec{i} ;
       \vec{j})\) :

\begin{enumerate}
\item Construire le point A tel que \(\overrightarrow{JA} =
       \vec{j}\). En déduire les coordonnées du point A.
\item Construire le point B tel que \(\overrightarrow{JB} =
       \vec{i} + \vec{j}\). En déduire les coordonnées du point B.
\item Construire le point C tel que \(\overrightarrow{JC} =
       \vec{i} - \vec{j}\). En déduire les coordonnées du point C.
\item Tracer les vecteurs \(\overrightarrow{OJ}, \overrightarrow{JC},
       \overrightarrow{JB}, \overrightarrow{JA}\) et vérifier que vous
avez bien obtenu la lettre K.
\end{enumerate}




\section{Exercice 44}
\label{sec:orgba27a70}

Dans le plan muni d'un repère orthonormé \((O ; \vec{i} ;
       \vec{j})\) :

\begin{enumerate}
\item En partant de O tracer le vecteur \(\vec{i}\), on nommera I son
extrémité.
\item En partant de O tracer le vecteur \(\vec{j}\), on nommera J son
extrémité.
\item Vérifier que l'enchaînement \(\overrightarrow{OI}\) puis
\(\overrightarrow{OJ}\) donne bien la lettre L.
\end{enumerate}


\section{Exercice 45}
\label{sec:orgae1c281}

Dans le plan muni d'un repère orthonormé \((O ; \vec{i} ;
       \vec{j})\) :

\begin{enumerate}
\item Construire le point A tel que \(\overrightarrow{JA} =
       \dfrac{1}{2}(\vec{i} - \vec{j})\). En déduire les coordonnées
du point A.
\item Construire le point B tel que \(\overrightarrow{AB} =
       \dfrac{1}{2}(\vec{i} + \vec{j})\). En déduire les coordonnées du point B.
\item Tracer les vecteurs \(\overrightarrow{OJ}, \overrightarrow{JA},
       \overrightarrow{AB}, \overrightarrow{BI}\) et vérifier que vous
avez bien obtenu la lettre M.
\end{enumerate}




\section{Exercice 46}
\label{sec:orgc53083a}

Dans le plan muni d'un repère orthonormé \((O ; \vec{i} ;
       \vec{j})\). 

\begin{enumerate}
\item Construire le point A tel que \(\overrightarrow{IA} =
       \vec{j}\). En déduire les coordonnées du point A.
\item Tracer les vecteurs \(\overrightarrow{OJ}, \overrightarrow{JI},
       \overrightarrow{IA}\) et vérifier que vous avez bien obtenu la
lettre N.
\end{enumerate}




\section{Exercice 47}
\label{sec:org78d586f}

Dans le plan muni d'un repère orthonormé \((O ; \vec{i} ;
       \vec{j})\).

\begin{enumerate}
\item Construire le point A tel que \(\overrightarrow{OA} =
       \vec{i} + \vec{j}\). En déduire les coordonnées du point A.
\item Tracer les vecteurs \(\overrightarrow{OJ}, \overrightarrow{OI},
       \overrightarrow{IA}, \overrightarrow{AJ}\) et vérifier que vous
avez bien obtenu la lettre O.
\end{enumerate}


\section{Exercice 48}
\label{sec:org025012e}

Dans le plan muni d'un repère orthonormé \((O ; \vec{i} ;
       \vec{j})\).

\begin{enumerate}
\item Construire le point A tel que \(\overrightarrow{OA} =
       \vec{i} + \vec{j}\). En déduire les coordonnées du point A.
\item Construire le point B tel que \(\overrightarrow{OB} =
       -\vec{j}\). En déduire les coordonnées du point B.
\item Tracer les vecteurs \(\overrightarrow{OJ}, \overrightarrow{OI},
       \overrightarrow{IA}, \overrightarrow{AJ}, \overrightarrow{OB}\)
et vérifier que vous avez bien obtenu la lettre P.
\end{enumerate}




\section{Exercice 49}
\label{sec:org29d9836}

Dans le plan muni d'un repère orthonormé \((O ; \vec{i} ;
       \vec{j})\) :

\begin{enumerate}
\item Construire le point A tel que \(\overrightarrow{OA} =
       \vec{i} + \vec{j}\). En déduire les coordonnées du point A.
\item Construire le point B tel que \(\overrightarrow{IB} =
       -\vec{j}\). En déduire les coordonnées du point B.
\item Tracer les vecteurs \(\overrightarrow{OJ}, \overrightarrow{OI},
       \overrightarrow{IA}, \overrightarrow{AJ}, \overrightarrow{IB}\)
et vérifier que vous avez bien obtenu la lettre Q.
\end{enumerate}



\section{Exercice 50}
\label{sec:orga7f95f4}

Dans le plan muni d'un repère orthonormé \((O ; \vec{i} ;
       \vec{j})\) :

\begin{enumerate}
\item Construire le point A tel que \(\overrightarrow{OA} =
       \vec{i} + \vec{j}\). En déduire les coordonnées du point A.
\item Construire le point B tel que \(\overrightarrow{IB} =
       -\vec{j}\). En déduire les coordonnées du point B.
\item Construire le point C tel que \(\overrightarrow{OC} =
       -\vec{j}\). En déduire les coordonnées du point C.
\item Tracer les vecteurs \(\overrightarrow{OJ}, \overrightarrow{OI},
       \overrightarrow{IA}, \overrightarrow{AJ}, \overrightarrow{JB},
       \overrightarrow{JC}\) et vérifier que vous avez bien obtenu la
lettre R.
\end{enumerate}



\section{Exercice 51}
\label{sec:org5782c94}

Dans le plan muni d'un repère orthonormé \((O ; \vec{i} ;
       \vec{j})\) :

\begin{enumerate}
\item Construire le point A tel que \(\overrightarrow{OA} =
       \vec{i} + \vec{j}\). En déduire les coordonnées du point A.
\item Construire le point B tel que \(\overrightarrow{IB} =
       -\vec{j}\). En déduire les coordonnées du point B.
\item Construire le point C tel que \(\overrightarrow{OC} =
       -\vec{j}\). En déduire les coordonnées du point C.
\item Tracer les vecteurs \(\overrightarrow{OI}, \overrightarrow{IB},
       \overrightarrow{BC}, \overrightarrow{OJ}, \overrightarrow{JA}\)
et vérifier que vous avez bien obtenu la lettre S.
\end{enumerate}




\section{Exercice 52}
\label{sec:orgd10c125}

Dans le plan muni d'un repère orthonormé \((O ; \vec{i} ;
       \vec{j})\). 

\begin{enumerate}
\item Construire le point A tel que \(\overrightarrow{OA} =
       \vec{i} + \vec{j}\). En déduire les coordonnées du point A.
\item Construire le point B tel que \(\overrightarrow{JB} =
       -\vec{u}\). En déduire les coordonnées du point B.
\item Construire le point C tel que \(\overrightarrow{OC} =
       -\vec{j}\). En déduire les coordonnées du point C.
\item Tracer les vecteurs \(\overrightarrow{OC}, \overrightarrow{OJ},
       \overrightarrow{JA}, \overrightarrow{JB}\) et vérifier que vous
avez bien obtenu la lettre T.
\end{enumerate}




\section{Exercice 53}
\label{sec:orgcf61a28}

Dans le plan muni d'un repère orthonormé \((O ; \vec{i} ;
       \vec{j})\) :

\begin{enumerate}
\item Construire le point A tel que \(\overrightarrow{OA} =
       \vec{i} + \vec{j}\). En déduire les coordonnées du point A.
\item Tracer les vecteurs \(\overrightarrow{OJ}, \overrightarrow{OI},
       \overrightarrow{IA}\) et vérifier que vous avez bien obtenu la
lettre U.
\end{enumerate}


\section{Exercice 54}
\label{sec:org0f336b0}

Dans le plan muni d'un repère orthonormé \((O ; \vec{i} ;
       \vec{j})\) 

\begin{enumerate}
\item Construire le point A tel que \(\overrightarrow{OA} =
       \dfrac{1}{2}\vec{i} + \vec{j}\). En déduire les coordonnées du
point A.
\item Construire le point B tel que \(\overrightarrow{OB} =
       \vec{j} - \dfrac{1}{2}\vec{i}\). En déduire les coordonnées du
point B.
\item Tracer les vecteurs \(\overrightarrow{OA},
       \overrightarrow{OB}\) et vérifier que vous avez bien obtenu la
lettre V.
\end{enumerate}



\section{Exercice 55}
\label{sec:org1022422}

Dans le plan muni d'un repère orthonormé \((O ; \vec{i} ;
       \vec{j})\)  :

\begin{enumerate}
\item Construire le point A tel que \(\overrightarrow{OA} =
       \dfrac{1}{2}\vec{i} + \vec{j}\). En déduire les coordonnées du
point A.
\item Construire le point B tel que \(\overrightarrow{OB} =
       \vec{j} - \dfrac{1}{2}\vec{i}\). En déduire les coordonnées du
point B.
\item Construire le point C tel que \(\overrightarrow{IC} =
       \dfrac{1}{2}\vec{i} + \vec{j}\). En déduire les coordonnées du
point C.
\item Tracer les vecteurs \(\overrightarrow{OA},
       \overrightarrow{OB}, \overrightarrow{IA}, \overrightarrow{IC}\)
et vérifier que vous avez bien obtenu la lettre W.
\end{enumerate}

\section{Exercice 56}
\label{sec:orgb02178f}

Dans le plan muni d'un repère orthonormé \((O ; \vec{i} ;
       \vec{j})\) 

\begin{enumerate}
\item Construire le point A tel que \(\overrightarrow{OA} =
       \dfrac{1}{2}\vec{i} + \vec{j}\). En déduire les coordonnées du
point A.
\item Construire le point B tel que \(\overrightarrow{OB} =
       \vec{j} - \dfrac{1}{2}\vec{i}\). En déduire les coordonnées du
point B.
\item Construire le point C tel que \(\overrightarrow{OC} =
       -\dfrac{1}{2}\vec{i} - \vec{j}\). En déduire les coordonnées du
point C.
\item Construire le point D tel que \(\overrightarrow{OD} =
       \dfrac{1}{2}\vec{i} - \vec{j}\). En déduire les coordonnées du
point D.
\item Tracer les vecteurs \(\overrightarrow{OA},
       \overrightarrow{OB}, \overrightarrow{OC}, \overrightarrow{OD}\)
et vérifier que vous avez bien obtenu la lettre X.
\end{enumerate}


\section{Exercice 57}
\label{sec:org7c46788}

Dans le plan muni d'un repère orthonormé \((O ; \vec{i} ;
       \vec{j})\). Veiller à NE PAS tracer le vecteur \(\vec{i}\).

\begin{enumerate}
\item Construire le point A tel que \(\overrightarrow{OA} =
       \dfrac{1}{2}\vec{i} + \vec{j}\). En déduire les coordonnées du
point A.
\item Construire le point B tel que \(\overrightarrow{OB} =
       \vec{j} - \dfrac{1}{2}\vec{i}\). En déduire les coordonnées du
point B.
\item Construire le point C tel que \(\overrightarrow{OC} =
       -\vec{j}\). En déduire les coordonnées du
point C.
\item Tracer les vecteurs \(\overrightarrow{OA},
       \overrightarrow{OB}, \overrightarrow{OC}\) et vérifier que vous
avez bien obtenu la lettre Y.
\end{enumerate}





\section{Exercice 58}
\label{sec:org735a5f2}

Dans le plan muni d'un repère orthonormé \((O ; \vec{i} ;
       \vec{j})\). 

\begin{enumerate}
\item Construire le point A tel que \(\overrightarrow{OA} =
       \vec{i} + \vec{j}\). En déduire les coordonnées du
point A.
\item Tracer les vecteurs \(\overrightarrow{OI},
       \overrightarrow{OA}, \overrightarrow{AJ}\) et vérifier que vous
avez bien obtenu la lettre Z.
\end{enumerate}


\chapter{Applications des vecteurs dans la vie réelle}
\label{sec:orga525dec}

\section{Exercice 59}
\label{sec:org7b82c50}

Un drone se déplace selon les vecteurs successifs :
\begin{align*}
\vec{v}_1&\begin{pmatrix}100\\50\end{pmatrix}  \\
\vec{v}_2&\begin{pmatrix}-30\\80\end{pmatrix}  \\
\vec{v}_3&\begin{pmatrix}-20\\-100\end{pmatrix} 
\end{align*}

Calculer sa position finale et la distance parcourue au total.


\section{Exercice 60}
\label{sec:orgea2a0ad}

On considère un échiquier standard comme celui sur la figure
ci-dessous :

\begin{center}
\includegraphics[width=.9\linewidth]{./img/chessboard_wiki.jpg}
\end{center}

L'échiquier est traditionnellement codé avec les lettres de a à h
pour indiquer la position horizontale et les nombres de 1 à 8 pour
indiquer la position verticale.

De cette façon le codage a1 correspond à la case en bas à gauche
et le codage h8 à la case en haut à droite.

Afin de rendre cette représentation utilisable dans le cadre
vectoriel conforme à la classe de seconde on va considérer que
l'origine du repère sera la case a1 qu'on identifiera à O.

De même, le vecteur \(\vec{i}\) est identifié à la translation de
a1 vers b1 et le vecteur \(\vec{j}\) à celle de a1 vers a2.

Considérons les déplacements possibles du fou blanc comme indiqué
sur la photo ci-dessous :

\begin{center}
\includegraphics[width=.9\linewidth]{./img/fou_motion.png}
\end{center}

\begin{enumerate}
\item Quelles sont les coordonnées du vecteur \(\vec{v}_1\) indiquant
le déplacement de la case h1 vers la case a8 (appelée aussi
anti-diagonale ou seconde diagonale) ?
\item Calculer la norme de ce vecteur
\(\lvert\lvert\vec{v}_1\rvert\rvert\).
\item Mêmes questions pour le fou noir et la diagonale a1 vers h8
représentée par le vecteur \(\vec{v}_2\).
\item Mêmes questions pour la sur-diagonale du fou blanc débutant à
la case a2 et finissant à la case g8, on nommera le vecteur
\(\vec{v}_3\).
\item Mêmes questions pour la sur-diagonale du fou noir débutant à la
case a3 et finissant à la case f8, on nommera le vecteur
\(\vec{v}_4\).
\item Idem pour \(\vec{v}_5\) qui part de a4 et termine en e8.
\item Idem pour \(\vec{v}_6\) qui part de a5 et termine en d8.
\item Idem pour \(\vec{v}_7\) qui commence en a6 et finit en c8.
\item Idem pour \(\vec{v}_8\) qui commence en a7 et finit en b8.
\item Idem pour \(\vec{v}_9\) qui commence en a8 et finit en a8.
\item Si on souhaite généraliser les déplacements du fou en
programmant un seul vecteur du type : \(\vec{v} = a\vec{i} +
        b\vec{j}\) avec
\[\vec{i}\begin{pmatrix}1\\0\end{pmatrix}\] et
\[\vec{j}\begin{pmatrix}0\\1\end{pmatrix}\]
les vecteurs de la base canonique, alors quelles sont les
valeurs possibles pour les nombres \(a\) et \(b\) ?
\end{enumerate}



\section{Exercice 61}
\label{sec:org03aabd7}

Un projectile est lancé depuis O(0;0) avec une vitesse initiale 
\(\vec{v}_0\) de coordonnées \(\begin{pmatrix}20\\30\end{pmatrix}\) m/s.

La gravité exerce une accélération constante 
\(\vec{g}\begin{pmatrix}0\\-10\end{pmatrix}\) m/s².

Après 1 seconde, la vitesse devient \(\vec{v}_1 = \vec{v}_0 + \vec{g}\).

\begin{enumerate}
\item Calculer les coordonnées de \(\vec{v}_1\)
\item Calculer la norme de \(\vec{v}_1\) (vitesse en m/s)
\item La position après 1s est \(\vec{p}_1 = \vec{v}_0 + \frac{1}{2}\vec{g}\).
Calculer les coordonnées du point P₁.
\item Après combien de temps le projectile retombe-t-il au sol (y=0) ?
\end{enumerate}





\section{Exercice 62 : Images vectorielles vs images bitmap}
\label{sec:orga0280e7}

\begin{figure}[h]
\centering
\begin{tikzpicture}[scale=0.8]
  % Image bitmap
  \begin{scope}
    \node at (2.25,4.5) {\textbf{Image Bitmap (PNG)}};
    \draw[step=0.5cm,gray,very thin] (0,0) grid (4,3);
    \foreach \x in {0,0.5,...,4}
      \foreach \y in {0,0.5,...,3}
        \fill[black!20] (\x,\y) rectangle ++(0.5,0.5);
    \draw[thick,blue] (0.5,0.5) -- (1.5,2.5) -- (2.5,0.5);
    \node[below] at (2,-0.5) {40 000 pixels pour 200×200};
  \end{scope}
  
  % Image vectorielle
  \begin{scope}[xshift=6cm]
    \node at (2.25,4.5) {\textbf{Image Vectorielle (SVG)}};
    \draw[thick,blue,->] (0.5,0.5) -- node[left] {$\vec{v}_1$} (1.5,2.5);
    \draw[thick,blue,->] (1.5,2.5) -- node[right] {$\vec{v}_2$} (2.5,0.5);
    \fill (0.5,0.5) circle (2pt) node[below left] {A};
    \fill (1.5,2.5) circle (2pt) node[above] {B};
    \fill (2.5,0.5) circle (2pt) node[below right] {C};
    \node[below] at (2,-0.5) {3 points + 2 vecteurs};
  \end{scope}
\end{tikzpicture}
\caption{Différence entre image bitmap et image vectorielle}
\end{figure}

Les images numériques peuvent être de deux types : \textbf{bitmap} 
(aussi appelées matricielles ou raster) ou \textbf{vectorielles}.

\begin{itemize}
\item Une image bitmap (JPEG, PNG, GIF) est constituée d'une grille 
de pixels. Chaque pixel a une couleur définie. Si on agrandit 
l'image, elle devient floue et pixelisée.

\item Une image vectorielle (SVG, PDF vectoriel) est constituée de 
formules mathématiques décrivant des formes géométriques 
(lignes, courbes, polygones). On peut l'agrandir infiniment 
sans perte de qualité.
\end{itemize}

\textbf{Questions :}

\begin{enumerate}
\item Dans un logiciel de dessin vectoriel, on trace un segment de 
A(100 ; 150) à B(300 ; 450). Quelles sont les coordonnées du 
vecteur \(\overrightarrow{AB}\) ?

\item Pour afficher ce segment à l'écran, l'ordinateur calcule sa 
longueur. Quelle est-elle (en pixels) ?

\item L'utilisateur applique un zoom ×2 (homothétie de rapport 2 
centrée à l'origine). Quelles sont les nouvelles coordonnées 
de A' et B' ?

\item Le vecteur \(\overrightarrow{A'B'}\) est-il colinéaire à 
\(\overrightarrow{AB}\) ? Justifier.

\item Quelle est la longueur du segment [A'B'] après le zoom ?

\item \textbf{\textbf{Question de réflexion}} : Pourquoi dit-on qu'une image 
vectorielle peut être agrandie "infiniment" sans perte de 
qualité, contrairement à une image bitmap ?
\end{enumerate}

\textbf{Note} : Les formats SVG (Scalable Vector Graphics) et PDF 
utilisent exactement ces principes mathématiques. C'est pourquoi 
les logos d'entreprises, icônes d'applications, et schémas 
techniques sont toujours créés en vectoriel !

\section{Exercice 63 : Décoder un fichier SVG}
\label{sec:orgd27f3dc}

Le format SVG (Scalable Vector Graphics) est un format d'image 
vectorielle basé sur du code XML. Voici un extrait simplifié 
d'un fichier SVG :

\begin{verbatim}
<svg
    width="500"
    height="400">
  <line
      x1="50" y1="100"
      x2="200" y2="250"
      stroke="blue" />
  <line
      x1="200" y2="250"
      x2="350" y2="100"
      stroke="red" />
  <circle
      cx="200" cy="175"
      r="80" fill="none"
      stroke="green" />
</svg>
\end{verbatim}

Ce code décrit :
\begin{itemize}
\item Une ligne bleue du point A(50 ; 100) au point B(200 ; 250)
\item Une ligne rouge du point B(200 ; 250) au point C(350 ; 100)
\item Un cercle vert de centre O(200 ; 175) et de rayon 80 pixels
\end{itemize}

\textbf{Attention} : En SVG, l'axe des ordonnées est \textbf{\textbf{inversé}} par 
rapport au repère mathématique classique (l'origine est en haut 
à gauche, y augmente vers le bas).

\textbf{Questions :}

\begin{enumerate}
\item Calculer les coordonnées des vecteurs \(\overrightarrow{AB}\) 
et \(\overrightarrow{BC}\) dans le repère SVG.

\item Les points A, B, C sont-ils alignés ? Justifier avec le 
déterminant.

\item Quelle est la nature du triangle ABC ?

\item Le centre O du cercle appartient-il au segment [AC] ? 
Justifier.

\item \textbf{\textbf{Défi}} : Écrire le code SVG d'un carré DEFG de côté 100 
pixels avec D(100 ; 100) comme sommet en haut à gauche.
\end{enumerate}

\textbf{Pour aller plus loin} : Ouvrez un fichier .svg avec un éditeur 
de texte (Bloc-notes, TextEdit) et observez le code. Vous verrez 
des vecteurs partout !

\section{Exercice 64 : Pourquoi les PDF sont-ils vectoriels ?}
\label{sec:org7551f56}

Un fichier PDF peut contenir du texte, des images et des 
graphiques. Le texte et les graphiques sont généralement 
stockés sous forme \textbf{vectorielle}.

Prenons l'exemple de la lettre "A" :
\begin{itemize}
\item Dans une image bitmap (photo), le "A" est une grille de 
pixels noirs et blancs
\item Dans un PDF vectoriel, le "A" est décrit par ses contours : 
deux segments obliques et un segment horizontal
\end{itemize}

\textbf{Situation} : Un \emph{designer} (graphiste en bon français) crée un logo avec la lettre "V" formée 
par deux segments :
\begin{itemize}
\item Segment 1 : de O(0 ; 100) à M(50 ; 0)
\item Segment 2 : de M(50 ; 0) à N(100 ; 100)
\end{itemize}

\textbf{Questions :}

\begin{enumerate}
\item Quelles sont les coordonnées des vecteurs 
\(\overrightarrow{OM}\) et \(\overrightarrow{MN}\) ?

\item Ces deux vecteurs sont-ils colinéaires ? Que peut-on en 
déduire sur la forme de la lettre "V" ?

\item Calculer les normes \(\|\overrightarrow{OM}\|\), 
\(\|\overrightarrow{MN}\|\) et \(\|\overrightarrow{ON}\|\). La
lettre "V" est-elle symétrique ?

\item Le \emph{designer} (graphiste en bon français) applique une transformation : tous les points 
sont multipliés par 2. Quelles sont les nouvelles coordonnées 
de O', M', N' ?

\item Les segments [O'M'] et [M'N'] conservent-ils les mêmes angles 
que [OM] et [MN] ? Pourquoi ?

\item \textbf{\textbf{Application}} : Expliquez pourquoi un PDF contenant du texte 
reste net même quand on zoom à 400\%, alors qu'une photo devient 
floue.
\end{enumerate}

\textbf{À retenir} : Quand vous convertissez un document Word en PDF, 
tout le texte est transformé en vecteurs. C'est pour ça qu'on peut 
zoomer sans pixelisation !

\section{Exercice 65 : Bitmap vs Vectoriel - Étude comparative}
\label{sec:org89067c6}

On souhaite créer un logo carré de 200×200 pixels pour une 
application mobile.

\textbf{Méthode 1 : Image bitmap (PNG)}
\begin{itemize}
\item Le logo est une grille de 200×200 = 40 000 pixels
\item Chaque pixel stocke sa couleur (3 octets RGB)
\item Taille du fichier : ≈ 120 Ko (avec compression)
\item Si on agrandit à 400×400, il faut recalculer 160 000 pixels 
par interpolation → image floue
\end{itemize}

\textbf{Méthode 2 : Image vectorielle (SVG)}
\begin{itemize}
\item Le logo est décrit par 4 segments formant un carré
\item Stockage : 4 vecteurs avec leurs coordonnées
\item Taille du fichier : ≈ 1 Ko
\item Si on agrandit, on multiplie simplement les coordonnées par 2 
→ image parfaitement nette
\end{itemize}

\textbf{Questions :}

\begin{enumerate}
\item Un carré vectoriel a pour sommets A(0;0), B(200;0), C(200;200), 
D(0;200). Donner les coordonnées des 4 vecteurs 
\(\overrightarrow{AB}\), \(\overrightarrow{BC}\), 
\(\overrightarrow{CD}\), \(\overrightarrow{DA}\).

\item On applique un zoom ×3. Quelles sont les nouvelles coordonnées 
des 4 sommets ?

\item Calculer le périmètre du carré initial et du carré agrandi. 
Vérifier que le rapport est bien 3.

\item \textbf{\textbf{Calcul de taille}} : 
a) Bitmap : Combien de pixels dans l'image agrandie (600×600) ?
b) Vectoriel : Combien de vecteurs dans l'image agrandie ?
\end{enumerate}


\textbf{Conclusion} : Les vecteurs que vous apprenez en seconde sont 
utilisés quotidiennement par des millions d'ordinateurs, 
smartphones et imprimantes pour afficher des textes, logos, 
icônes et graphiques !

\section{Exercice 66 : Créer une icône vectorielle (Projet)}
\label{sec:org7621703}

Les icônes d'applications sur smartphone sont toujours créées 
en vectoriel pour s'adapter aux différentes tailles d'écran 
(iPhone, iPad, etc.).

\textbf{Projet} : Créer l'icône ⚡ (éclair) en utilisant uniquement 
des vecteurs.

\textbf{Étape 1 : Conception}
On dessine l'éclair avec 7 points :
\begin{itemize}
\item A(100 ; 0)     [sommet haut]
\item B(60 ; 80)
\item C(80 ; 80)
\item D(40 ; 200)    [sommet bas]
\item E(80 ; 120)
\item F(60 ; 120)
\item G(100 ; 0)     [retour au début]
\end{itemize}

\textbf{Questions :}

\begin{enumerate}
\item Calculer les vecteurs \(\overrightarrow{AB}\), 
\(\overrightarrow{BC}\), \(\overrightarrow{CD}\), 
\(\overrightarrow{DE}\), \(\overrightarrow{EF}\), 
\(\overrightarrow{FG}\).

\item Vérifier que \(\overrightarrow{AB} + \overrightarrow{BC} +
       \overrightarrow{CD} + \overrightarrow{DE} +
       \overrightarrow{EF} + \overrightarrow{FG} = \vec{0}\).
Que signifie ce résultat ?

\item L'icône doit être affichée en 3 tailles : 32×32, 64×64, 128×128 
pixels. Pour chaque taille, donner le facteur d'homothétie à 
appliquer sachant que le dessin initial fait 200×200.
\end{enumerate}

\textbf{Pour aller plus loin} : Les \emph{designer} (graphiste en bon
français)s utilisent des logiciels comme Adobe Illustrator,
Inkscape (gratuit), ou Figma pour créer des images
vectorielles. Tous ces outils manipulent des vecteurs
mathématiques en arrière-plan !



\chapter{Chiffres}
\label{sec:org02cc156}
\section{Introduction}
\label{sec:org19763a3}

Dans cette section, vous allez apprendre à maîtriser les vecteurs en 
traçant tous les chiffres indo-arabes du système décimal. Chaque
chiffre est une construction géométrique précise utilisant les
opérations vectorielles.


\textbf{Objectifs} :
\begin{itemize}
\item Renforcer la manipulation des vecteurs de base \(\vec{i}\) et \(\vec{j}\)
\item Visualiser concrètement les combinaisons linéaires
\item Développer la précision du tracé géométrique
\item S'amuser avec les mathématiques !
\end{itemize}

\textbf{Consignes générales} :
\begin{enumerate}
\item Utilisez une règle et un compas pour les tracés
\item Respectez scrupuleusement l'échelle (1 carreau = 1 unité recommandé)
\item Vérifiez visuellement que vous obtenez bien la lettre attendue
\item Si nécessaire, recommencez le tracé pour gagner en précision
\end{enumerate}

\section{Exercice 67}
\label{sec:org65bf7d7}

Dans le plan muni du repère orthonormé canonique
\[(O ; \vec{i} ; \vec{j})\]

\begin{enumerate}
\item Placer le point A tel que \(\overrightarrow{OA} =
           -\vec{i}\). En déduire ses coordonnées.
\item Placer le point B tel que \(\overrightarrow{JB} =
           \vec{j}\). En déduire ses coordonnées.
\item Placer le point C tel que \(\overrightarrow{BC} =
           -\vec{i} - \vec{j}\). En déduire ses coordonnées.
\item Tracer les vecteurs \(\overrightarrow{OA}, \vec{i},
           \vec{j}, \overrightarrow{JB}, \overrightarrow{BC}\).

Vérifier que vous avez bien obtenu le chiffre 1.
\end{enumerate}



\section{Exercice 68}
\label{sec:orge672b47}

Dans le plan muni du repère orthonormé canonique
\[(O ; \vec{i} ; \vec{j})\]

\begin{enumerate}
\item Placer le point A tel que \(\overrightarrow{JA} =
           \vec{i}\). En déduire ses coordonnées.
\item Placer le point B tel que \(\overrightarrow{AB} =
           \vec{j}\). En déduire ses coordonnées.
\item Placer le point C tel que \(\overrightarrow{BC} =
           -\vec{i}\). En déduire ses coordonnées.
\item Tracer les vecteurs \(\vec{i}, \vec{j},
           \overrightarrow{JA}, \overrightarrow{AB}, \overrightarrow{BC}\).

Vérifier que vous avez bien obtenu le chiffre 2.
\end{enumerate}



\section{Exercice 69}
\label{sec:org63b759e}

Dans le plan muni du repère orthonormé canonique
\[(O ; \vec{i} ; \vec{j})\]

\begin{enumerate}
\item Placer le point A tel que \(\overrightarrow{IA} =
           \vec{j}\). En déduire ses coordonnées.
\item Placer le point B tel que \(\overrightarrow{AB} =
           \vec{j}\). En déduire ses coordonnées.
\item Placer le point C tel que \(\overrightarrow{BC} =
           -\vec{i}\). En déduire ses coordonnées.
\item Tracer les vecteurs \(\vec{i}, \overrightarrow{IA},
           \overrightarrow{AJ}, \overrightarrow{AB}, \overrightarrow{BC}\).

Vérifier que vous avez bien obtenu le chiffre 3.
\end{enumerate}


\section{Exercice 70}
\label{sec:orge7947a5}

Dans le plan muni du repère orthonormé canonique
\[(O ; \vec{i} ; \vec{j})\]

\begin{enumerate}
\item Placer le point A tel que \(\overrightarrow{IA} =
           \vec{j}\). En déduire ses coordonnées.
\item Placer le point B tel que \(\overrightarrow{AB} =
           \vec{j}\). En déduire ses coordonnées.
\item Placer le point C tel que \(\overrightarrow{JC} =
           \vec{j}\). En déduire ses coordonnées.
\item Tracer les vecteurs \(\overrightarrow{IA},
           \overrightarrow{AJ}, \overrightarrow{AB}, \overrightarrow{JC}\).

Vérifier que vous avez bien obtenu le chiffre 4.
\end{enumerate}

\section{Exercice 71}
\label{sec:org3938be4}

Dans le plan muni du repère orthonormé canonique
\[(O ; \vec{i} ; \vec{j})\]

\begin{enumerate}
\item Placer le point A tel que \(\overrightarrow{IA} =
           \vec{j}\). En déduire ses coordonnées.
\item Placer le point B tel que \(\overrightarrow{JB} =
           \vec{j}\). En déduire ses coordonnées.
\item Placer le point C tel que \(\overrightarrow{BC} =
           \vec{i}\). En déduire ses coordonnées.
\item Tracer les vecteurs \(\vec{i}, \overrightarrow{IA},
           \overrightarrow{AJ}, \overrightarrow{JB}, \overrightarrow{BC}\).

Vérifier que vous avez bien obtenu le chiffre 5.
\end{enumerate}


\section{Exercice 72}
\label{sec:orgd34bb22}

Dans le plan muni du repère orthonormé canonique
\[(O ; \vec{i} ; \vec{j})\]

\begin{enumerate}
\item Placer le point A tel que \(\overrightarrow{IA} =
           \vec{j}\). En déduire ses coordonnées.
\item Placer le point B tel que \(\overrightarrow{JB} =
           \vec{j}\). En déduire ses coordonnées.
\item Placer le point C tel que \(\overrightarrow{BC} =
           \vec{i}\). En déduire ses coordonnées.
\item Tracer les vecteurs \(\vec{i}, \vec{j}, \overrightarrow{IA},
           \overrightarrow{AJ}, \overrightarrow{JB}, \overrightarrow{BC}\).

Vérifier que vous avez bien obtenu le chiffre 6.
\end{enumerate}


\section{Exercice 73}
\label{sec:org6948811}

Dans le plan muni du repère orthonormé canonique
\[(O ; \vec{i} ; \vec{j})\]

\begin{enumerate}
\item Placer le point A tel que \(\overrightarrow{OA} =
           \vec{i} + 2\vec{j}\). En déduire ses coordonnées.
\item Placer le point B tel que \(\overrightarrow{AB} =
           -\vec{i}\). En déduire ses coordonnées.
\item Tracer les vecteurs \(\overrightarrow{OA},
           \overrightarrow{AB}\).

Vérifier que vous avez bien obtenu le chiffre 7.
\end{enumerate}


\section{Exercice 74}
\label{sec:orgeb3c98b}

Dans le plan muni du repère orthonormé canonique
\[(O ; \vec{i} ; \vec{j})\]

\begin{enumerate}
\item Placer le point A tel que \(\overrightarrow{IA} =
           \vec{j}\). En déduire ses coordonnées.
\item Placer le point B tel que \(\overrightarrow{AB} =
           \vec{j}\). En déduire ses coordonnées.
\item Placer le point C tel que \(\overrightarrow{BC} =
           -\vec{i}\). En déduire ses coordonnées.
\item Tracer les vecteurs \(\vec{i}, \vec{j}, \overrightarrow{IA},
           \overrightarrow{AB}, \overrightarrow{BC},
           \overrightarrow{CJ}, \overrightarrow{JA}\).

Vérifier que vous avez bien obtenu le chiffre 8.
\end{enumerate}

\section{Exercice 75}
\label{sec:org35b1c9f}

Dans le plan muni du repère orthonormé canonique
\[(O ; \vec{i} ; \vec{j})\]

\begin{enumerate}
\item Placer le point A tel que \(\overrightarrow{IA} =
           \vec{j}\). En déduire ses coordonnées.
\item Placer le point B tel que \(\overrightarrow{AB} =
           \vec{j}\). En déduire ses coordonnées.
\item Placer le point C tel que \(\overrightarrow{BC} =
           -\vec{i}\). En déduire ses coordonnées.
\item Tracer les vecteurs \(\vec{i}, \overrightarrow{IA},
           \overrightarrow{AB}, \overrightarrow{BC},
           \overrightarrow{CJ}, \overrightarrow{JA}\).

Vérifier que vous avez bien obtenu le chiffre 9.
\end{enumerate}

\section{Exercice 76}
\label{sec:org6c4d9a9}

Dans le plan muni du repère orthonormé canonique
\[(O ; \vec{i} ; \vec{j})\]

\begin{enumerate}
\item Placer le point A tel que \(\overrightarrow{IA} =
           \vec{j}\). En déduire ses coordonnées.
\item Placer le point B tel que \(\overrightarrow{AB} =
           \vec{j}\). En déduire ses coordonnées.
\item Placer le point C tel que \(\overrightarrow{BC} =
           -\vec{i}\). En déduire ses coordonnées.
\item Tracer les vecteurs \(\vec{i}, \vec{j}, \overrightarrow{IA},
           \overrightarrow{AB}, \overrightarrow{BC},
           \overrightarrow{CJ}\).

Vérifier que vous avez bien obtenu le chiffre 0.
\end{enumerate}

\chapter{Paradoxe de Simpson}
\label{sec:org346c191}
\section{Exercice 77}
\label{sec:org3628c29}

\begin{enumerate}
\item Placer les points O(0 ; 0) et A(1 ; 0) et tracer le vecteur
\[\vec{u}_1 = \overrightarrow{OA}\]
en bleu.
\item Placer le point B(3 ; 1) et tracer le vecteur
\[\vec{v}_1 = \overrightarrow{OB}\]
en rouge.
\item Placer les points C(0 ; 3) et D(2 ; 7) et tracer le
vecteur
\[\vec{u}_2 = \overrightarrow{CD}\]
en bleu.
\item Placer le point E(0 ; 4) et tracer le vecteur
\[\vec{v}_2 = \overrightarrow{CE}\]
en rouge.
\item Vérifier que le pente de \(\vec{u}_1\) est inférieure à celle de
\(\vec{v}_1\).
\item Vérifier que le pente de \(\vec{u}_2\) est inférieure à celle de
\(\vec{v}_2\).
\item Placer les points F(4 ; 2) et G(7 ; 4) et tracer le vecteur
\[\vec{v} = \vec{v}_1 + \vec{v}_2 = \overrightarrow{FG}\]
en rouge.
\item Placer le point H(7 ; 6) et tracer le vecteur
\[\vec{u} = \vec{u}_1 + \vec{u}_2 = \overrightarrow{FH}\]
en bleu.
\item Comparer les pentes des vecteurs \(\vec{u}\) et
\(\vec{v}\). Que remarquez-vous ?
\end{enumerate}

\chapter{Arithmétique}
\label{sec:org45024c2}
\section{Exercice 78}
\label{sec:orgbccc6ed}

On se place dans le plan muni d'un repère \((O ; \vec{i} ,
    \vec{j})\) avec \((\vec{i} , \vec{j})\) la base canonique.

\begin{enumerate}
\item Pour \(n\in\{1, 2, 3, 4, 5, 6\}\) tracer les 6 vecteurs
\[\vec{u}_n\begin{pmatrix}n\\ 6\end{pmatrix}\]
\item Calculer leurs pentes respectives \(p_n\).
\item Lesquelles sont des nombres entiers ?
\item Qu'en déduisez-vous pour les abscisses des vecteurs dont la
pente est un entier ?
\item Tracer les 6 vecteurs
\[\vec{v}_n\begin{pmatrix}6\\ n\end{pmatrix}\]
Que remarquez-vous pour \(\vec{v}_6\) ?
\item Calculer les pentes respectives \(q_n\).
\item Avez-vous eu besoin de les calculer à partir de zéro ou alors y
avait-il un moyen de les obtenir à partir des \(p_n\) ?
\item Quelle est la relation entre les \(p_n\) et les \(q_n\) ?
\item Combien y a-t-il de vecteurs du plan dans le cadrant supérieur
droit de coordonnées entières tels que leur produit vaut 6 ?
\item Tracer tous les vecteurs à coordonnées entières vérifiant les
conditions de la question précédente.
\end{enumerate}

\section{Exercice 79}
\label{sec:org0c88401}

On se place dans le plan muni d'un repère \((O ; \vec{i} ,
    \vec{j})\) avec \((\vec{i} , \vec{j})\) la base canonique.

\begin{enumerate}
\item Pour \(n\in\{1, 2, \dots , 12\}\) tracer les 12 vecteurs
\[\vec{u}_n\begin{pmatrix}n\\ 12\end{pmatrix}\]
\item Calculer leurs pentes respectives \(p_n\).
\item Lesquelles sont des nombres entiers ?
\item Qu'en déduisez-vous pour les abscisses des vecteurs dont la
pente est un entier ?
\item Tracer les 12 vecteurs
\[\vec{v}_n\begin{pmatrix}12\\ n\end{pmatrix}\]
Que remarquez-vous pour \(\vec{v}_12\) ?
\item Calculer les pentes respectives \(q_n\).
\item Avez-vous eu besoin de les calculer à partir de zéro ou alors y
avait-il un moyen de les obtenir à partir des \(p_n\) ?
\item Quelle est la relation entre les \(p_n\) et les \(q_n\) ?
\item Combien y a-t-il de vecteurs du plan dans le cadrant supérieur
droit de coordonnées entières tels que leur produit vaut 12 ?
\item Tracer tous les vecteurs à coordonnées entières vérifiant les
conditions de la question précédente.
\end{enumerate}

\section{Exercice 80}
\label{sec:org7b446e8}

On poursuit la même logique que les deux exercices précédents.

Mais cette fois on va calculer sans représenter les vecteurs parce
qu'on va prendre \(n = 60\).

\begin{enumerate}
\item Pour \(n\in\{1, 2, \dots , 60\}\) quels sont les vecteurs
\[\vec{u}_n\begin{pmatrix}n\\ 60\end{pmatrix}\]
qui ont une pente entière ?
\item Qu'en déduisez-vous pour les abscisses des vecteurs dont la
pente est un entier ?
\item Combien y a-t-il de vecteurs du plan dans le cadrant supérieur
droit de coordonnées entières tels que leur produit vaut 60 ?
\end{enumerate}

\chapter{Trigonométrie}
\label{sec:org81fd02f}
\section{Exercice 81}
\label{sec:orga3da12b}

Dans un repère orthonormé \((O ; \vec{i} ; \vec{j})\)

\begin{enumerate}
\item Placer le point A tel que \(\overrightarrow{OA} = \vec{u} =
       3\vec{i}\). Placer le point B tel que \(\overrightarrow{OB} =
       3(\vec{i} + \vec{j})\). Placer le point C tel que
\(\overrightarrow{OC} = 3\vec{i} + 4\vec{j}\). Placer le point
D tel que \(\overrightarrow{OD} = 4(\vec{i} +
       \vec{j}\). Comparer les triangles OIJ et BCD.
\item En géométrie classique l'angle AOB peut se lire dans les deux
sens avec la même valeur, c'est ce qu'on appelle l'angle
géométrique. Grâce aux vecteurs on peut définir un sens et donc
faire la différence entre un angle positif et un angle
négatif. On appelle sens positif (on dit aussi direct ou encore
trigonométrique) le sens inverse des aiguilles d'une montre (on
dit aussi anti-horaire) c'est-à-dire lorsqu'on parcours un
chemin de façon à ce que ce soit la main gauche qui tienne la
"rampe". Par exemple, l'angle orienté \((\vec{i} ; \vec{j})\)
est dans le sens direct donc on notera :
\[(\vec{i} ; \vec{j}) = 90\]
alors que l'angle \((\vec{j} ; \vec{i})\) est dans le sens
indirect (ou négatif) et on le notera :
\[(\vec{j} ; \vec{i}) = -90\]
Exprimer le cosinus de l'angle \((\overrightarrow{OA} ,
       \overrrightarrow{OB})\)
\item 
\end{enumerate}


\part{Erreurs fréquentes à éviter}
\label{sec:orge71d7a8}

\chapter{Erreur 1 : Confondre vecteur et longueur}
\label{sec:orgf10665c}
❌ \(\overrightarrow{AB} = 5\) 
✓ \(\lvert\lvert\overrightarrow{AB}\rvert\rvert = 5\) ou \(AB = 5\)

\chapter{Erreur 2 : Oublier le sens dans la relation de Chasles}
\label{sec:org251be92}
❌ \(\overrightarrow{AB} + \overrightarrow{CA} = \overrightarrow{CB}\)
✓ \(\overrightarrow{AB} + \overrightarrow{BC} = \overrightarrow{AC}\)

\chapter{Erreur 3 : Confondre coordonnées de point et de vecteur}
\label{sec:org1e22beb}
Si A(2;3) et B(5;7), alors :
❌ \(\overrightarrow{AB}\begin{pmatrix}5\\7\end{pmatrix}\)
✓ \(\overrightarrow{AB}\begin{pmatrix}3\\4\end{pmatrix}\)


\part{Comment obtenir les solutions ?}
\label{sec:org3f4a26e}

Les solutions détaillées de tous les exercices sont disponibles 
en téléchargement sur notre site web :

\textbf{\url{https://votresite.com/solutions-vecteurs}}

Indiquez simplement votre adresse email pour recevoir le PDF 
complet des corrigés.    


\part{Remerciements et feedback}
\label{sec:org7aade6d}

Merci d'avoir choisi ce livre ! Vos retours sont précieux pour 
améliorer les prochaines éditions.

Contactez-moi : laurent.garnier@votreemail.com
Site web : www.votresite.com

Si ce livre vous a aidé, n'hésitez pas à le recommander !   
\end{document}
